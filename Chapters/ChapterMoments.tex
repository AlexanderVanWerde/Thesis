\chapter{Moments of Holonomic Distributions}
In order to do mathematics one has to pick some class of objects which is sufficiently specific that one can make interesting observations but is also sufficiently general.
For instance, many real-world functions are analytic but representing a general analytic function requires an infinite amount of information.
Hence, the representation of general analytic functions is not possible inside of a computer.
On the other hand, a polynomial only has a finite amount of information and can be dealt with in a computer but these are not sufficiently general for most real-world applications.
It was observed by \cite{zeilberger1990holonomic} that the class of holonomic functions provides an acceptable compromise.
Many functions are holonomic and one only requires a finite amount of information to encode the corresponding differential equations.

An introduction to the literature on holonomic functions is provided in \cref{sec: HolAlgorithm}.
A class of distributions of probabilistic interest is discussed in \cref{sec: SubGSubE}.
We demonstrate how the holonomic toolbox may be applied to such probabilistic problems in \cref{sec: RecursionMoment}.
\section{Holonomic functions}\label{sec: HolAlgorithm}
Let $\k$ denote a field of characteristic zero.
The Weil algebra $D_n(\k)$ is the $\k$-algebra found from $\k[x_1,\ldots,x_n]$ found by adjoining new variables $\partial_1,\ldots,\partial_n$ subject to the usual commutation relations
$$\partial_i x_j = x_j \partial_i + \delta_{ij};\qquad \partial_i \partial_j = \partial_j \partial_i $$
where $\delta_{ij}$ denotes the Kronecker delta.
One has the corresponding notions of a order filtration, graded objects, characteristic varieties and holonomic $D_n(\k)$-modules precisely as in \cref{Ch: ChapterDX}.

An object which gives rise to a $D_n(\k)$-module is said to be holonomic over $\k$ precisely when the corresponding module is so.
For instance, a holonomic function over $\C$ is an analytic function $f:U\to \C$ on some open $U\subseteq \C^n$ such that $f$ satisfies a system of differential equations $P_{1}(x,\partial)f = 0,\ldots, P_{k}(x,\partial)f=0$ with $D_n(\C)/(P_1,\ldots,P_k)$ holonomic over $D_n(\C)$.
For a parameter-dependent function $f_s(x)$ one can speak of holonomicity over $\C(s_1,\ldots,s_r)$.
These notions also apply to distributions.
\begin{example}
  Let $f_s(x) = \exp(sx^2)$ on $\C$ then  $\partial_{x}f_s - sf_s = 0$.
  It follows that the characteristic variety is the zero section of $T^*\C(s)$.
  Hence, the parameter-dependent function $f_s(x)$ is holonomic over $\C(s)$.

  More generally, for any polynomial $p$ on $\C^n$ with undetermined coefficients $s_1,\ldots,s_r$ it holds that $f_s = \exp(p(x))$ is holonomic over $\C(s_1,\ldots,s_r)$.
\end{example}
\begin{example}
  Consider a Dirac delta $\delta_{s}$ at the point $(s_1,\ldots,s_n)$ in $\C^n$.
  Then $\delta_{s}\cdot (x_i - s_i) = 0$ for all $i=1,\ldots,n$ whence it follows that $\delta_s$ is holonomic over $\C(s_1,\ldots,s_n)$.
\end{example}
In the foregoing chapter we had a notion of relative holonomicity over $\C[s]$ with $s = (s_1,\ldots,s_r)$.
One can go from relative holonomicity to holonomicity over $\C(s)$ by use of the following result.
\begin{lemma}
    For any $\D_{\C^n}[s]$-module $\M$ with $\dim\Chrel\M\leq n+r$ it holds that the global sections of $\M\otimes_{\C[s]}\C(s)$ form a holonomic $D_n(\C(s))$-module.
  \end{lemma}
  \begin{proof}
    Consider a good filtration on $\M$ and equip $\M\otimes_{\C[s]}\C(s)$ with the induced filtration.
    Since localisation is an exact functor it holds that $\grrel\M\otimes_{\C[s]}\C(s) \cong \gr(\M\otimes_{\C[s]}\C(s)) $.

    Suppose that $\Ch(\M\otimes_{\C[s]}\C(s))$ has dimension strictly greater than $n$ as a variety over $\C(s)$.
    Let $\mathfrak{m}$ be a maximal ideal of the coordinate ring of $\Ch (\M\otimes_{\C[s]}\C(s))$.
    The maximal ideal $\mathfrak{m}$ corresponds to a prime ideal $\mathfrak{p}$ of the coordinate ring of $\Chrel\M$ which does not intersect $\C[s]\setminus\{0\}$.
    Moreover it follows from the assumption that  $\dim\Ch(\M\otimes_{\C[s]}\C(s))> n$ that $\mathfrak{p}$ has height $>n$.

    Now the subvariety $V = Z(\mathfrak{p})$ of $\Chrel\M$ has codimension $>n$ and is not contained in any set of the form $Z(b(s))$ for $b(s)\in \C[s]\setminus\{0\}$.
    Since $\dim\Chrel\M\leq n+s$ it follows that $\dim V <r$.

    Denote $\pi:\C^{n+r}\to\C^r $ for the projection map and observe that $\dim\operatorname{cl}\pi(V)\leq \dim(V) < r$ where $\operatorname{cl}$ denotes the closure in the Zariski topology. %\footnote{\url{https://math.stackexchange.com/q/95794 }}
    This contradicts the assumption that $V\nsubseteq Z(b(s))$ for any $b(s)\in \C[s]\setminus\{0\}$ and we conclude that $\dim \Ch (\M\otimes_{\C[s]}\C(s)) \leq n$.
    \end{proof}
    \subsection{Closure properties}
    The class of holonomic functions is closed under the usual operations.
    These closure properties are typically effective.
    This means that there are algorithms to compute the relations over $D_n(\k)$ satisfied by the output of the operation.
    Such algorithms often rely on the theory of Gr\"obner bases and have been implemented in software packages such as SINGULAR or Mathematica.
    \begin{theorem}{\cite[Proposition 3.1]{zeilberger1990holonomic}}
      Let $f,g$ be holonomic functions over $\k$.
      If $f+g$ is defined it is also holonomic over $\k$.
    \end{theorem}
    \begin{theorem}{\cite[Proposition 3.2]{zeilberger1990holonomic}}
      Let $f,g$ be holonomic functions over $\k$ such that $f\cdot g$ is defined.
      If $f\cdot g$ is defined it is also holonomic over $\k$.
      When $\k=\R$, $\k = \C$ or $\k = \C(s)$ the statement also holds if $g$ is replaced by a distribution.
    \end{theorem}
    \begin{theorem}{\cite[Theorem 2.7]{stanley1980differentiably}}
      Let $f$ and $a$ be monovariate holonomic functions and algebraic functions over $\k$ respectively.
      If $f\circ a$ is defined it is also holonomic over $\k$.
    \end{theorem}
\section{Subgaussian random variables}\label{sec: SubGSubE}
\section{Recursion on moments}\label{sec: RecursionMoment}
Let $\mu$ be a fixed holonomic distribution on $\R^n$ and assume that the moments
$$m_\alpha = \int_{\R^n} x^\alpha d\mu$$
exist for all $\alpha \in \mathbb{Z}_{\geq 0}^n$.

From here on out we will assume that $n = 1$ which is the main case of interest in the probabilistic context.
\begin{proposition}
  Suppose that $\mu$ is annihilated by $P \in \D_{\C}^n$. If $P = \sum_{ij} c_{ij}x^i \partial^j$ then it holds that
  $$ \sum_{ij} c_{ij} (n+ i)(n+ i -1)\cdots (n+i-j + 1) m_{n + i - j} = 0$$
  for any $n \geq 0$.
\end{proposition}

%Similar observations have been made in \cite{brehard2019moment} whose main focus is on distributions of the form $\mu = \exp(p(x))1_{G}(x)dx$ with $p(x)$ a multivariate polynomial and $G$ a semi-algebraic set.
%There it is discussed when it is possible to recover the coefficients of $p(x)$ and some polynomial $g(x)$ vanishing on the boundary of $G$ given a finite number of moments $m_n = \int x^n d\mu$.

%We will not address this reconstruction problem and instead focus on the asymptotic rate of growth of the moments.
%This problem is of interest in non-asymptotic probability theory.
%For non-asymptotic concentration results it is often necessary to assume that the tails of the probability distribution decrease sufficiently rapidly.
%This gives rise to the notions of subgaussian and subexponential random variables.
%The goal of this chapter is to provide a easy way to verify whether a given holonomic distribution satisfies these conditions.
