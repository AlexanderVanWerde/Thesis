\chapter{Moments of holonomic distributions}
It was observed by \cite{zeilberger1990holonomic} that the class of holonomic functions provide an good class of functions for practical use.
Many real-world functions are holonomic and only a finite amount of information is required to encode the corresponding differential equations.

It has been observed by \cite{bitoun2019feynman} that the holonomic toolset can be used to reduce the computations stemming from Feynman integrals in certain problems of quantum field theory.
Holonomicity has also been used to investigate the statistical problem of estimating the parameters of the distribution based on a finite number of moments.
This problem was investigated by \cite{batenkov2009moment} for holonomic distributions on a compact interval $[a,b]\subseteq \mathbb{R}$ and by \cite{brehard2019moment} for holonomic distributions on a compact semi-algebraic subset $G\subseteq \mathbb{R}^n$.

The goal of this chapter is to investigate possible applications of the holonomic toolset to theoretical problems from probability theory.
These problems are mainly interesting for unbounded random variables so it is necessary to remove the compactness assumptions of \cite{batenkov2009moment} and \cite{brehard2019moment}.

Preliminaries with on the theory of distributions are provided in \cref{sec: Distributions}.
An introduction to the literature on holonomic functions is provided in \cref{sec: HolAlgorithm}.
A class of distributions of probabilistic interest is discussed in \cref{sec: SubGSubE}.
Finally, we demonstrate how the holonomic toolbox may be applied to probabilistic problems in \cref{sec: RecursionMoment}.
\section{Distributions}\label{sec: Distributions}
This section provides some basic definitions regarding distributions.
An in-depth treatment of the theory may be found in \cite{grubb2008distributions}.

For any subset $S\subseteq \mathbb{R}^n$ denote $\mathcal{C}_c^\infty(S)$ for the space of smooth real functions on $S$ with compact support.
For any compact set $K\subseteq \mathbb{R}^n$ equip $\mathcal{C}^\infty_c(K)$ with the topology generated by the seminorms
$$p_{\alpha}(f) = \sup_{x\in K} \abs{\partial^\alpha f(x)};\qquad \alpha\in \mathbb{Z}_{\geq 0}^n.$$
Equip $\mathcal{C}_c^\infty(\mathbb{R}^n)$ with the final topology for the system of inclusions $\mathcal{C}_c^\infty(K)\subseteq \mathcal{C}_c^\infty(\mathbb{R}^n)$ where $K$ runs over all compact subsets.
\begin{definition}\label{def: Distribution}
  A distribution on $\mathbb{R}^n$ is a continuous linear functional on $\mathcal{C}_c^\infty(\mathbb{R}^n)$.
  The space of distributions $\mathcal{D}(\mathbb{R}^n)$ is equipped with the corresponding $\text{weak}^*$-topology which is defined by the seminorms $p_{f}(u) = \abs{u(f)}$ where $f\in \mathcal{C}_c^\infty(\mathbb{R}^n).$
\end{definition}
The following definitions are not a part of the standard literature on distributions but will be convenient for our purposes.
\begin{definition}
  The space of slowly increasing functions $\mathcal{S}(\mathbb{R}^n)$ consists of all functions $f\in \mathcal{C}^\infty(\mathbb{R}^n)$ such that for any $\alpha\in \mathbb{Z}_{\geq 0}^n$ there exists some polynomial $p\in \mathbb{R}[x]$ with $\abs{\partial^\alpha f(x)}\leq \abs{p(x)}$ for all $x\in\mathbb{R}^n$.

  This space is equipped with the topology defined by the seminorms $$p_{\alpha}(f) = \sup_{x\in \mathbb{R}^n} \vert\exp(-\textstyle\sum_{i=1}^n x_i^2)\partial^\alpha f(x)\vert ;\qquad \alpha\in \mathbb{Z}_{\geq 0}^n.$$
\end{definition}
\begin{definition}\label{def: DistributionWithMoments}
  A distribution with moments is a continuous linear functional on $\mathcal{S}(\mathbb{R}^n)$.
  The space of distributions with moments $\mathcal{M}(\mathbb{R}^n)$ is equipped with the corresponding $\text{weak}^*$-topology which is defined by the seminorms  $p_{f}(u) = \abs{u(f)}$ where $f\in \mathcal{S}(\mathbb{R}^n).$
\end{definition}
A-priori it is not guaranteed that a distribution with moments corresponds to a distribution in the sense of \cref{def: Distribution}.
The following lemma shows that this is indeed the case and that the element of $\mathcal{M}(\mathbb{R}^n)$ is uniquely determined by the corresponding distribution.
\begin{lemma}
  The restriction $\rho: \mathcal{M}(\mathbb{R}^n)\to \mathcal{D}(\mathbb{R}^n)$
  is a well-defined injective morphism of topological vector spaces.
\end{lemma}
\begin{proof}
   For the well-definedness we must show that for any $u\in \mathcal{M}(\mathbb{R}^n)$ the restriction $\rho(u)$ defines a continuous functional on $\mathcal{C}_c^\infty(\mathbb{R}^n)$.
   By the continuity of $u$ and the definition of the topology on $\mathcal{C}_c^\infty(\mathbb{R}^n)$ it is sufficient to show that for every basis-open $B\subseteq \mathcal{S}(\mathbb{R}^n)$ and compact $K\subseteq \mathbb{R}^n$ it holds that $B\cap \mathcal{C}_c^\infty(K)$ is an open subset of $\mathcal{C}_c^\infty(K)$.
   Let $f\in B\cap \mathcal{C}_c^\infty(K)$, by definition of the topology on $\mathcal{M}(\mathbb{R}^n)$ there exist
   $\alpha_1,\ldots,\alpha_k\in \mathbb{Z}_{\geq 0}^n$ and $\delta >0$ such that for any $g\in\mathcal{S}(\mathbb{R}^n)$ it holds that $g\in U$ whenever $\abs{p_{\alpha_j}(f-g)}<\delta$ for all $j=1,\ldots,k$ with $p_{\alpha_j}$ the seminorms from \Cref{def: DistributionWithMoments}.
   Take $C>0$ such that $\exp(-\sum_{i=1}^n x_i^2)>C$ for all $x\in K$ and conclude that the open set determined by the conditions $\abs{p_{\alpha_j}(f-g)}<C\delta$ with $p_{\alpha_j}$ the seminorms defining the topology on $\mathcal{C}_c^\infty(K)$ define an open neighbourhood of $f$ in $B\cap\mathcal{C}_c^\infty(K)$.

   For the injectivity it must be shown that any $u\in \mathcal{M}(\R^n)$ is entirely determined by it's values on $\mathcal{C}_c^\infty(\mathbb{R}^n)$.
   Pick some $f\in \mathcal{S}(\mathbb{R}^n)$.
   For any $j\geq 0$ it is possible to find some bump function $g_j\in \mathcal{C}_c^\infty(\mathbb{R}^n)$ with $\abs{\partial^\alpha g_j(x)}<1/j$ for all $x\in\mathbb{R}^n$ and $\abs{\alpha} <j$ and $g_j(x) = 1$ for all $x\in \mathbb{R}^n$ with $\norm{x}<j$.
   It follows that for every $\alpha\in \mathbb{Z}_{\geq 0}^n$ the value $p_\alpha(g_jf - f)$ tends to zero as $j$ tends to infinity.
   By definition of the topology on $\mathbb{S}(\mathbb{R}^n)$ this means that $f$ is the limit of the sequence of functions $g_jf\in \mathcal{C}_c^\infty(\mathbb{R}^n)$.
   By the continuity of $u$ it follows that $\lim_{j\to \infty} u(g_jf) = u(f)$ which proves the desired result.

   The fact that $\rho$ is continuous is immediate from the definitions of the respective $\text{weak}^*$-topologies.
   Pick some basis-open $B\subseteq \mathcal{D}(\mathbb{R}^n)$ centered on $u$
   By definition of the $\text{weak}^*$ topology there exist $f_j\in \mathcal{C}_c^\infty(\R^n)$, $\alpha_j\in \mathbb{Z}^n_{\geq 0}$ and  $\varepsilon_j >0$ for $j=1,\ldots, k$ such that
   $$B = \cap_{j=1}^k \{h\in \mathcal{D}(\R^n): \abs{h(f_j)-u(f_j)} <\varepsilon_j\}. $$
   Now observe that
   $$\rho^{-1}(B) = \cap_{j=1}^k \{h\in \mathcal{M}(\R^n): \abs{h(f_j)-u(f_j)} <\varepsilon_j\}$$
   is a basis open for the $\text{weak}^*$-topology on $\mathcal{M}(\mathbb{R}^n)$.
   This concludes the proof.
\end{proof}
The distributions with moments $\mathcal{M}(\mathbb{R}^n)$ come equipped with a right $\D_{\mathbb{R}^n}(\mathbb{R}^n)$-module structure when $\mathbb{R}^n$ is considered as an algebraic variety.
Indeed, note that $\partial_i f\in \mathcal{S}(\R^n)$ and $gf\in \mathcal{S}(\R[x])$ whenever $f\in \mathbb{R}$ and $g\in \R[x]$.
The right $\D_{\mathbb{R}^n}(\mathbb{R}^n)$-module structure on $\mathcal{M}(\mathbb{R}^n)$ is now defined by
$$(u\cdot \partial_i)(f) := -u(\partial_i f); \qquad (u\cdot g)(f) = u(gf)$$
with $f,g$ as above.
A similar right $\D_{\mathbb{R}^n}(\mathbb{R}^n)$-module structure applies to $\mathcal{D}(\R^n)$.


%The following notions lead to a space of distributions for which Fourier analysis is applicable.
%The space of rapidly decreasing functions $\mathscr{S}(\mathbb{R}^n)$ consists of all functions $f\in \mathcal{C}^\infty(\mathbb{R}^n)$ such that $x^\alpha\partial^\beta f$ is bounded for all multi-indices $\alpha,\beta \in \mathbb{Z}_{\geq 0}^n$.
%One equips $\mathscr{S}(\mathbb{R}^n)$ with the topology generated by the seminorms
%$$p_{j}(f):= \sup_{x\in \mathbb{R}^n}\{\abs{x^\alpha \partial^\beta f}: \abs{\alpha},\abs{\beta} \leq j\}.$$
%\begin{definition}
%  A temperate distribution is a continuous linear functional on $\mathscr{S}(\mathbb{R}^n)$.
%  The space of temperate distributions $\mathscr{S}'(\mathbb{R}^n)$ is equipped with the corresponding $\text{weak}^*$-topology which is defined by the seminorms
%  $$p_{f}(u) = \abs{u(f)};\qquad f\in \mathscr{S}(\mathbb{R}^n).$$
%\end{definition}

%The following definitions are not in the standard literature of distribution theory but will be convenient for our purposes.
%The space of slowly increasing functions $\mathcal{S}(\mathbb{R}^n)$ consists of all functions $f\in \mathcal{C}^\infty(\mathbb{R}^n)$ such that for any multi-index $\alpha\in \mathbb{Z}_{\geq 0}^n$ there exists some multi-index $\gamma \in \mathbb{Z}_{\geq 0}^n$ and constant $C\in \mathbb{R}_{> 0}$ with $\abs{\partial^\alpha f} \leq c(1 + x^\gamma)$.
%\begin{definition}
%  A moderate distribution is a linear functional on $\mathcal{S}(\mathbb{R}^n)$ which is continuous when restricted to
%\end{definition}

\section{Holonomic functions}\label{sec: HolAlgorithm}
Let $K$ denote a field of characteristic zero.
The Weyl algebra $D_n(K)$ is the $K$-algebra found from $K[x_1,\ldots,x_n]$ found by adjoining new variables $\partial_1,\ldots,\partial_n$ subject to the usual commutation relations
$$\partial_i x_j = x_j \partial_i + \delta_{ij};\qquad \partial_i \partial_j = \partial_j \partial_i $$
where $\delta_{ij}$ denotes the Kronecker delta.
One has the corresponding notions of a order filtration, graded objects, characteristic varieties and holonomic $D_n(K)$-modules precisely as in \cref{Ch: ChapterDX}.

An object which gives rise to a $D_n(K)$-module is said to be holonomic over $K$ precisely when the corresponding module is so.
For instance, a holonomic function over $\C$ is an analytic function $f:U\to \C$ on some open $U\subseteq \C^n$ such that $f$ satisfies a system of differential equations $P_{1}(x,\partial)f = 0,\ldots, P_{k}(x,\partial)f=0$ with $D_n(\C)/(P_1,\ldots,P_k)$ holonomic over $D_n(\C)$.
For a parameter-dependent function $f_s(x)$ one can speak of holonomicity over $\C(s_1,\ldots,s_r)$.
Similar notions apply to distributions.
\begin{example}
  Let $f_s(x) = \exp(sx^2)$ on $\C$ then  $\partial_{x}f_s - sf_s = 0$.
  It follows that the characteristic variety is the zero section of $T^*\C(s)$.
  Hence, the parameter-dependent function $f_s(x)$ is holonomic over $\C(s)$.

  More generally, for any polynomial $p$ on $\C^n$ with undetermined coefficients $s_1,\ldots,s_r$ it holds that $f_s = \exp(p(x))$ is holonomic over $\C(s_1,\ldots,s_r)$.
\end{example}
\begin{example}
  The parameter-dependent distribution $f_{s_1,s_2} = \chi_{[s_1,s_2]}dx$ on $\mathbb{R}$ with $\chi_{[s_1,s_2]}$ the indicator function is holonomic over $\R(s_1,s_2)$.
  Indeed, $\chi_{[s_1,s_2]}dx\cdot \partial_x = \delta{s_1} -\delta_{s_2}$ with $\delta$ the Dirac delta.
  Hence, $f_{s_1,s_2}$ is annihilated by $\partial_x(x-s_2)(x-s_1)$ from which the holonomicity follows.
\end{example}
In the foregoing chapter we had a notion of relative holonomicity over $\C[s]$ with $s = (s_1,\ldots,s_r)$.
One can go from relative holonomicity to holonomicity over $\C(s)$ by use of the following result.
\begin{lemma}
    For any $\D_{\C^n}[s]$-module $\M$ with $\dim\Chrel\M\leq n+r$ it holds that the global sections of $\M\otimes_{\C[s]}\C(s)$ form a holonomic $D_n(\C(s))$-module.
\end{lemma}
  \begin{proof}
    Consider a good filtration on $\M$ and equip $\M\otimes_{\C[s]}\C(s)$ with the induced filtration.
    Since localisation is an exact functor it holds that $\grrel\M\otimes_{\C[s]}\C(s) \cong \gr(\M\otimes_{\C[s]}\C(s)) $.

    Suppose that $\Ch(\M\otimes_{\C[s]}\C(s))$ has dimension strictly greater than $n$ as a variety over $\C(s)$.
    Let $\mathfrak{m}$ be a maximal ideal of the coordinate ring of $\Ch (\M\otimes_{\C[s]}\C(s))$.
    The maximal ideal $\mathfrak{m}$ corresponds to a prime ideal $\mathfrak{p}$ of the coordinate ring of $\Chrel\M$ which does not intersect $\C[s]\setminus\{0\}$.
    Moreover it follows from the assumption that  $\dim\Ch(\M\otimes_{\C[s]}\C(s))> n$ that $\mathfrak{p}$ has height $>n$.

    Now the subvariety $V = Z(\mathfrak{p})$ of $\Chrel\M$ has codimension $>n$ and is not contained in any set of the form $Z(b(s))$ for $b(s)\in \C[s]\setminus\{0\}$.
    Since $\dim\Chrel\M\leq n+s$ it follows that $\dim V <r$.

    Denote $\pi:\C^{n+r}\to\C^r $ for the projection map and observe that $\dim\operatorname{cl}\pi(V)\leq \dim(V) < r$ where $\operatorname{cl}$ denotes the closure in the Zariski topology. %\footnote{\url{https://math.stackexchange.com/q/95794 }}
    This contradicts the assumption that $V\nsubseteq Z(b(s))$ for any $b(s)\in \C[s]\setminus\{0\}$ and we conclude that $\dim \Ch (\M\otimes_{\C[s]}\C(s)) \leq n$.
    \end{proof}
    \subsection{Closure properties}
    The class of holonomic functions is closed under the usual operations.
    These closure properties are typically effective.
    This means that there are algorithms to compute the relations over $D_n(K)$ satisfied by the output of the operation.
    Such algorithms often rely on the theory of Gr\"obner bases and have been implemented in software packages such as SINGULAR or Mathematica.

    \begin{theorem}{\cite[Proposition 3.1]{zeilberger1990holonomic}}
      Let $f,g$ be holonomic functions or distributions over $K$.
      If $f+g$ is defined it is also holonomic over $K$.
    \end{theorem}
    \begin{theorem}{\cite[Proposition 3.2]{zeilberger1990holonomic}}
      Let $f,g$ be holonomic functions over $K$.
      If $f\cdot g$ is defined it is also holonomic over $K$.
      The same result holds if $g$ is a holonomic distribution.
    \end{theorem}
    \begin{theorem}{\cite[Corollary 1]{harris1985reciprocals}}
      Let $f$ be a monovariate holonomic function over $K$.
      If $1/f$ is defined it is holonomic over $K$ if and only if the logarithmic derivative $\partial f/f$ is algebraic over $K$.
    \end{theorem}
    \begin{theorem}{\cite[Theorem 2.7]{stanley1980differentiably}}\label{thm: AlgebraicPrecomp}
      Let $f$ and $a$ be a monovariate holonomic function and a monovariate algebraic functions over $K$ respectively.
      If $f\circ a$ is defined it is also holonomic over $K$.
    \end{theorem}
    \subsection{Asymptotics of holonomic sequences}
    It also makes sense to speak about holonomic sequences.
    In this case it is convenient to replace the notation $x_i,\partial_i$ by $m_i,S_i$ respectively.
    The Weyl algebra $D_n(\k)$ acts on multivariate sequences $u:\mathbb{Z}_{\geq 0}^n \to K$ by $(m_iu)(m):= m_i u(m)$ and $(S_iu)(m) = u(m-e_i)$ with $e_i$ the $i$-th standard basis vector.

    This section is concerned with the asymptotics of monovariate sequences $u:\mathbb{Z}_{\geq 0}\to K$.
    The statement that $u(m) = O(b(m))$ for some positive sequence $b:\mathbb{Z}_{\geq 0}\to \mathbb{R}_{\geq 0}$ means that there exists some $C,M\geq 0$ such that for all $m\geq m$ it holds that $\abs{u(m)}\leq Cb(m)$.
    \begin{theorem}{\cite[Theorem 1.5]{stanley1980differentiably}}\label{thm: GeneratingFunction}
      A sequence $u:\mathbb{Z}_{\geq 0}\to K$ is holonomic over $K$ if and only if its generating sequence $f(z) = \sum_{m=0}^{\infty} u(m)z^m$ is so.
    \end{theorem}
    A sequence $u:\mathbb{Z}_{\geq 0}\to K$ is holonomic if and only if there exists some homogeneous linear recursion relation
    $$p_d(m)u(m+d) + p_{d-1}(m)u(m+d-1) + \cdots + p_0(m)u(m) = 0 $$
    with polynomial coefficients $p_k \in K[m]$.
    A holonomic sequence is called $K$-recurrent if the polynomials $p_k$ can be taken to be constants.
    If the $K$-recurrent sequence $u$ has recurrence relation $\sum_{j=0}^d c_{j}u(m+j)= 0$ with $d$ minimal and $c_d = 1$ one calls $\sum_{j=0}^d c_jx^j\in K[x]$ the minimal polynomial of $u$.
    In this case the asymptotics of the recursions are well-understood, see for instance \cite[Chapter 2]{everest2003recurrence}, in particular the structure of the solutions to such recursions is explicitly known.
    \begin{theorem}{\cite[Theorem 1.6]{nobleAsymptotics}}
      Let $u:\mathbb{Z}_{\geq 0}$ be a $K$-recurrent sequence and denote $\lambda_1,\ldots,\lambda_k$ for the distinct roots of the minimal polynomial of $u$.
      Then there exist polynomials $P_1,\ldots,P_k \in \overline{K}[n]$ such that
      $$u(m) = \sum_{j=1}^k P_j(m)\lambda_j^m$$
      for all $m\in \mathbb{Z}_{\geq 0}$.
    \end{theorem}
    The case of holonomic sequences is more subtle but an asymptotic expansion in the case with $K = \C$ has been proved by \cite{birkhoff1933analytic}.
    It should be noted that there are allegedly gaps in the complicated proof.
    Still, the results have so far matched up with reality.
    Algorithms to compute the expansions have been implemented by \cite{kauers2011mathematica} and \cite{zeilberger1990holonomic}.
    In particular, from the asymptotic expansion the following theorem follows.
    \begin{theorem}(\cite{birkhoff1933analytic})
      Suppose that $u:\mathbb{Z}_{\geq 0}\to \C$ is holonomic. Then there exist constants $c,\alpha, \beta, \gamma \in \C$, positive integers $r,k\in \mathbb{Z}_{\geq 1}$ and a polynomial $p\in \C[x]$ such that
      $$u(m) = O\left(cm^{\alpha m} e^{p(m^{1/r})}\beta^m m^\gamma \log(m)^{k-1}\right).$$
    \end{theorem}
    Another approach in the case with $K = \C$ is due to \cite{flajolet1990singularity}.
    Due to the algorithms associated to \Cref{thm: GeneratingFunction} one gets an implicit description of the generating function $f$ of $u$.
    Suppose that $f$ defines an analytic function near the origin and assume that $f$ has a unique singularity of minimal modulus.
    By renormalization it may be assumed that this singularity occurs at $1$.
    For any $r>1$ and $\theta \in (0,\pi/2)$  consider the closed indented disk
    $$\Delta(r,\theta) := \{z\in \mathbb{C}: \abs{z}\leq r, \quad \abs{\operatorname{Arg}(z-1)}\geq \theta\}.$$
    \begin{theorem}{\cite[Corollary 3]{flajolet1990singularity}}
      With conventions as above, suppose that $f(z)$ is analytic in some domain $\Delta(r,\theta)$ and assume that as $z$ tends to $1$
      $$f(z) = \sum_{j=1}^k c_{j}(1-z)^{\alpha_j} + O(\abs{1-z}^A)$$
      for $c_j,\alpha_j,A\in \C$ with $\operatorname{Re}(\alpha_0)\leq \cdots \leq \operatorname{Re}(\alpha_k)<A$.
      Then, as $m$ tends to infinity
      $$u(m) = \sum_{j=1}c_j m^{-\alpha_j' - 1} + O(m^{-A -1}).$$
    \end{theorem}
    Other methods based on generating functions are also known, for instance due to \cite{hayman1956generalisation}.
    Algorithms to determine the asymptotics of $u$ based on the generating function have been implemented by \cite{salvy1991examples}.
\section{Tail condition for random variables}\label{sec: SubGSubE}
In probability and statistics it is often necessary to have some assumption regarding the concentration of a random variable.
For instance, the Cauchy distribution can be hard to understand because of it's large tails.
These large tails cause pathological behaviour such as undefined variance and expected value.

\begin{theorem}{\cite[Proof of Proposition 2.5.2]{vershynin2018high}}\label{thm: EquivalentLp}
  Let $X$ be a real random variable and let $p\in \mathbb{R}_{>0}$. The following are equivalent
  \begin{enumerate}[label = (\roman*)]
    \item There exists some $R_1\in \mathbb{R}_{>0}$ such that $\mathbb{P}(\abs{X}\geq t) \leq 2\exp(-t^p/R_1^p)$
    \item There exists some $R_2\in \mathbb{R}_{>0}$ such that $\mathbb{E}\exp(\abs{X}^p/R_2^p) \leq 2.$
    \item There exists some constant $R_3\in \mathbb{R}_{>0}$ such that $\left(\mathbb{E} \abs{X}^k \right)^{1/k} \leq R_3k^{1/p}$
    for all $k\in \mathbb{Z}_{\geq 0}$.
  \end{enumerate}
  Moreover, the constants $R_i$ which occur in these equivalent conditions can differ by at most an absolute constant factor.
\end{theorem}
\begin{definition}
  Let $p\in \mathbb{R}_{\geq 1}$.
  A random variable $X$ belongs to $L_{\psi_p}$ if it satisfies any of the equivalent conditions of \Cref{thm: EquivalentLp}.
  In this case the minimal constant $R_2$ which satisfies part (ii) of \Cref{thm: EquivalentLp} is denoted $\norm{X}_{\psi_p}$.
\end{definition}
\begin{remark}
  One can show that, whenever $p\geq 1$, $(L_{\psi_p},\norm{\cdot}_p)$ is a normed vector space.
  The variables in $L_{\psi_1}$ are called sub-exponential and the variables in $L_{\psi_2}$ are called sub-gaussian.
\end{remark}
Consider the following theorem as an example of the usefulness of these assumptions.
\begin{theorem}{\cite[Theorem 4.4.5]{vershynin2018high}}
  There exists some constant $C>0$ such that for any  $m\times n$ random matrix $A$ with independent sub-gaussian entries $A_{ij}$ satisfying $\mathbb{E}A_{ij} = 0$, $\mathbb{E}A_{ij}^2 = 1$ and $\norm{A_{ij}}_{\psi_2}\leq R$
  it hold that
  $$\mathbb{P}\left(\norm{A} \leq  CR(\sqrt{m} + \sqrt{n} + t)\right) \geq 1-2\exp(-t^2)$$
  for all $t>0$.
  Here $\norm{A}$ denotes the operator norm of $A$.
\end{theorem}

\section{Moments of holonomic distributions}\label{sec: RecursionMoment}

\section{Recursion in }
