\chapter{Moments of holonomic distributions}
In probability and statistics it is often necessary to have some assumption regarding the tails of a random variable.
For instance, the Cauchy distribution can be hard to understand because of it's large tails.
These large tails cause pathological behaviour such as undefined expected value.

This chapter is a proof-of-concept that it may be possible to use holonomicity to study the tails of random variables.
It should be emphasised that it is not meant that holonomicity is always the best method to control the tails.
In many cases the tails can be investigated by a direct study of the decay of the distribution function.
Still, holonomicity may provide a useful criterion for theoretical investigations.
By consideration of distribution functions our approach allows for an uniform treatment of discrete and continuous random variables.
One could further hope that one can find algorithms for holonomic random variables by exploiting the algorithms which are available for holonomic distributions.

Preliminaries with on the theory of distributions are provided in \cref{sec: Distributions}.
An introduction to the literature on holonomic functions is provided in \cref{sec: HolAlgorithm}.
We demonstrate how the holonomic toolbox may be applied to probabilistic problems in \cref{sec: RecursionMoment}.
\section{Distributions}\label{sec: Distributions}
For any subset $S\subseteq \mathbb{R}^n$ denote $\mathcal{C}_c^\infty(S)$ for the space of smooth real functions on $S$ with compact support.
For any compact set $K\subseteq \mathbb{R}^n$ equip $\mathcal{C}^\infty_c(K)$ with the topology generated by the seminorms
$$p_{K,\alpha}(f) = \sup_{x\in K} \abs{\partial^\alpha f(x)};\qquad \alpha\in \mathbb{Z}_{\geq 0}^n.$$
Equip $\mathcal{C}_c^\infty(\mathbb{R}^n)$ with the final topology for the system of inclusions $\mathcal{C}_c^\infty(K)\subseteq \mathcal{C}_c^\infty(\mathbb{R}^n)$ where $K$ runs over all compact subsets.
\begin{definition}\label{def: Distribution}
  A distribution on $\mathbb{R}^n$ is a continuous linear functional on $\mathcal{C}_c^\infty(\mathbb{R}^n)$.
  The space of distributions $\mathcal{D}(\mathbb{R}^n)$ is equipped with the corresponding $\text{weak}^*$-topology defined by the seminorms $p_{f}(w) = \abs{w(f)}$ where $f\in \mathcal{C}_c^\infty(\mathbb{R}^n).$
\end{definition}
Let us warn that the following terminology is not standard in distribution theory but will be convenient for our purposes.
The reader is referred to \cite[Chapter 4]{horvath2012topological} for some related notions.
\begin{definition}
  The space of functions with subpolynomial derivatives $\mathcal{P}(\mathbb{R}^n)$ consists of all functions $f\in \mathcal{C}^\infty(\mathbb{R}^n)$ such that for any $\alpha\in \mathbb{Z}_{\geq 0}^n$ there exists some polynomial $p\in \mathbb{R}[x]$ with $\abs{\partial^\alpha f(x)}\leq \abs{p(x)}$ for all $x\in\mathbb{R}^n$.
  This space is equipped with the topology defined by the seminorms $$p_{g,\alpha}(f) = \sup_{x\in \mathbb{R}^n} \vert g(x)\partial^\alpha f(x)\vert ;\qquad \alpha\in \mathbb{Z}_{\geq 0}^n$$
  where $g$ runs over all non-vanishing smooth functions with $\limsup_{x\to \infty} \abs{g(x)p(x)} = 0$ for any polynomial $p\in \mathbb{R}[x]$.
\end{definition}
\begin{definition}\label{def: DistributionWithMoments}
  A distribution with finite moments is a continuous linear functional on $\mathcal{P}(\mathbb{R}^n)$.
  The space of distributions with finite moments $\mathcal{M}(\mathbb{R}^n)$ is equipped with the corresponding $\text{weak}^*$-topology defined by the seminorms  $p_{f}(w) = \abs{w(f)}$ where $f\in \mathcal{P}(\mathbb{R}^n).$
\end{definition}
A-priori it is not guaranteed that a distribution with finite moments actually corresponds to a distribution in the sense of \cref{def: Distribution}.
The following lemma shows that this is indeed the case and that the element of $\mathcal{M}(\mathbb{R}^n)$ is uniquely determined by the corresponding distribution.
\begin{lemma}
  The restriction $\rho: \mathcal{M}(\mathbb{R}^n)\to \mathcal{D}(\mathbb{R}^n)$
  is a well-defined injective morphism of topological vector spaces.
\end{lemma}
\begin{proof}
   For the well-definedness it must be shown that $\rho(w)$ defines a continuous functional on $\mathcal{C}_c^\infty(\mathbb{R}^n)$ for any $w\in \mathcal{M}(\mathbb{R}^n)$.
   By the continuity of $w$ and the definition of the topology on $\mathcal{C}_c^\infty(\mathbb{R}^n)$ it is sufficient to show that for every open $V\subseteq \mathcal{P}(\mathbb{R}^n)$ and compact $K\subseteq \mathbb{R}^n$ it holds that $V\cap \mathcal{C}_c^\infty(K)$ is an open subset of $\mathcal{C}_c^\infty(K)$.
   Let $f\in V\cap \mathcal{C}_c^\infty(K)$, by definition of the topology on $\mathcal{M}(\mathbb{R}^n)$ there exist seminorms $p_{g_1,\alpha_1},\ldots,p_{g_k,\alpha_k}$ and $\delta >0$ such that for any $h\in\mathcal{P}(\mathbb{R}^n)$ it holds that $h\in V$ whenever $\abs{p_{g_j,\alpha_j}(f-h)}<\delta$ for all $j$.
   Take $C_j>0$ such that $\abs{g_j(x)}>C$ for all $x\in K$ and conclude that the open set determined by the conditions $\abs{p_{K,\alpha_j}(f-h)}<C_j\delta$ defines an open neighbourhood of $f$ in $V\cap\mathcal{C}_c^\infty(K)$.
   Since $f$ was arbitrary this shows that $V\cap\mathcal{C}_c^\infty(K)$ is open.

   For the injectivity it must be shown that any $w\in \mathcal{M}(\R^n)$ is entirely determined by it's values on $\mathcal{C}_c^\infty(\mathbb{R}^n)$.
   Pick some $f\in \mathcal{P}(\mathbb{R}^n)$.
   For any $j\geq 0$ it is possible to find some bump function $b_j\in \mathcal{C}_c^\infty(\mathbb{R}^n)$ such that $\abs{\partial^\alpha b_j(x)}<1/j$ for all $x\in\mathbb{R}^n$ whenever $\abs{\alpha} <j$ and $b_j(x) = 1$ for all $x$ with $\norm{x}<j$.
   It follows that for any seminorm $p_{g,\alpha}$ the value $p_{g,\alpha}(b_jf - f)$ tends to zero as $j$ tends to infinity.
   By definition of the topology on $\mathcal{P}(\mathbb{R}^n)$ this means that $f$ is the limit of the sequence of compactly supported functions $b_jf$.
   By the continuity of $w$ it follows that $\lim_{j\to \infty} w(b_jf) = w(f)$ which proves the desired result.

   The fact that $\rho$ is continuous is immediate from the definitions of the respective $\text{weak}^*$-topologies.
   Pick some basis-open $B\subseteq \mathcal{D}(\mathbb{R}^n)$ centred on $w$.
   By definition of the $\text{weak}^*$ topology there exist $f_j\in \mathcal{C}_c^\infty(\R^n)$, $\alpha_j\in \mathbb{Z}^n_{\geq 0}$ and  $\varepsilon_j >0$ for $j=1,\ldots, k$ such that
   $$B = \cap_{j=1}^k \{u\in \mathcal{D}(\R^n): \abs{w(f_j) -u(f_j)} <\varepsilon_j\}. $$
   Now observe that
   $$\rho^{-1}(B) = \cap_{j=1}^k \{u\in \mathcal{M}(\R^n): \abs{w(f_j)-u(f_j)} <\varepsilon_j\}$$
   is a basis open for the $\text{weak}^*$-topology on $\mathcal{M}(\mathbb{R}^n)$.
   This concludes the proof.
\end{proof}

The distributions with finite moments $\mathcal{M}(\mathbb{R}^n)$ come equipped with a right $\D_{\mathbb{R}^n}(\mathbb{R}^n)$-module structure when $\mathbb{R}^n$ is considered as an algebraic variety.
Indeed, note that $\partial_i f\in \mathcal{P}(\R^n)$ and $pf\in \mathcal{P}(\R^n)$ whenever $f\in \mathbb{R}$ and $p\in \R[x]$.
The right $\D_{\mathbb{R}^n}(\mathbb{R}^n)$-module structure on $\mathcal{M}(\mathbb{R}^n)$ is now defined by
$$(w\cdot \partial_i)(f) := -w(\partial_i f); \qquad (w\cdot p)(f) = w(pf)$$
with $f,g$ as above and $w\in \mathcal{M}(\mathbb{R}^n)$.
A similar right $\D_{\mathbb{R}^n}(\mathbb{R}^n)$-module structure applies to $\mathcal{D}(\R^n)$.
\begin{remark}
  Any smooth function $f\in \mathcal{C}^\infty(\mathbb{R}^n)$ induces a distribution, denoted $w = fdx$, which acts on compactly supported functions $g\in \mathcal{C}_c^\infty(\mathbb{R}^n)$ by
  $$w(g) =\int_{\mathbb{R}^n} g(x)f(x) dx.$$
  Observe, using integration by parts, that $w\cdot \partial_i$ corresponds to the distribution $\partial_i f dx$.
\end{remark}

Given a distribution with finite moments $w$ and a multi-index $\alpha\in \mathbb{Z}_{\geq 0}^n$ the $\alpha$th moment is $m_{\alpha}(w):= w(x^\alpha)$.
Observe for later use that
$$ m_{\alpha}(w\cdot \partial_i) = -\alpha_i m_{\alpha -e_i}(w); \qquad m_{\alpha}(w\cdot x_i) = m_{\alpha + e_i}(w)$$
for any $\alpha \in \mathbb{Z}_{\geq 0}^{n}$ where $e_i$ denotes the $i$th standard basis vector.

Finally, observe that the same procedure may be used to define complex distribution $\mathcal{D}(\C^n)$ and complex distributions with finite moments $\mathcal{M}(\C^n)$.
For instance, one should replace $\mathcal{C}_c^\infty(\mathbb{R}^n)$ by the compactly supported smooth functions with complex coefficients on $\C^n$.

%The following notions lead to a space of distributions for which Fourier analysis is applicable.
%The space of rapidly decreasing functions $\mathscr{S}(\mathbb{R}^n)$ consists of all functions $f\in \mathcal{C}^\infty(\mathbb{R}^n)$ such that $x^\alpha\partial^\beta f$ is bounded for all multi-indices $\alpha,\beta \in \mathbb{Z}_{\geq 0}^n$.
%One equips $\mathscr{S}(\mathbb{R}^n)$ with the topology generated by the seminorms
%$$p_{j}(f):= \sup_{x\in \mathbb{R}^n}\{\abs{x^\alpha \partial^\beta f}: \abs{\alpha},\abs{\beta} \leq j\}.$$
%\begin{definition}
%  A temperate distribution is a continuous linear functional on $\mathscr{S}(\mathbb{R}^n)$.
%  The space of temperate distributions $\mathscr{S}'(\mathbb{R}^n)$ is equipped with the corresponding $\text{weak}^*$-topology which is defined by the seminorms
%  $$p_{f}(u) = \abs{u(f)};\qquad f\in \mathscr{S}(\mathbb{R}^n).$$
%\end{definition}

%The following definitions are not in the standard literature of distribution theory but will be convenient for our purposes.
%The space of slowly increasing functions $\mathcal{S}(\mathbb{R}^n)$ consists of all functions $f\in \mathcal{C}^\infty(\mathbb{R}^n)$ such that for any multi-index $\alpha\in \mathbb{Z}_{\geq 0}^n$ there exists some multi-index $\gamma \in \mathbb{Z}_{\geq 0}^n$ and constant $C\in \mathbb{R}_{> 0}$ with $\abs{\partial^\alpha f} \leq c(1 + x^\gamma)$.
%\begin{definition}
%  A moderate distribution is a linear functional on $\mathcal{S}(\mathbb{R}^n)$ which is continuous when restricted to
%\end{definition}

\section{Holonomic functions}\label{sec: HolAlgorithm}
Many real-world functions are analytic but it requires an infinite amount of information to encode an analytic function.
Polynomials only require a finite amount of information and can therefore be encoded in computers.
However, not all real-world functions are polynomials.

It was observed by \cite{zeilberger1990holonomic} that the class of holonomic functions provide good compromise.
Many real-world functions are holonomic and only a finite amount of information is required to encode the corresponding differential equations.
\subsection{Weyl algebra}
Let $K$ denote a field of characteristic zero.
The Weyl algebra $D_n(K)$ is the $K$-algebra found from $K[x_1,\ldots,x_n]$ found by adjoining new variables $\partial_1,\ldots,\partial_n$ subject to the usual commutation relations
$$\partial_i x_j = x_j \partial_i + \delta_{ij};\qquad \partial_i \partial_j = \partial_j \partial_i $$
where $\delta_{ij}$ denotes the Kronecker delta.
One has the corresponding notions of a order filtration, graded objects, characteristic varieties and holonomic $D_n(K)$-modules precisely as in \cref{Ch: ChapterDX}.

An object which gives rise to a $D_n(K)$-module is said to be holonomic over $K$ precisely when the corresponding module is so.
For instance, a holonomic function over $\C$ is a analytic function $f:U\to \C$ on some open $U\subseteq \C^n$ such that $f$ satisfies a system of differential equations $P_{1}(x,\partial)f = 0,\ldots, P_{k}(x,\partial)f=0$ with $D_n(\C)/(P_1,\ldots,P_k)$ holonomic over $D_n(\C)$.
For a parameter-dependent function $f_s(x)$ one can speak of holonomicity over $\C(s_1,\ldots,s_r)$.
Similar notions apply to distributions and real functions.

One can also speak about holonomic sequences.
A left $D_n(K)$-module structure on the multivariate sequence $u:\mathbb{Z}_{\geq 0}^n \to K$ is defined by
$$( \partial_iu)(\alpha) := -\alpha_i u(\alpha-e_i); \qquad (x_iu)(\alpha):= u(\alpha + e_i)$$
where $u(\alpha) = 0$ whenever $\alpha_i<0$ for some $i$.
Observe that, by the remarks at the end of \cref{sec: Distributions}, it follows that for any holonomic distribution with moments $w$ the moment sequence $m_\alpha(w)$ is a holonomic sequence on $\mathbb{Z}_{\geq 0}^n$.
\begin{example}
  Let $f_s(x) = \exp(sx^2)$ on $\C$ then  $\partial f_s - sf_s = 0$.
  It follows that the characteristic variety is the zero section of $T^*\C(s)$.
  Hence, the parameter-dependent function $f_s(x)$ is holonomic over $\C(s)$.

  More generally, for any polynomial $p$ on $\C^n$ with undetermined coefficients $s_1,\ldots,s_r$ it holds that $f_s = \exp(p(x))$ is holonomic over $\C(s_1,\ldots,s_r)$.
\end{example}
\begin{example}\label{ex: Indicator}
  The parameter-dependent distribution $w_{s_1,s_2} = \chi_{[s_1,s_2]}(x)dx$ on $\mathbb{R}$ with $\chi_{[s_1,s_2]}$ the indicator function is holonomic over $\R(s_1,s_2)$.
  Indeed, $w_{s_1,s_2}\cdot \partial = \delta_{s_1} -\delta_{s_2}$ with $\delta$ the Dirac delta.
  Hence, $w_{s_1,s_2}$ is annihilated by $\partial(x-s_2)(x-s_1)$ from which the holonomicity follows.
\end{example}

In the foregoing chapter we had a notion of relative holonomicity over $\C[s]$ with $s = (s_1,\ldots,s_r)$.
One can go from relative holonomicity to holonomicity over $\C(s)$ by use of the following result.
\begin{lemma}
    For any $\D_{\C^n}[s]$-module $\M$ with $\dim\Chrel\M\leq n+r$ it holds that the global sections of $\M\otimes_{\C[s]}\C(s)$ form a holonomic $D_n(\C(s))$-module.
\end{lemma}
  \begin{proof}
    Consider a good filtration on $\M$ and equip $\M\otimes_{\C[s]}\C(s)$ with the induced filtration.
    Since localisation is an exact functor it holds that $\grrel\M\otimes_{\C[s]}\C(s) \cong \gr(\M\otimes_{\C[s]}\C(s)) $.

    Suppose that $\Ch(\M\otimes_{\C[s]}\C(s))$ has dimension strictly greater than $n$ as a variety over $\C(s)$.
    Let $\mathfrak{m}$ be a maximal ideal of the coordinate ring of $\Ch (\M\otimes_{\C[s]}\C(s))$.
    The maximal ideal $\mathfrak{m}$ corresponds to a prime ideal $\mathfrak{p}$ of the coordinate ring of $\Chrel\M$ which does not intersect $\C[s]\setminus\{0\}$.
    Moreover it follows from the assumption that  $\dim\Ch(\M\otimes_{\C[s]}\C(s))> n$ that $\mathfrak{p}$ has height $>n$.

    Now the subvariety $V = Z(\mathfrak{p})$ of $\Chrel\M$ has codimension $>n$ and is not contained in any set of the form $Z(b(s))$ for $b(s)\in \C[s]\setminus\{0\}$.
    Since $\dim\Chrel\M\leq n+s$ it follows that $\dim V <r$.

    Denote $\pi:\C^{n+r}\to\C^r $ for the projection map and observe that $\dim\operatorname{cl}\pi(V)\leq \dim(V) < r$ where $\operatorname{cl}$ denotes the closure in the Zariski topology. %\footnote{\url{https://math.stackexchange.com/q/95794 }}
    This contradicts the assumption that $V\nsubseteq Z(b(s))$ for any $b(s)\in \C[s]\setminus\{0\}$ and we conclude that $\dim \Ch (\M\otimes_{\C[s]}\C(s)) \leq n$.
    \end{proof}
    \subsection{Closure properties}
    The class of holonomic functions is closed under the usual operations.
    These closure properties are typically effective.
    This means that there are algorithms to compute the relations over $D_n(K)$ satisfied by the output of the operation.
    Such algorithms often rely on the theory of Gr\"obner bases and have been implemented in software packages such as SINGULAR or Mathematica.

    \begin{theorem}{\cite[Proposition 3.1]{zeilberger1990holonomic}}
      Let $f,g$ be holonomic functions or distributions over $K$.
      If $f+g$ is defined it is also holonomic over $K$.
    \end{theorem}
    \begin{theorem}{\cite[Proposition 3.2]{zeilberger1990holonomic}}
      Let $f,g$ be holonomic functions over $K$.
      If $f\cdot g$ is defined it is also holonomic over $K$.
      The same result holds if $g$ is a holonomic distribution.
    \end{theorem}
    \begin{theorem}{\cite[Corollary 1]{harris1985reciprocals}}
      Let $f$ be a monovariate holonomic function over $K$.
      If $1/f$ is defined it is holonomic over $K$ if and only if the logarithmic derivative $\partial f/f$ is algebraic over $K$.
    \end{theorem}
    \begin{theorem}{\cite[Theorem 2.7]{stanley1980differentiably}}\label{thm: AlgebraicPrecomp}
      Let $f$ and $a$ be a monovariate holonomic function and a monovariate algebraic functions over $K$ respectively.
      If $f\circ a$ is defined it is also holonomic over $K$.
    \end{theorem}
    \subsection{Asymptotics of holonomic sequences}
    This section is concerned with the asymptotics of monovariate sequences $u:\mathbb{Z}_{\geq 0}\to K$.
    The statement that $u(m) = O(b(m))$ for some positive sequence $b:\mathbb{Z}_{\geq 0}\to \mathbb{R}_{\geq 0}$ means that there exist $C,M\geq 0$ such that for all $m\geq M$ it holds that $\abs{u(m)}\leq Cb(m)$.

    A sequence $u:\mathbb{Z}_{\geq 0}\to K$ is holonomic if and only if there exists some homogeneous linear recursion relation
    $$p_d(m)u(m+d) + p_{d-1}(m)u(m+d-1) + \cdots + p_0(m)u(m) = 0 $$
    with polynomial coefficients $p_k \in K[m]$.
    A holonomic sequence is called $K$-recurrent if the polynomials $p_k$ can be taken to be constants.
    If the $K$-recurrent sequence $u$ has recurrence relation $\sum_{j=0}^d c_{j}u(m+j)= 0$ with $d$ minimal and $c_d = 1$ one calls $\sum_{j=0}^d c_jx^j\in K[x]$ the minimal polynomial of $u$.
    In this case the asymptotics of the recursions are well-understood, see for instance \cite[Chapter 2]{everest2003recurrence}, in particular the structure of the solutions to such recursions is explicitly known.
    \begin{theorem}{\cite[Theorem 1.6]{nobleAsymptotics}}\label{thm: KRecurrent}
      Let $u:\mathbb{Z}_{\geq 0}$ be a $K$-recurrent sequence and denote $\lambda_1,\ldots,\lambda_k$ for the distinct roots of the minimal polynomial of $u$.
      Then there exist polynomials $P_1,\ldots,P_k \in \overline{K}[n]$ such that
      $$u(m) = \sum_{j=1}^k P_j(m)\lambda_j^m$$
      for all $m\in \mathbb{Z}_{\geq 0}$.
    \end{theorem}
    The case of holonomic sequences is more subtle but an asymptotic expansion in the case with $K = \C$ has been proved by \cite{birkhoff1933analytic}.
    It should be noted that there are allegedly gaps in the complicated proof.
    Still, the results have so far matched up with reality.
    Algorithms to compute the expansions have been implemented by \cite{kauers2011mathematica} and \cite{zeilberger1990holonomic}.
    The asymptotic expansion implies the following theorem.
    \begin{theorem}(\cite{birkhoff1933analytic})\label{thm: Birkhoff}
      Suppose that $u:\mathbb{Z}_{\geq 0}\to \C$ is holonomic. Then there exist constants $c,\alpha, \beta, \gamma \in \C$, positive integers $r,k\in \mathbb{Z}_{\geq 1}$ and a polynomial $p\in \C[x]$ such that
      $$u(m) = O\left(cm^{\alpha m} e^{p(m^{1/r})}\beta^m m^\gamma \log(m)^{k-1}\right).$$
    \end{theorem}
    Another approach in the case with $K = \C$ is due to \cite{flajolet1990singularity} and employs the generating function.
    \begin{theorem}{\cite[Theorem 1.5]{stanley1980differentiably}}\label{thm: GeneratingFunction}
      A sequence $u:\mathbb{Z}_{\geq 0}\to K$ is holonomic over $K$ if and only if its generating sequence $f(z) = \sum_{m=0}^{\infty} u(m)z^m$ is so.
    \end{theorem}
    Due to the algorithms associated to \Cref{thm: GeneratingFunction} one gets an implicit description of the generating function $f$ of $u$ by means of a differential equation.
    Suppose that $f$ defines an analytic function near the origin and assume that $f$ has a unique singularity of minimal modulus.
    By renormalization it may be assumed that this singularity occurs at $1$.
    For any $r>1$ and $\theta \in (0,\pi/2)$  consider the closed indented disk
    $$\Delta(r,\theta) := \{z\in \mathbb{C}: \abs{z}\leq r, \quad \abs{\operatorname{Arg}(z-1)}\geq \theta\}.$$
    \begin{theorem}{\cite[Corollary 3]{flajolet1990singularity}}
      With conventions as above, suppose that $f(z)$ is analytic in some domain $\Delta(r,\theta)$ and assume that as $z$ tends to $1$
      $$f(z) = \sum_{j=1}^k c_{j}(1-z)^{\alpha_j} + O(\abs{1-z}^A)$$
      for $c_j,\alpha_j,A\in \C$ with $\operatorname{Re}(\alpha_0)\leq \cdots \leq \operatorname{Re}(\alpha_k)<A$.
      Then, as $m$ tends to infinity
      $$u(m) = \sum_{j=1}c_j m^{-\alpha_j' - 1} + O(m^{-A -1}).$$
    \end{theorem}
    Other methods based on generating functions are also known, for instance due to \cite{hayman1956generalisation}.
    Algorithms to determine the asymptotics of $u$ based on the generating function have been implemented by \cite{salvy1991examples}.
\section{Holonomic random variables}
It has been observed by \cite{bitoun2019feynman} that the holonomic toolset can be used to reduce the computations stemming from Feynman integrals in certain problems of quantum field theory.
Holonomicity has also been used to investigate the statistical problem of estimating the parameters of a distribution based on a finite number of moments.
This problem was investigated by \cite{batenkov2009moment} for holonomic distributions on a compact interval $[a,b]\subseteq \mathbb{R}$ and by \cite{brehard2019moment} for holonomic distributions on a compact semi-algebraic subset $G\subseteq \mathbb{R}^n$.

The goal of this section is to investigate possible applications of the holonomic toolset to the study of the tails of random variables.
\subsection{Tail condition for random variables}\label{sec: SubGSubE}
\begin{theorem}{\cite[Proof of Proposition 2.5.2]{vershynin2018high}}\label{thm: EquivalentLp}
  Let $X$ be a real random variable and let $p\in \mathbb{R}_{>0}$. The following are equivalent
  \begin{enumerate}[label = (\roman*)]
    \item There exists some $R_1\in \mathbb{R}_{>0}$ such that $\mathbb{P}(\abs{X}\geq t) \leq 2\exp(-t^p/R_1^p)$
    \item There exists some $R_2\in \mathbb{R}_{>0}$ such that $\mathbb{E}\exp(\abs{X}^p/R_2^p) \leq 2.$
    \item There exists some constant $R_3\in \mathbb{R}_{>0}$ such that $\left(\mathbb{E} \abs{X}^k \right)^{1/k} \leq R_3k^{1/p}$
    for all $k\in \mathbb{Z}_{\geq 0}$.
  \end{enumerate}
  Moreover, the constants $R_i$ which occur in these equivalent conditions can differ by at most an absolute constant factor.
\end{theorem}
\begin{definition}
  Let $p\in \mathbb{R}_{\geq 1}$.
  A random variable $X$ belongs to $L_{\psi_p}$ if it satisfies any of the equivalent conditions of \Cref{thm: EquivalentLp}.
  In this case the minimal constant $R_2$ which satisfies part (ii) of \Cref{thm: EquivalentLp} is denoted $\norm{X}_{\psi_p}$.
\end{definition}
\begin{remark}
  One can show that for any $p\geq 1$ it holds that $(L_{\psi_p},\norm{\cdot}_p)$ is a normed vector space.
  The variables in $L_{\psi_1}$ are called sub-exponential and the variables in $L_{\psi_2}$ are called sub-gaussian.
\end{remark}
Consider the following theorem as an example of the usefulness of these assumptions.
\begin{theorem}{\cite[Theorem 4.4.5]{vershynin2018high}}
  There exists some constant $C>0$ such that for any  $m\times n$ random matrix $A$ with independent sub-gaussian entries $A_{ij}$ satisfying $\mathbb{E}A_{ij} = 0$, $\mathbb{E}A_{ij}^2 = 1$ and $\norm{A_{ij}}_{\psi_2}\leq R$
  it hold that
  $$\mathbb{P}\left(\norm{A} \leq  CR(\sqrt{m} + \sqrt{n} + t)\right) \geq 1-2\exp(-t^2)$$
  for all $t>0$.
  Here $\norm{A}$ denotes the operator norm of $A$.
\end{theorem}
\subsection{Recursion in moments}
The following result is originally due to \cite[Theorem 1]{loeser1991caracterisation} and a simplified proof may be found in \cite[Appendix B]{bitoun2019feynman}.
The result was used by \cite{bitoun2019feynman} to reduce a computational problem associated to Feynman integrals in quantum field theory.
We apply the same ideas but replace Feynman integrals by the moments of a distribution.
\begin{theorem}\label{thm: HolonomicFiniteDimTorus}
  Let $M$ be a holonomic left $D_n(K)$-module.
  Then $K(x\partial)\otimes_{K[x\partial]}M$ is a finite dimensional $K(x\partial)$-vector space.
  Moreover, it holds that
  $$\dim_{K(x\partial)} K(x\partial)\otimes_{K[x\partial]}M  = \chi\left((K^*)^n , \iota^* \M\right)$$
  where $\M$ denotes the $\D_{K^n}$-module associated to $M$ and $\iota:(K^*)^n \to K^n$ is the inclusion of the algebraic torus.
\end{theorem}
In the following result $K=\R$ or $K = \C$.
\begin{corollary}\label{cor: Recursion}
  Let $w_s$ be a parameter-dependent distribution with finite moments and suppose that $w_s$ is holonomic over $K(s)$.
  Then there exists $p_0\in K[s,\alpha]^\times$ and $\beta_1,\ldots,\beta_k\in \mathbb{Z}_{\geq 0}^n$ such that for any $\alpha \in \mathbb{Z}_{\geq 0}^n$ there exists $p_1,\ldots,p_k \in K[s,\alpha]$ with
  $$p_0(s,\alpha) m_{\alpha}(w_s) = p_1(s,\alpha) m_{\beta_1}(w_s)+ \cdots + p_k(s,\alpha) m_{\beta_k}(w_s).$$
\end{corollary}
\begin{proof}
  Denote $F = K(s).$
  As observed in \cref{sec: HolAlgorithm} it follows from the holonomicity of $w_s$ that the $D_n(F)$-module $M$ associated to $m_{\alpha}(w_s)$ is holonomic.
  Write $m$ for the generator of the $M$.
  By \Cref{thm: HolonomicFiniteDimTorus} it holds that $d:=\dim_{F(x\partial)}F(x\partial)\otimes_{F[x\partial]}M$ is finite.
  Take $\{\beta_1,\ldots,\beta_k\} = \{0,1,\ldots,d\}^n$.

  We proceed by induction on $\abs{\alpha}$, the base case is trivially satisfied.
  Now assume that $\abs{\alpha}>0$.
  It may further be assumed that there is some $\alpha_i>d$.
  Let $\ell \in \{0,\ldots,d\}$ be minimal with $m, x_im,\ldots,x_i^{\ell}m$ linearly dependent in $F(x\partial)\otimes_{K[x\partial]}M$.
  Then we can find polynomials $p_0,\ldots, p_\ell\in K$ with $\sum_{j=0}^\ell p_j x^{\ell - j} m =0$ in $F(x\partial)\otimes_{K[x\partial]}M$.
  By the minimality of $\ell$ it will further be the case that $p_0\neq 0$.

  Hence, there exists some $N\geq 0$ such that $\prod_{i=1}^n (x_i\partial_i)^N(\sum_{j=0}^\ell p_j x^{\ell - j} m) = 0$ in $M$.
  Replacing every $p_j$ by $\prod_{i=1}^n(x_i\partial_i)^Np_j$ conserves the fact that $p_0\neq 0$ so it may be assumed that $\sum_{j=0}^\ell p_j x^{\ell - j} m_{\alpha} =0$.
  Now use the definition of the $D_n(F)$-module structure on sequences to deduce that
  $$ p_0(\gamma + ke_i)  m_{\gamma +ke_i}(w_s) + \cdots + p_1(\gamma +e_i)m_{\gamma + e_i}(w_s) + p_0(\gamma)m_{\gamma}(w_s) =0$$
  for any $\gamma \in \mathbb{Z}_{\geq 0}^n$.

  Since $\alpha_i >k$ this may be applied with $\gamma = \alpha - ke_i$ and the induction hypothesis yields the desired result.
\end{proof}
\begin{example}\label{ex: GaussianMono}
  Let $w_{\mu,\sigma}$ be the Gaussian distribution on $\mathbb{R}$ with mean $\mu$ and standard deviation $\sigma>0$
  $$w_{\mu,\sigma} =\frac{1}{\sigma \sqrt{2\pi}} \exp\left(-\frac{(x-\mu)^2}{2\sigma^2}\right)dx.$$
  Then $w_{\mu,\sigma}$ is annihilated by $\partial +(x-\mu)/\sigma^2$. Hence, the moments satisfy a recursion relation
  $$m_{\alpha + 1}(w_{\mu,\sigma}) = \mu m_{\alpha}(w_{\mu,\sigma}) + \alpha \sigma^2 m_{\alpha-1}(w_{\mu,\sigma})$$
  with initial conditions $m_{0}(w_{\mu,\sigma}) =1$ and $ m_1(w_{\mu,\sigma}) = \mu.$
  By iterating the recursion relation one can indeed relate any $m_\alpha(w_{\mu,\sigma})$ to the first two moments as predicted by \Cref{cor: Recursion}.
  In particular, when $\mu = 0$ it follows that $m_{2j + 1}= 0$ and $m_{2j} = (2j - 1)(2j - 3)\cdots 1 \sigma^{2j}$ for any $j\in \mathbb{Z}_{\geq 0}$.
\end{example}
\begin{example}
  Let $w_{s_1,s_2} = \chi_{[s_1,s_2]}(x)dx$ as distribution on $\mathbb{R}$.
  As in \Cref{ex: Indicator} it holds that $w_{s_1,s_2}\cdot (x-s_1)(x-s_2) \partial =0$ so
  $$\alpha m_{\alpha +1}(w_{s_1,s_2}) = \alpha (s_1 + s_2)m_{\alpha}(w_{s_1,s_2}) - \alpha s_1 s_2 m_{\alpha}(w_{s_1,s_2})$$
  for any $\alpha \geq 0$.
  This demonstrates the relevance of the factor $p_0$ in \Cref{cor: Recursion}, dividing both sides by $\alpha$ could yield a wrong result when $\alpha = 0$.
\end{example}
\begin{remark}
  Let $p\in \R[x_1,\ldots,x_n]$ be a polynomial and consider a compact semi-algebraic set $G\subseteq \mathbb{R}^n$ with boundary contained in the zero set of a polynomial $g\in \mathbb{R}[x_1,\ldots,x_n]$.

  Recursions in the moments of the distribution $w = \exp(p(x)) \chi_G(x) dx$ have been considered by \cite{brehard2019moment}.
  It is shown in loc. cit. that the coefficients of the polynomial $p$ and a polynomial $g$ as above can be recovered given finitely many moments.

  It would be interesting to study similar reconstruction problems in our non-compact case.
\end{remark}
\subsection{Tail conditions and holonomicity}
A random variable $X$ with values in $\C^n$ or $\R^n$ will be called holonomic if its probability measure can be viewed as a holonomic distribution.

Recall that, by \Cref{thm: EquivalentLp}, establishing a tail condition on a real random variable $X$ may be reduced to the study of the absolute moments $\mathbb{E}\abs{X}^k$.
Note that whenever $X$ is holonomic with finite moments the same will hold for $\abs{X}$.
The following examples demonstrate how the asymptotic results in \cref{sec: HolAlgorithm} may be used to establish tail conditions.
The computations were done in the Mathematica package by \cite{kauers2011mathematica} which provides an implementation for \Cref{thm: Birkhoff} due to \cite{birkhoff1933analytic}.

\begin{example}
  Let $X=\abs{Y}$ with $Y$ coming from a Gaussian distribution with mean $0$ and standard deviation $\sigma$.
  The distribution of $X$ is equal to
  $$w = \frac{2}{\sqrt{2\pi}}\exp\left(-\frac{x^2}{2\sigma^2}\right)\chi_{[0,\infty)}(x)dx.$$
  This distribution is annihilated by the differential operator $x(\partial + x/\sigma^2).$
  The corresponding recursion for the moments $\mathbb{E}X^k = m_k(w)$ is then
  $$\mathbb{E}X^{k+1}= \sigma^2 k\mathbb{E}^{k-1} $$
  for all $k\geq 1$.
  Using this recursion in Mathematica yields that
  $$\mathbb{E}\abs{Y}^{k} = O\left(e^{-k/2}k^{k/2}\sigma^k\right) $$
  which establishes that a centerd Gaussian random variable is sub-gaussian as was to be expected.
\end{example}
\begin{example}
  Let $X$ come from an exponential distribution with shape parameter $\lambda>0$.
  This means that $X$ has distribution
  $$w =  \lambda e^{-\lambda x} \chi_{[0,\infty)} dx.$$
  The distribution is annihilated by $x(\partial - \lambda )$ so that
  $$\mathbb{E}X^{k+1}= \frac{1}{\lambda}k\mathbb{E}X^k$$
  for all $k\geq 1$.
  This implies that $\mathbb{E}X^k = \lambda^{-k}k!$ so Stirling's formula finishes the asymptotic estimate.
  Alternatively Mathematica yields that
  $$\mathbb{E}X^k = O(\lambda^{-k}e^{-k}k^{\frac{1}{2} + k}) $$
  which agrees with the result from Stirling's formula.
  This establishes that the exponential distribution with shape parameter 1 is sub-exponential as was to be expected.
\end{example}
Mathematica can only deal with symbolic constants to a limited degree, the calculations break breaks down for a normal distribution with both undetermined mean and variance.
Let us now consider a less trivial example which also demonstrates how singularities may be disregarded.
\begin{example}
  Let $Y$ come from a Weibull distribution with scale parameter $\lambda>0$ and shape $k = 1/2$.
  This means that the density of $Y$ takes the form
  $$f_{Y,\lambda}(x) = (2 \lambda \sqrt{x/\lambda})^{-1}\exp(-\sqrt{x/\lambda})\chi_{[0,\infty)}(x).$$
  This density does not determine a distribution due to the pole at zero.
  Instead consider the distribution
  $$w_\lambda = f_{Y,\lambda}(x)\chi_{[1\infty)}(x)dx $$
  and note that $\mathbb{E}Y^k \leq m_k(w) + 1$ for all $k\geq 0$.
  The distribution $w$ is annihilated by $(x-1)^2(4 \lambda x \partial^2 + 6\lambda\partial - 1)$
  whence it follows that the moments satisfy the recursion
  $$m_{k+1}(w_\lambda) = -(4\lambda k^2 - 2\lambda k + 2)m_{k+1}(w_\lambda) + (8\lambda k^2 + 4\lambda k + 1)m_k - (4\lambda k^2 + 2\lambda k)m_{k-1}(w_\lambda).$$
  Mathematica now produces the following asymptotic estimate
  $$\mathbb{E}Y^k = O\left(4^k\lambda^k e^{-2k}k^{2k -5/2}\right).$$
\end{example}
