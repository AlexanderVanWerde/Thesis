\chapter{Moments of Holonomic Distributions}
In this chapter we discuss how holonomicity may be used to determine recursion relations for the moments of a probability distribution.
Similar observations have been made in \cite{brehard2019moment} whose main focus is on distributions of the form $\mu = \exp(p(x))1_{G}(x)dx$ with $p(x)$ a multivariate polynomial and $G$ a semi-algebraic set.
There it is discussed when it is possible to recover the coefficients of $p(x)$ and some polynomial $g(x)$ vanishing on the boundary of $G$ given a finite number of moments $m_n = \int x^n d\mu$.

We will not address this reconstruction problem and instead focus on the asymptotic rate of growth of the moments.
This problem is of interest in non-asymptotic probability theory.
For non-asymptotic concentration results it is often necessary to assume that the tails of the probability distribution decrease sufficiently rapidly.
This gives rise to the notions of subgaussian and subexponential random variables.
The goal of this chapter is to provide a easy way to verify whether a given holonomic distribution satisfies these conditions.

In \cref{sec: Motivation} we provide the probabilistic motivation for the problem.




\section{Probabilistic Motivation}\label{sec: Motivation}

\section{Holonomic Distributions}

\section{Recursion on the Moments}
Let $\mu$ be a fixed holonomic distribution on $\mathbb{R}^n$ and assume that the moments
$$m_\alpha = \int_{\mathbb{R}^n} x^\alpha d\mu$$
exist for all $\alpha \in \mathbb{Z}_{\geq 0}^n$.

From here on out we will assume that $n = 1$ which is the main case of interest in the probabilistic context.
\begin{proposition}
  Suppose that $\mu$ is annihilated by $P \in \D_{\C}^n$. If $P = \sum_{ij} c_{ij}x^i \partial^j$ then it holds that
  $$ \sum_{ij} c_{ij} (n+ i)(n+ i -1)\cdots (n+i-j + 1) m_{n + i - j} = 0$$
  for any $n \geq 0$.
\end{proposition}
