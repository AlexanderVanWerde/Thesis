\chapter{Categorical Preliminaries}\label{ch: ChapterCategory}
This chapter contains some categorical preliminaries on the topic of derived category theory and spectral sequences.

Derived category theory allows to measure the lack of exactness in a functor $F:\AA \to \mathcal{B}$ by encoding error-terms in derived functors $R^iF:\AA \to \mathcal{B}$.
For instance the non-exactness of the tensor product may be measured by $\Tor$-functors.

Spectral sequences were historically developed by Leray to compute the cohomology of the pushforeward of a sheaf.
There is some overlap between derived category theory and spectral sequences.
In particular the Grothendieck spectral sequence allows one to compute the derived functor of some composition $F\circ G$ based on the derived functors of $F$ and $G$ individually.
This theorem is a essential technical ingredient in the proofs of \cref{ch: ChapterRelHol}.

The discussion of derived category theory in this chapter summarises the relevant parts of \cite[Chapters 1, 2 and 5]{dimca2004sheaves}.
The section on spectral sequences is based on \cite{weibel1995introduction}.
\section{Spectral Sequences}
Fix a abelian category $\mathcal{A}$.
Denote $C(\AA)$ for the category with complexes of objects in $\AA$
$$X^\bullet :\cdots \xrightarrow{}  X^{-1}\xrightarrow{d^{-1}} X^0 \xrightarrow{d^{0}} X^1\xrightarrow{d^1} \cdots.$$
A double complex $E^{\bullet \bullet}$ gives rise to a total complex with terms  $\operatorname{Tot}(E)^n = \oplus_{i+j=n} E^{ij}$.
The motivating question behind spectral sequences is how the cohomology of the total complex may be computed.

We will not go into detail on this motivating question but the idea is that one can first compute horizontal cohomology to get data $E^{**}_1$.
By the commutativity of the double complex there are vertical differentials on $E^{**}_1$ and one can compute the vertical cohomology to get $E^{**}_2$.
Diagram chasing allows to construct higher-order differentials leading to the following notion.
\begin{definition}
  A cohomology spectral sequence starting at the $a$-th sheet consists of families of objects $ \{E^{p,q}_r\}_{p,q\in \mathbb{Z}}$ for $r\geq a$ and maps
  $d_{pq}^r : E^{pq}_r \to E^{p+r,q-r+1}_r $
  such that
  \begin{enumerate}
    \item[(i)]  The maps $d^r_{pq}$ are differentials in the sense that
    $d^r \circ d^r = 0.$
    \item[(ii)] The $(r+1)$-st sheet is the cohomology of the $r$-th sheet
    $E_{pq}^{r+1} \cong \ker(d_{pq}^r)/ \img(d_{p-r,q+r-1}^r).$
  \end{enumerate}
\end{definition}
\begin{definition}
  A cohomology spectral sequence is said to be bounded if for each $n$ there are finitely many non-zero terms $E^{pq}_a$ with $p+q = n$.
\end{definition}
In a bounded complex there is, for each choice of $p,q$, a value $r_0$ such that $E_{r}^{pq}= E_{r+1}^{pq}$ for all $r \geq r_0$. This stable value is denoted $E^{pq}_\infty$.
\begin{definition}
  A bounded spectral sequence is said to converge to a family of objects $H^*$ if any $H^n$ admits a finite filtration
  $$0 = F^s H^n \subseteq \cdots F^p H^n \subseteq F^{p+1}H^n \subseteq \cdots \subseteq F^t H^n = H^n $$
  such that $E^{pq}_\infty \cong F^p H^{p+q} / F^{p-1} H^{p+q}$.
\end{definition}
Observe that $H^*$ is not necessarily uniquely identified by a convergent spectral sequence.
The total complex in the motivating problem comes equipped with a filtration $F^{m} \operatorname{Tot}(E)^n = \oplus_{p+q = n, p <m} E^{pq}.$
\begin{definition}
  A filtration of a complex $C_\bullet$ is a family of subcomplexes $\{F^m C^\bullet\}_{m\in \Z}$.
  The filtration is said to be exhaustive if $C^\bullet = \cup_m F^mC^\bullet$.
\end{definition}
\begin{proposition}
 A exhaustive filtration of a complex $C^\bullet$ determines a spectral sequence starting with $E^{pq}_0 = F^p C^{p+q}/F^{p-1}C^{p+q}$ and $E^{pq}_1= H^{p+q} E^{p\bullet}_0$.
\end{proposition}
\begin{proof}
  The construction for this spectral sequence may be found in \cite[Chapter 5]{weibel1995introduction}
\end{proof}
\begin{definition}
  A filtration on a complex $C^\bullet$ is said to be bounded
 if, for each $n$, there are integers $s<t$ such that $F^s C^n = 0$ and $F^t C^n = C^n$.
\end{definition}
The following proposition may be used in the motivating problem to recover some information on the cohomology of the total complex.
 \begin{proposition}\label{prop: FiltrationSpectral}
   Let $C^\bullet$ be a complex with a exhaustive bounded filtration.
   Then, the associated spectral sequence is bounded and converges to $H^*(C^\bullet)$.
 \end{proposition}
 \begin{proof}
   This result may be found in \cite[Chapter 5]{weibel1995introduction}.
 \end{proof}
\section{Derived Categories}
The category $C(\AA)$ contains full subcategories $C^*(\AA)$ with $*\in \{ +, -, b\}$ denoting that the complexes in $\AA$ are bounded below, above or bounded on both sides respectively.
For example $C^+(\AA)$ may contain complexes of the form $\cdots \to 0\to \cdots X^{-1} \to X^0 \to \cdots$.
For a complex $X^\bullet$ and $k\in \Z$ one has a shifted complex $X^\bullet[k]$ with $(X^\bullet[k])^s = X^{k+s}$.
Further, denote $\operatorname{Hom}^k(X^\bullet, Y^\bullet) :=  \op{Hom}(X^\bullet, Y^\bullet[k])$ which are the chain maps that change the grading by $k$.
\begin{definition}
 Two complex morphisms $u,v:X^\bullet\to Y^\bullet$ are called homotopic if there exists $h\in \operatorname{Hom}^{-1}(X^\bullet, Y^\bullet)$ such that $u-v = d_Y h + hd_X$.
\end{definition}
\begin{definition}
 A morphism $u:X^\bullet\to Y^\bullet$  of complexes in $C^*(\AA)$ is called a quasi-isomorphism if the induced morphism in cohomology $H^k(u):H^k(X^\bullet) \to H^k(Y^\bullet)$ is a isomorphism for all $k$. This may be denoted $u\sim v$.
\end{definition}
The idea behind the following definition is to retain the same objects as $C^*(\AA)$ but turn quasi-isomorphisms into isomorphisms.
The technicalities may be found in \cite[Chapter 8]{deligne1977sga}.
\begin{definition}
  The derived category $D^*(\AA)$ is defined as the category obtained from $C^*(\AA)$ by localising with respect to the multiplicative system formed by the quasi-isomorphisms.
\end{definition}
This definition can be made more concrete provided the category has enough injectives.
\begin{definition}
 A abelian category $\AA$ has enough injectives if for any object $X$ in $\AA$ there is an exact sequence $0\to X \to I$ in $\AA$ with $I$ injective.
\end{definition}
\begin{definition}
 Let $\AA$ be a abelian category.
 The homotopical category of complexes $K^*(\AA)$ of $\AA$ has the same objects as $C^*(\AA)$ and as morphisms
 $$\operatorname{Hom}_{K^*(\AA)}(X^\bullet,Y^\bullet):= \operatorname{Hom}_{C^*(\AA)}(X^\bullet,Y^\bullet)/\sim.$$
\end{definition}
Observe that two homotopic maps induce the same morphism in cohomology.
It follows that there is a well-defined functor $p_\AA^*:K^*(\AA)\to D^*(\AA)$.
\begin{proposition}\label{prop: DerivedCategoryInjectives}
 Let $\AA$ be a abelian category with enough injectives and denote $I(\AA)$ for the full subcategory from the injective objects.
 Then the natural functor
 $$p_\AA^*: K^+(I(\AA))\to D^+(\AA) $$
 is a equivalence of categories.
\end{proposition}
\begin{proof}
  This result may be found in \cite[Chapter 1]{dimca2004sheaves}.
\end{proof}
By passing to the opposite categories one gets a similar theorem in categories with enough projectives for $D^-(\AA)$.
\section{Triangulated Categories}
The categories $K^*(\AA)$ and $D^*(\AA)$ remain additive but may fail to be exact.
In particular, the notion of short exact sequences no longer makes sense.
Instead, $K^*(\AA)$ and $D^*(\AA)$ may be viewed as triangulated categories which is to say that they come equipped with a notion of exact triangles.
\begin{definition}
 Let $u:X^\bullet \to Y^\bullet$ be a morphism of complexes in $C^*(\AA)$.
 The mapping cone of $u$ is the complex in $C^*(\AA)$ given by
 $$C_u^\bullet := Y^\bullet \oplus (X^\bullet[1]) $$
 with $d_u(y,x)= (dy + u(x) , -dx)$.
\end{definition}
The concept of a mapping cone originated in a construction from algebraic topology which explains the name.
%From the perspective of $X$ the $j$-th cohomology of the mapping cone measures the $x\in X^j$ which have a primitive in $Y^{j-1}$ but not in $X^{j-1}$.
Observe that the mapping cone gives rise to a triangle
$$T_u:X^\bullet \xrightarrow{u} Y^\bullet \to C_u^\bullet \to X^\bullet[1]$$
which may be denoted more intuitively as
$$
 \begin{tikzcd}
   X^\bullet \arrow{rr}{u}& & Y^\bullet \arrow{dl}{q}\\
   & C_u^\bullet\arrow{lu}{+1}
 \end{tikzcd}
$$
The triangles $T_u$ may be used to encode short exact sequences.
\begin{proposition}\label{prop: SESYieldsTriangle}
Given a short exact sequence in $C^*(\AA)$
$$0 \to X^\bullet \xrightarrow{u} Y^\bullet \xrightarrow{v} Z^\bullet \to 0 $$
there exists a quasi-isomorphism $m:C_u^\bullet \to Z^\bullet$ with $m\circ q = v$.
\end{proposition}
\begin{proof}
  This result may be found in \cite[Chapter 1]{dimca2004sheaves}.
\end{proof}
This shows that a short exact sequence induces a triangle isomorphic to a standard triangle $T_u$ in $D^*(\AA)$.
Further evidence that the triangles $T_u$ behave like short exact sequences is given by the following result.
\begin{proposition}
 Let $u:X^\bullet \to Y^\bullet$ be a morphism in $C^*(\AA)$.
 \begin{enumerate}
   \item[(i)] The composition of any two consecutive maps in $T_u$ is homotopic to $0$.
   \item[(ii)] The triangle $T_u$ induces a long exact sequence in cohomology
   $$\cdots \to H^k(X^\bullet) \xrightarrow{u} H^k(Y^\bullet) \to H^k(C_u^\bullet) \xrightarrow{\delta} H^{k+1}(X^\bullet) \to \cdots$$
   where the connecting morphism $\delta$ comes from the map $C_u^\bullet \to X^\bullet[1]$.
 \end{enumerate}
\end{proposition}
\begin{proof}
  This result may be found in \cite[Chapter 1]{dimca2004sheaves}.
\end{proof}
Further investigation of the properties of $T_u$ gives rise to the concept of a triangulated category.
These definitions and properties are pleasant in their own right so we go into some detail.

The distinguish triangles $\mathcal{T}$ in $K^*(\AA)$ or $D^*(\AA)$ are the family of triangles which are isomorphic to a triangle of the form $T_u$.
Observe that these categories have a shift functor $T$ given by $TX^\bullet = X^\bullet[1]$.
\begin{definition}
 An additive category $\mathcal{D}$ equipped with a self-equivalence $T$ and family of distinguished triangles $\mathcal{T}$ is called a triangulated category if the following axioms are satisfied.
 \begin{enumerate}
   \item[(Tr1)] Any triangle isomorphic to a distinguish triangle is distinguished. For any object $X$ the triangle $X\to X \to 0 \to TX$ is distinguish where the first morphism is the identity.
   Any morphism $u:X\to Y$ is part of some distinguished triangle $X\xrightarrow{u} Y \to Z \to TX$.
   \item[(Tr2)] A triangle $X\xrightarrow{u} Y \xrightarrow{v} Z \xrightarrow{w} TX$ is distinguished if and only if the triangle $Y \xrightarrow{v} Z \xrightarrow{w} TX \xrightarrow{-Tu} TY$ is distinguished.
   \item[(Tr3)] A commutative diagram of the following from whose rows are distinguished triangles gives rise to a morphism of triangles
   $$
     \begin{tikzcd}
       X\arrow{d}\rar& Y\arrow{d}\rar & Z\rar & TX\\
       A\rar& B\rar & C\rar & TA
     \end{tikzcd}
    $$
   \item[(Tr4)] For any triple of distinguished triangles
   $$
     \begin{tikzcd}[row sep=small]
       X \arrow{r}{u} & Y \arrow{r}{x}& A \arrow{r}{} & TX \\
       Y \arrow{r}{v} & Z \arrow{r}{}& B \arrow{r}{y} & TY\\
       X \arrow{r}{vu} & Z\arrow{r}{} & C\arrow{r}{} & TX
     \end{tikzcd}
   $$
   there is a distinguished triangle
   $$\begin{tikzcd}
     A \arrow{r}{a} & C \arrow{r}{b}& B \arrow{r}{(Tx)y} & TA \\
   \end{tikzcd} $$
   such that $(id_X, v,a)$ and $(u,id_Z,b)$ are morphisms of triangles.
   \end{enumerate}
\end{definition}
\begin{proposition}
 Let $\mathcal{A}$ be a abelian category. Then $K^*(\AA)$ and $D^*(\AA)$ are triangulated categories.
\end{proposition}
\begin{proof}
  This result may be found in \cite[Chapter 1]{dimca2004sheaves}.
\end{proof}
A triangle $X\to Y \to Z \to TX$ will also be denoted $X\to Y \to Z \xrightarrow{+1} X$ and $T^m X$ may be denoted with $X[m]$.
Now the data of the final axiom can be organised as follows.
Correspondingly, $(Tr4)$ is also referred to as the octahedral axiom.
\begin{equation*}
\xymatrix@=1.6em{
& C \ar@{->}[dr] \ar[ddl] & \\
A \ar@{->}[ur] \ar[d]& &
   B\ar[ll]|(0.25)\hole|(0.75)\hole
      \ar[ddl] \\
X \ar[rr]^{v \circ u} \ar[dr]_u & &
   Z \ar[u] \ar[uul]\\
& Y \ar[ur]_v \ar[uul]&
}
\end{equation*}
\begin{definition}
 Let $\mathcal{D}$ be a triangulated category. A subcategory $\mathcal{C}$ of $\mathcal{D}$ is said to be stable under extensions if any distinguished triangle in $\mathcal{D}$ with two vertices in $\mathcal{C}$ also has it's third vertex in $\mathcal{D}$.
\end{definition}
\begin{definition}
 Let $\mathcal{C}$ be a full additive subcategory of a triangulated category $\mathcal{D}$. One calls $\mathcal{C}$ is a triangulated subcategory if $\mathcal{C}$ is stable under extensions and $T\mathcal{C}\subseteq \mathcal{C}$.
\end{definition}

\begin{definition}
 Let $\mathcal{D}$ be a triangulated category and $\AA$ a abelian category.
 An additive functor $F:\mathcal{D} \to \AA$ is a cohomological functor if for any distinguished triangle in $\mathcal{D}$
 $$X \to Y \to Z\xrightarrow{+1} X $$
 the induced sequence $F(X) \to F(Y) \to F(Z) $
 is a exact in $\AA$. If $F$ is a cohomological functor one sets $F^i = F\circ T^i$.

 The family of functors $F^i$ is conservative if for any distinguished triangle
 $$X \to Y \to Z \xrightarrow{+1} X$$
 the induced long sequence
 $$\cdots \to F^i(X) \to F^i(Y) \to F^i(Z) \to F^{i+1}(X) \to \cdots $$
 is exact.
\end{definition}
The key example for the above definition is given by the cohomological functor $H^0: K^*(\AA) \to \AA$ and the conservative system of functors $H^k$.
\begin{definition}
 Let $\mathcal{D}, \mathcal{D}'$ be triangulated categories.
 A functor $F:\mathcal{D} \to \mathcal{D}'$ is called a functor of triangulated categories if it is compatible with the shift functor and transforms distinguished triangles in $\mathcal{D}$ into distinguished triangles of $\mathcal{D}'$.
\end{definition}
\section{Derived Functors}
Given abelian categories $\AA, \mathcal{B}$ and a functor of triangulated categories $F:K^*(\AA)\to K(\mathcal{B})$ one may wonder if there is a natural lift to the derived categories.
\begin{definition}
 Let $F$ be as above. The right derived functor of $F$ is a initial couple $(R^*F,\xi_F)$ consisting of a functor of triangulated categories $R^*F:D^*(\AA) \to D^*(\mathcal{B})$ and a natural transformation $\xi_F:p_B\circ F \to R^*F \circ p_\AA^*$.
 By initial it is mean that for any other such couple $(G,\zeta)$ there is a unique natural transformation $\eta: R^*F\to G$ such that $\zeta = (\eta \circ p_\AA^*)\circ \xi_{F}$.
\end{definition}
The dual notion is a left derived functor. This is a final couple $(L^*F,\xi_F)$ consisting of a functor of triangulated categories $F^*F:D^*(\AA) \to D^*(\mathcal{B})$ and a natural transformation $\xi_F:  L^*F \circ p_\AA^* \to p_B\circ F$.
It is clear that, if a derived functor exists, it is unique up to unique isomorphism.

%One can show that the derived functor of $F:K(\AA)\to K(\mathcal{B})$ can be defined at a object $X\in K^+(\AA)$ if and only if the restriction $F^+:K^+(\AA) \to K(\mathcal{B})$ is defined at $X$ and the values are canonically isomorphic.
%This means that there are no problems
There are general theorems on the existence of derived functors which may be found in \cite[Chapter 1]{dimca2004sheaves}.
The following will be sufficient for our applications.
\begin{theorem}\label{thm: InjectivesAllowDerivedFunctor}
 Consider a functor of triangulated categories $F:K^+(\AA)\to K(\mathcal{B})$ .
 If $\AA$ has enough injectives and $F$ is additive then the right derived functor $R^+F$ exists.
\end{theorem}
By dualising, a similar theorem applies to $F:K^-(\AA)\to K(\mathcal{B})$ for the existence of $L^- F$ in categories with enough projectives.

The main use of derived functors is to fix a lack of exactness in $F$.
Recall from \cref{prop: SESYieldsTriangle} that a short-exact sequence in $C^+(\AA)$ induces a distinguished triangle in $D^+(\AA)$.
Applying $R^+F$ to the distinguished triangle returns a distinguished triangle by $R^+F$ being a functor of triangulated categories.
Further, there is a associated long exact sequence.
The higher-order terms measure measures to what degree the original functor failed to be exact.
\begin{definition}
 Let $F:K^*(\AA) \to K(B)$ be a functor of triangulated categories such that $R^*F$ exists.
 For any $n\in \Z$ one defines $R^nF:\AA \to \mathcal{B}$ to be the composition
 $$\AA\xrightarrow{\iota} D^*(\AA) \xrightarrow{R^*F} D(\mathcal{B}) \xrightarrow{H^n} \mathcal{B} $$
 where $\iota$ sends a object to the chain complex with a single non-trivial term.
 Similarly, one defines $\mathbb{R}^nF:D^*(\AA)\to \mathcal{B}$ as the composition
 $$D^*(\AA) \xrightarrow{R^*F} D(\mathcal{B}) \xrightarrow{H^n} \mathcal{B}. $$
\end{definition}
\begin{proposition}
   Let $F:\AA\to \mathcal{B}$ be a functor of triangulated categories.
   Suppose that the derived functor $R^*F$ of the induced functor $F:K^*(\AA)\to K(\mathcal{B})$ exists.
   Then, for any short exact sequence in $\AA$
   $$0\to X \to Y \to Z \to 0 $$
   there is a long exact sequence in $\mathcal{B}$
   $$ \cdots \to R^{i}F(X)\to R^{i}F(Y) \to R^iF(Z) \to R^{i+1}F(X)\to \cdots. $$
\end{proposition}
\begin{proof}
  This is immediate by $R^*F$ being a functor of triangulated categories and the fact that the cohomologies $H^k$ form a conservative system of functors.
\end{proof}
In the situation of \cref{thm: InjectivesAllowDerivedFunctor} the derived functor can be computed explicitly.
Pick some object $X^\bullet$ in $D^+(\AA)$.
By \cref{prop: DerivedCategoryInjectives} there is a quasi-isomorphism $X^\bullet\to I^\bullet$ for some complex of injective objects $I^\bullet$.
Then one has explicitly
$$R^+F(X^\bullet) \cong p_\mathcal{B} \circ F(I^\bullet). $$
Further, if $F$ is exact one has that $F(I^\bullet)$ is quasi-isomorphic to $F(X^\bullet)$ whence $R^+F(X^\bullet)$ is $p_\mathcal{B}\circ F(X^\bullet)$.

In practice, it is often difficult to find a concrete injective resolution.
\begin{definition}
 Let $F:\AA \to \mathcal{B}$ be a left exact functor.
 A object $X$ in $\AA$ is $F$-acyclic if $R^iF(X) = 0$ for all $i\geq 1$.
\end{definition}
Computation derived functors can also be done using $F$-acyclic resolutions.
One can show that injective objects are $F$-acyclic for any left-exact functor.
Hence, this generalises the earlier computations.
\begin{proposition}\label{prop: GrothendieckIsomorphism}
  Let $F:\AA \to \mathcal{B}$ and $G:\mathcal{B}\to \mathcal{C}$ be two additive functors between abelian categories with enough injective objects. Suppose that $F$ is left-exact and that $G$ transforms injective objects into $F$-acyclic objects, then there is an isomorphism
  $$R^+(F\circ G) = R^+F \circ R^+ G.$$
\end{proposition}
\begin{proof}
  This result may be found in \cite[Chapter 1]{dimca2004sheaves}.
\end{proof}
\begin{theorem}[Grothendieck Spectral Sequence]
  Let $F,G$ be as in the previous proposition. Then, for any object $X$ of $\AA$, there is a spectral sequence
  $$E_2^{pq} = R^pF(R^qG(X)) $$
  converging to $R^{p+q}(F\circ G)(X).$
\end{theorem}
\begin{proof}
  This result may be found in \cite[Chapter 1]{dimca2004sheaves}  or \cite[Chapter 5]{weibel1995introduction}.
  The main idea is to consider the double complex $F(I^\bullet)\to J^{\bullet\bullet}$ provided by the dual of the following lemma.
  The double complex $G(J^{\bullet\bullet})$ may be filtrated vertically or horizonally and comparing the spectral sequences from \cref{prop: FiltrationSpectral} yields the desired result.
\end{proof}
\begin{lemma}[Cartan-Eilenberg]\label{lem: Cartan-Eilenberg}
  For any complex $X^\bullet$ there exists a lower half-plane double complex $P^{\bullet\bullet}$ of projective objects such that
  \begin{enumerate}
    \item[(i)] There is a map $P^{0,\bullet}\to X^\bullet$ such that $J^{\bullet,p}\to X^{p}$ is a projective resolution for every $p$.
    \item[(ii)] If $X^p = 0$ the corresponding column $P^{\bullet,p}$ is zero.
    \item[(iii)] The horizontal cocycles, coboundaries and cohomology on $P^{\bullet,p}$ form projective resolutions for the $p$-th cycles, boundaries and cohomology of $X^\bullet$ respectively.
  \end{enumerate}
\end{lemma}
\begin{proof}
  The proof may be found in \cite[Chapter 5]{weibel1995introduction}.
  For later use on the structure of teh Cartan-Eilenberg resolution we remark that the columns $P^{\bullet,p}$ are found as direct sums of projective resolutions for the boundaries and cohomology at level $p$ and $p+1$.
\end{proof}

We conclude this section by considering a few important examples of derived functors which will be used later on.

Let $X$ be a topological space equipped with a sheaf of rings $\AA_X$ which need-not be commutative.
The corresponding categories of complexes of left or right modules are denoted $C^{*,\ell}(\AA_X)$ and $C^{*,r}(\AA_X)$ respectively.
Similarly, the category of complexes of bimodules is denoted $C^{*,\ell r}(\AA_X)$.
Using \cref{thm: InjectivesAllowDerivedFunctor} one can establish that the global sections functor $\Gamma(X,\blank)$ on $C^{*,*}(\AA_X)$ has a derived functor $R^+\Gamma(X,\blank)$.
The cohomology of a sheaf of modules is given by the functors $H^k(X,\blank) := R^k \Gamma(X,\blank)$ and the hypercohomology of a complex of modules is given by the functors $\mathbb{H}^k(X,\blank):= \mathbb{R}^k\Gamma(X,\blank)$.
The cohomology sheaf of a complex $\M^\bullet$ is the sheaf associated to the presheaf $U\mapsto \mathbb{H}^k(U,\M^\bullet)$ and is denoted $\H^k(\M^\bullet)$.
The cohomology sheaves of a module $\M$ are defined similarly and also denoted $\H^k(\M)$.

Cohomology measures the failure of sections to be global.
Correspondingly, acyclic objects are given by sheaves which have no such failure.
\begin{definition}
 A sheaf of $\AA_X$-modules $\F$ is called flabby if for any open $U\subseteq X$ the restriction morphism $\rho_U^X:\F(X)\to\F(U) $ is surjective.
\end{definition}
\begin{proposition}
 If $\F$ is flabby then $\F$ is $\Gamma(X,\blank)$-acyclic.
\end{proposition}
\begin{proof}
  This result may be found in \cite[Chapter 2]{dimca2004sheaves}.
\end{proof}
The hypercohomology of a sheaf complex can be computed using flabby resolutions.
Concrete flabby resolutions may be found using the Godement resolution.
A sheaf $\M$ gives rise to a flabby sheaf $\F$ by the formal product of stalks.
The same argument applies to the cokernel of $\M\to \F$ and iterating the argument yields a flabby resolution for $\M$.
For a sheaf complex $\M^\bullet$ the flabby resolutions of the $\M^{j}$ produce a double complex $\F^{\bullet,\bullet}$.
The total complex of $\F^{\bullet, \bullet}$ yields a flabby resolution of $\M^\bullet$.

Let $f:Y\to X$ be a continuous map between topological spaces.
The direct image of a sheaf $\mathcal{S}$ on $Y$ is the sheaf $f_*\mathcal{S}$ on $X$ defined by
$$(f_*\mathcal{F})(U) = \mathcal{F}(f^{-1}(U)). $$
Suppose that $Y,X$ are equipped with sheaves of rings $\AA_Y, \AA_X$ respectively and that $f_*\AA_Y$ is a $\AA_X$-algebra.
Then the direct image yields a functor from the category of $\AA_Y$-modules to the category of $\AA_X$-modules.
This is a left-exact functor so it may be computed by injective resolutions.
One can verify that flabby sheaves are $f_*$-acyclic so that flabby resolutions may also be used in the computations.

A classical example of a non-exact functor is given by the tensor product.
This may be considered as a bifunctor
$$ \otimes_{\AA_X} : C^{-,\ell}(\AA_X)\times C^{-,r}(\AA_X)\to C^{-,\ell r}(\AA_X)$$
where
$(\M^\bullet \otimes_{\AA_X} \N^\bullet)^n = \oplus_{i + j = n} \M^i \otimes_{\AA_X} \N^j $
with differentials defined at $\M^i \otimes_{\AA_X} \N^j$ by
$d(m\otimes n) = dm \otimes n + (-1)^i m \otimes dn.$
The category of $\AA_X$-modules admits locally free resolutions.
In particular, it has enough projective objects.
Essentially by the remark after \cref{thm: InjectivesAllowDerivedFunctor} it is then possible to construct a derived left-derived functor
$$ \otimes_{\AA_X}^L : D^{-,\ell}(\AA_X)\times D^{-,r}(\AA_X)\to D^{-,\ell r}(\AA_X).$$
This yields $\Tor$-sheaves
$\Tr{\AA_X}{k}(X^\bullet, Y^\bullet) = H^{-k}(X^\bullet\otimes_{\AA_X}^L Y^\bullet).$

A similar procedure applies to the $\Hom_{\AA_X}$-bifunctor which is defined by
$\Hom_{\AA_X}^n(\M^\bullet, \N^\bullet) \ab = \prod_{j\in Z} \Hom_{\AA_X}(\M^j , \N^{n+j}) $
with the differentials on $\Hom_{\AA_X}^n(M^\bullet,N^\bullet)$ given by $d\varphi = d_N\circ \varphi - (-1)^n \varphi \circ d_M$.
There is a induced derived bifunctor
$$R\Hom_{\AA_X}^\bullet(\blank,\blank):: D^{-,\ell}(\AA_X)^{opp}\times D^{+,\ell r}(\AA_X)\to D^{r}(\AA_X).$$
This yields the $\Ext$-sheaves
$\Ex{\AA_X}{n}(M^\bullet,N^\bullet) =  R^n\Hom_{\AA_X}^\bullet(M^\bullet, N^\bullet).$


\section{$t$-structures}
A generalisation of positive and negatively supported complexes is given by the concept of a $t$-structure.
\begin{definition}
 A $t$-structure on a triangulated category $\mathcal{D}$ consists of two strictly full subcategories $\mathcal{D}^{\leq 0}$ and $\mathcal{D}^{\geq 0}$ such that, setting $\mathcal{D}^{\leq n} := \mathcal{D}^{\leq 0}[-n]$ and $\mathcal{D}^{\geq n} := \mathcal{D}^{\geq 0} [-n]$, the following properties hold.
 \begin{enumerate}
   \item[(i)] It holds that $D^{\leq 0}$ is a subcategory of $D^{\leq 1}$ and $D^{\geq 1}$ is a subcategory of $D^{\geq 0}$.
   \item[(ii)] For any objects $X$ in $\mathcal{D}^{\leq 0}$ and $Y$ of $\mathcal{D}^{\geq 1}$ it holds that $\operatorname{Hom}(X,Y) = 0$.
   \item[(iii)] For any object $X$ of $\mathcal{D}$ there is a distinguished triangle
   $$A \to X \to B \xrightarrow{+1} A $$
   with $A$ in $\mathcal{D}^{\leq 0}$ and $B$ in $\mathcal{D}^{\geq 1}$.
 \end{enumerate}
\end{definition}
\begin{definition}
 Let $\mathcal{D}$ be a triangulated category with a $t$-structure. Then $\mathcal{D} = \mathcal{D}^{\leq 0} \cap \mathcal{D}^{\geq 0}$ is called the heart of the $t$-structure.
\end{definition}
In the motivating case of $K^*(\AA)$ and $D^*(\AA)$ the heart of the $t$-structure recovers the original abelian category $\AA$.
\begin{proposition}\label{prop: HeartExtension}
 The heart $\mathcal{D}$ of a $t$-structure is an abelian category which is stable by extensions.
\end{proposition}
\begin{proof}
  This result may be found in \cite[Chapter 5]{dimca2004sheaves}.
\end{proof}
Observe that $D^*(\AA)$ comes equipped with a truncation functors
$\tau_{\leq m}:D^*(\AA)\to D^-(\AA)$ which sends a complex $X^\bullet$ to
$$\tau_{\leq m}X^\bullet : \cdots \to X^{m-1} \to \ker d \to 0 \to 0 \to \cdots$$
and similarly a truncation functor $\tau_{\geq m}$ is defined by
$$\tau_{\geq m}X^\bullet: \cdots \to 0 \to 0 \to \operatorname{coim} d \to X^{m+1}\to \cdots.$$
This generalises to $t$-structures.
\begin{proposition}
 Let $\mathcal{D}$ be a triangulated category with a $t$-structure.
 Then the inclusion of $\mathcal{D}^{\leq n}$ in $\mathcal{D}$ has a right adjoint functor $\tau_{\leq n}$.
 Similarly, the inclusion of $\mathcal{D}^{\geq n}$ in $\mathcal{D}$ has a left adjoint $\tau_{\geq n}$.
\end{proposition}
\begin{proof}
  This result may be found in \cite[Chapter 5]{dimca2004sheaves}.
\end{proof}
Observe that in the example of $D^*(\AA)$ one has that $\tau_{\geq 0} \tau_{\leq 0} X^\bullet$ is the complex with a single entry $H^0(X^\bullet)$.
This generalises to $t$-structures by viewing $\tH^0:= \tau_{\geq 0}\tau_{\leq 0}$ as a functor from $\mathcal{D}$ to it's heart $\mathcal{C}$.
Further let $\tH^i:= \tH^0 \circ T^i$.
\begin{definition}
 A $t$-structure is said to be non-degenerated if $\cap \mathcal{D}^{\leq n} = \cap \mathcal{D}^{\geq n}= \operatorname{Null}$ where $\operatorname{Null}$ denotes the family of objects which are isomorphic to the zero object in $\mathcal{D}$.
\end{definition}
\begin{proposition}
 Let $\mathcal{D}$ be a triangulated category with a $t$-structure. Then $\tH^0:\mathcal{D}\to \mathcal{C}$ is a cohomological functor.
\end{proposition}
\begin{proof}
  This result may be found in \cite[Chapter 5]{dimca2004sheaves}.
\end{proof}
\begin{proposition}
 Let $\mathcal{D}$ be a triangulated category with a non-degenerated $t$-structure. Then the system of functors $\tH^i$ is conservative.
\end{proposition}
\begin{proof}
  This result may be found in \cite[Chapter 5]{dimca2004sheaves}.
\end{proof}
\begin{proposition}\label{prop: tStructCohomD}
  Let $\mathcal{D}$ be a triangulated category with a non-degenerated $t$-structure.
  Then $X\in \mathcal{D}^{\leq 0}$ if and only if $\tH^i(X) = 0$ for $i>0$. Similarly $X\in \mathcal{D}^{\geq 0}$ if and only if $\tH^i(X)= 0$ for $i<0$.
\end{proposition}
\begin{proof}
  This result may be found in \cite[Chapter 5]{dimca2004sheaves}.
\end{proof}
\begin{definition}
 Let $\mathcal{D}_1,\mathcal{D}_2$ be triangulated categories equipped with $t$-structures. A functor of triangulate categories $F:\mathcal{D}_1\to \mathcal{D}_2$ is called left or right $t$-exact if $F(\mathcal{D}_1^{\geq 0}) \subseteq \mathcal{D}_2^{\geq 0}$ or $F(\mathcal{D}_1^{\leq 0}) \subseteq \mathcal{D}_2^{\leq 0}$ respectively.
 The functor $F$ is called $t$-exact if it is left and right $t$-exact.
\end{definition}
\begin{definition}
 Let $\mathcal{D}_1,\mathcal{D}_2$ be triangulated categories equipped with $t$-structures and let $F:\mathcal{D}_1 \to \mathcal{D}_2$ be a functor of triangulated categories.
 The perverse functor $^pF$ associated to $F$ is the induced functor on the hearts $^pF = \tH^0 \circ F \circ j_1$ where $j_1$ denotes the inclusion functor $j_1:\mathcal{C}_1\to \mathcal{C}_2$.
\end{definition}
\begin{proposition}\label{prop: FunctorHeart}
  Let $F:\mathcal{D}_1 \to \mathcal{D}_2$ be a $t$-exact functor of triangulated categories. Then $F$ sends the heart $\mathcal{C}_1$ into the heart $\mathcal{C}_2$ and the induced functor $F:\mathcal{C}_1 \to \mathcal{C}_2$ is naturally isomorphic to $^pF$.
\end{proposition}
\begin{proof}
  This result may be found in \cite[Chapter 5]{dimca2004sheaves}.
\end{proof}
