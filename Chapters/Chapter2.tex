\chapter{The Behaviour of $\A_X^R$-Modules}
\todo[inline]{Mention BVWZ}
\todo[inline]{Stylistically, do we want to put all definitions in italics or use environments? I think italics for general knowledge and environments for specific definitions (+ italics for the word) may work. Check Robin's thesis. }
\todo[inline]{We may want to work out standard results in more detail. Check with Nero. For example, I could prove $\Ch^{rel}$ is independent of the chosen good filtration. }
\section{Relative Holonomic Modules}
Kashiwara's classical approximation of the roots of the $b$-polynomial \todo{cite} relies on ...\todo{describe} being holonomic. This is no longer true in the multivariate case but a refined assumption, called relative holonomicity, due to Maisonobe \cite{maisonobe2016filtration} still holds. This refinement works with $\D_X\times\C[s]$-modules whence one gets characteristic varieties inside $T^*X \times \C^p$\todo{p?}. \todo{Maybe also mention the example Robin put on the whiteboard? Possibly in the main body? } 
\subsection{Notation}
Let $X$ be a smooth complex irreducible algebraic variety of dimension $n$ and denote $\D_X$ for it's sheaf of rings of algebraic differential operators. For a regular commutative $\C$-algebra integral domain $R$ we define a sheaf of rings on $X\times \Spec R$ by
$$\A_{X}^R = \D_X \otimes_\C \O_{\Spec R}; \qquad \A_X = \A_{X}^{\C[s]}.$$

The order filtration $F_p\D_X$ extends to a filtration $F_p\A^R_X = F_p\D_X \otimes_\C \O_R$ on $\A^R_X$ which is called the relative filtration. For any coherent $\A_X^R$-module $\M$ one can find a good filtration, meaning that the graded object $\grrel \M$ is coherent over $\grrel \A_X^R$.\todo{Check that coherence is the right notion, was $\sigma$ additional assumption or automatically fine? } The relative characteristic variety $\Chrel \M$ is defined to be the support of $\grrel \M$ on $T^*X \times \Spec R$ and is independent of the chosen good filtration. 

A coherent $\A_X^R$-module $\M$ is said to be relative holonomic over $R$ if 
$$\Chrel \M = \cup_w \Lambda_w \times S_w$$
for irreducible conic Lagrangian subvarieties $\Lambda_w\subseteq T^*X$ and irreducible closed subvarieties $S_w\subseteq \Spec R$. 
\subsection{Direct Image Functor for $\A_X^R$-modules}
    In this section we state the natural generalisation of the direct image functor for $\D_X$-modules to the relative case of $\A_X^R$-modules. As with $\D$-modules this is the most natural for right-modules.  \todo{This is all general knowledge, should anything be cited? }
    
    Let $\mu:Y\to X$ be some morphism of smooth algebraic varieties.  A-priori it is not even clear what $\A_X^R$-module should correspond to $\A_Y^R$ since there is no natural push forward of vector fields. This issue may be resolved by use of the transfer $(\A_Y^R,\mu^{-1}\A_X^R)$-bimodule $\A_{Y\to X}^R:= \O_Y \otimes_{\mu^{-1}\O_X}\mu^{-1}\A_X^R$. 
    
    For any $\A_Y^R$-module $\M$ the direct image in the derived category $\bD^{b,r}(\A_X^R)$ is defined by $\int \M := \bR\mu_* (\M \otimes_{\D_Y}^L \A_{X\to Y})$ which yields the $\A_X^R$-modules $\int^j \M = \H^j \int \M$.
    To compute $\int^j \M$ a resolution for $\A_{Y\to X}$ is required. 
    \begin{definition}
        Let $\M$ be a right $\A_Y^R$-module, the relative Spencer complex $\Sp_Y^\bullet(\M)$ is a complex of right $\A_Y^R$-modules, concentrated in negative degrees, with $\Sp_Y^{-k}(\M) = \M \otimes_{\O_Y} \wedge^{k}\Theta_Y$ and as differential the right-$\A_Y^R$-linear map $\delta$ given by
        \begin{align*}
            m\otimes \xi_1 \wedge \cdots \wedge \xi_k &\mapsto \sum_{i<j}(-1)^{i+j} m \otimes [\xi_i,\xi_j]\wedge \xi_1 \wedge \cdots \wedge \widehat{\xi}_i \wedge\cdots \widehat{\xi}_j \wedge \cdots \wedge \xi_k\\
            &- \sum_{i=1}^k (-1)^{i} m\xi_i \otimes \xi_1 \wedge \cdots \wedge \widehat{\xi}_i\wedge \cdots \wedge \xi_k 
        \end{align*}
        \begin{lemma}
            The complex of $(\A_X,f^{-1}\A_Y)$-bimodules $\Sp^\bullet_{Y\to X}(\A_Y^R) := \Sp^\bullet_Y(\A_Y^R) \otimes_{\O_Y}\A_{Y \to X}^R$ is a resolution of $\A_{Y\to X}^R$ as a bimodule by locally free left $\A_Y^R$-modules.
        \end{lemma}
        \begin{proof}
            This is analogous to the case of $\D_Y$-modules in \cite[p33]{sabbah2011introduction}.\todo{Should work out the appropriate exercises in the reference sometime.}
        \end{proof}
    \end{definition}
    %Denote by $\bD^{b,r}_{qc}(\A_Y^{R})$ the full subcategory of $\bD^{b,r}(\A_Y^R)$ consisting of those complexes of right $\A_Y^R$-modules whose cohomology sheaves are quasi-coherent over $\O_Y\times \O_{\Spec R}$. Similarly for $\bD^{b,r}_{coh}(\A_Y^R)$ with the cohomology being coherent $\A_Y^R$-modules. 
    %\begin{theorem}
    %    Let $\mu:X\to Y$ be a morphism of nonsingular algebraic varieties. Then the direct image $\int$ takes $\bD^{b,r}_{qc}(\A_Y^{R})$ into $\bD^{b,r}_{qc}(\A_X^{R})$. Moreover, when $\mu$ is proper the direct image takes $\bD^{b,r}_{coh}(\A_Y^R)$ into $\bD^{b,r}_{coh}(\A_X^R)$.
    %\end{theorem}
    %\begin{proof}
    %\todo{See \url{http://www.math.stonybrook.edu/~cschnell/mat615/lectures/lecture18.pdf}}
    %\end{proof}
\subsection{Preservation of Relative Holonomicity} 
\subsection{Kashiwara's Estimate for the Characteristic Variety}
Let $\mu:Y\to X$ be a proper morphism of COMPLEX MANIFOLDS\todo{This requires a different proof than the algebraic case, although the complex manifold can be adjusted to the varieties case but this needs $\mu$ to be proper to get coherence results and this intermediate step is not actually necessary.}. Given a coherent $\A_X^R$-module $\M$ with relative characteristic variety $\Chrel \M$. We desire to estimate $\Chrel \int^j \M$ in terms of $\M$. Such a estimate in the non-relative case is known due to Kashiwara \cite{kashiwara1976b}.  We note that the assumption that $\mu$ is proper can be relaxed but this version will suffice for our purposes.

The original proof by Kashiwara \cite{kashiwara1976b} uses the theory of microlocal differential operators. The proof we consider here is adapted from a proof by Malgrange \cite{malgrange1985images} and we follow the exposition by Sabbah \cite[p36]{sabbah2011introduction}. 

Consider the following cotangent diagram 
$$
\begin{tikzcd}
    & \mu^* T^* X\times \Spec R \arrow[swap]{ld}{T^* \mu} \arrow{rd}{\widetilde{\mu}} & \\
    T^* Y\times \Spec R & & T^*X\times \Spec R
\end{tikzcd}
$$
where the maps $T^*\mu$ and $\widetilde{\mu}$ act on the first component. 
\begin{theorem}
    Let $\M$ be a relative holonomic $\A_Y^R$-module. Then, for any $j\geq 0$, we have \todo{Check required assumptions, really we only need a good filtration but this is apparently nontrivial. } 
    $$\Chrel\left(\int^j \M \right)\subseteq  \widetilde{\mu}\left((T^*\mu)^{-1}(\Chrel \M) \right).$$
\end{theorem}
The first step is to note that a similar inclusion is easy for the $\grrel \A_Y^R$-modules. For any $\grrel \A_Y^R$-module $\mathcal{M}$ define $\int^j \mathcal{M} :=  \H^j\left(\bR \mu_* (\bL(T^*\mu)^* \mathcal{M})\right).$ Note that $(T^*\mu)^*$ produces a sheaf on $\mu^* T^* X \times \Spec R$ by the tensor product  $\blank\otimes_{f^{-1}\O_X\times \O_{\Spec R}} \grrel \A_X^R$. Hence, looking at the supports, the following result is immediate. 
\begin{lemma}
    For any $\grrel\A_Y^R$-module $\mathcal{M}$ it holds that
    $$\Supp \int^j \mathcal{M}\subseteq \widetilde{\mu}\left((T^* \mu)^{-1} \Supp \mathcal{M}\right).$$
\end{lemma}
Applying this to $\grrel \M$ it remains to understand the difference between $\grrel\int^j \M$ and $\int^j \grrel \M$. This may be done using relative Rees modules.
\begin{definition}
    Let $z$ be a new variable. The relative Rees sheaf of rings $\Rees\A_Y^R$ is defined as the subsheaf $\oplus_p F_p \A_Y^R z^p $ of $\A_Y^R \otimes_\C \C[z]$. Similarly, any filtered $\A_Y^R$-module $\M$ gives rise to a $\Rees\A_Y$-module $\Rrel\M := \oplus_p F_p \M z^p$. 
\end{definition}
The following obvious isomorphisms of filtered modules allow us to view the relative Rees module as a way to interpolate between $\M$ and $\grrel\M$
$$\frac{\Rrel\M}{(z-1)\Rrel\M} \cong \M; \qquad \frac{\Rrel\M}{z\Rrel\M} = \grrel \M.$$
Conversely, the second formula may be used to produce a filtered $\A_Y^R$-module from any graded $\Rees\A_Y^R$-module without $\C[z]$-torsion.

One can define $\int^j \Rrel \M$ similarly to the $\D$-module direct image and these are coherent $\A_X^R$-modules by the following adaptation of \todo{theorem 3.4.1 in Sabbah}
