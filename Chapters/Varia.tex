\section{Varia}
\subsection{Relation $\Ch$ and $\Ch^\#$}
In the notation of \cite{budur2020zero} $\Ch^\#$ yields a bound for the classical characteristic variety of cyclic $\D_X$-modules with a $\A_X^{\C[s]}$-structure where $s_1, \ldots, s_n$ act as a vector field on the generator.
Any $P(x,\xi, s) \in \Ann_{\gr^\#\A_X^{\C[s]}} \gr^\# \M$ induces a element of $\Ann_{\gr \D_X} \gr \M$.

\begin{lemma}\label{lem: RelHolGs}
  The sheaf of $\A_X^R$-modules $\M := \A_Y G^s$ is relatively holonomic with relative characteristic variety
  $$\Chrel \M := \bigcup_{J\subseteq \{1,\ldots, n\}} T^\perp Y_J \times \C^p $$
  where $Y_J = \{y\in Y: g_j(y) = 0  \text{ for all }j\in J \}$. \footnote{$T^\perp$ denotes covectors annihilating the tangent space. }
\end{lemma}
\begin{proof}
  Working on a affine open $U$ we may assume that $G^s=x_1^{a_1s_1}\cdots x_k^{a_k s_k} u_{k+1}^{s_{k+1}} \cdots, u_{p}^{s_p}$ for coordinate functions $x_1,\ldots, x_p$, natural numbers $a_1,\ldots, a_k>0$ and invertible sections $u_{k+1},\ldots,u_{p}$ of $\O_Y$.
  We claim that $\A_U G^S \cong \A_U^R/\mathcal{I}$ where $\mathcal{I}$ is the left ideal sheaf generated by the $x_i\partial_i - a_is_i$ and $\partial_{j} - s_{j}u_j^{-1}$.

  Denoting $\varphi:\A_U \to \A_U G^s$ for the obvious surjection we certainly have that $\mathcal{I}$ is a subsheaf of $\ker \varphi$.
  It remains to show that $\ker\varphi / \mathcal{I} = 0$.
  Let $P=\sum c_{\alpha \beta} x^\alpha \partial^\beta$ represent some section in $\ker \varphi/\mathcal{I}$ where the non-zero $c_{\alpha \beta}$ do not vanish in $0$.
  By the relations $\partial_{j} - s_{j}u_j^{-1}$ it can be assumed that the only nonzero components of the multi-indices $\beta$ lie in $1,\ldots, k$.
  By $\A_U$-linear combinations of $x_i \partial_i - a_is_i$ it can further be enforced that the terms are either have $\alpha_i = 0$ or $\beta_i = 0$ for any $i=1,\ldots, k$. When acting on $G^s$ with the remainder the coefficients all end on different monomial coefficients to $G^s$ which means they have to be zero in order for $P$ to be in the kernel. This shows $\ker\varphi = \mathcal{I}$ as desired.

  It follows that $\grrel \mathcal{A}_UG^s \cong \grrel\A_U /\grrel \mathcal{I}$. It holds that $\grrel \mathcal{I}$ is generated by $x_i \xi_i$ and $\xi_j$ whence the result follows.
\end{proof}

\begin{lemma}\label{lem: RelHolGs}
  The sheaf of $\A_X^R$-modules $\M := \A_Y G^s$ is relatively holonomic with relative characteristic variety
  $$\Chrel \M := \bigcup_{J\subseteq \{1,\ldots, n\}} T^\perp Y_J \times \C^p $$
  where $Y_J = \{y\in Y: g_j(y) = 0  \text{ for all }j\in J \}$. \footnote{$T^\perp$ denotes covectors annihilating the tangent space. }
\end{lemma}
\begin{proof}
  Working on a affine open $U$ we may assume that $G^s=x_1^{a_1s_1}\cdots x_k^{a_k s_k} u_{k+1}^{s_{k+1}} \cdots, u_{p}^{s_p}$ for coordinate functions $x_1,\ldots, x_p$, natural numbers $a_1,\ldots, a_k>0$ and invertible sections $u_{k+1},\ldots,u_{p}$ of $\O_Y$.
  We claim that $\A_U G^S \cong \A_U^R/\mathcal{I}$ where $\mathcal{I}$ is the left ideal sheaf generated by the $x_i\partial_i - a_is_i$ and $\partial_{j} - s_{j}u_j^{-1}$.

  Denoting $\varphi:\A_U \to \A_U G^s$ for the obvious surjection we certainly have that $\mathcal{I}$ is a subsheaf of $\ker \varphi$.
  It remains to show that $\ker\varphi / \mathcal{I} = 0$.
  Let $P=\sum c_{\alpha \beta} x^\alpha \partial^\beta$ represent some section in $\ker \varphi/\mathcal{I}$ where the non-zero $c_{\alpha \beta}$ do not vanish in $0$.
  By the relations $\partial_{j} - s_{j}u_j^{-1}$ it can be assumed that the only nonzero components of the multi-indices $\beta$ lie in $1,\ldots, k$.
  By $\A_U$-linear combinations of $x_i \partial_i - a_is_i$ it can further be enforced that the terms are either have $\alpha_i = 0$ or $\beta_i = 0$ for any $i=1,\ldots, k$. When acting on $G^s$ with the remainder the coefficients all end on different monomial coefficients to $G^s$ which means they have to be zero in order for $P$ to be in the kernel. This shows $\ker\varphi = \mathcal{I}$ as desired.

  It follows that $\grrel \mathcal{A}_UG^s \cong \grrel\A_U /\grrel \mathcal{I}$. It holds that $\grrel \mathcal{I}$ is generated by $x_i \xi_i$ and $\xi_j$ whence the result follows.
\end{proof}


Using \cref{thm: InjectivesAllowDerivedFunctor} one can establish that the global sections functor $\Gamma(X,\bar)$ on $C^{*,*}(\AA_X)$ has a derived functor $R^+\Gamma(X,\bar)$.
The cohomology of a sheaf of modules is given by the functors $H^k(X,\blank) := R^k \Gamma(X,\blank)$ and the hypercohomology of a complex of modules is given by the functors $\mathbb{H}^k(X,\blank):= \mathbb{R}^k\Gamma(X,\blank)$.
The cohomology sheaf of a complex $\M^\bullet$ is the sheaf associated to the presheaf $U\mapsto \mathbb{H}^k(U,\M^\bullet)$ and is denoted $\H^k(\M^\bullet)$.
The cohomology sheaves of a module $\M$ are defined similarly and also denoted $\H^k(\M)$.
