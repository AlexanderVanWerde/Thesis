\chapter{$\D_X$-modules and the Riemann-Hilbert Correspondence}\label{Ch: ChapterDX}
\todo{Noetherian}
The subject of this chapter are modules over rings of differential operators.
Throughout $X$ can be a smooth algebraic variety or a complex manifold of dimension $n$.
The rings of differential operators $\D_X$ will be defined formally in the next section.
For the purpose of this section it's sufficient to note that local sections of $\D_X$ are of the form $\sum c_{\alpha } \partial^\alpha$ with $c_{\alpha}$ local sections of $\O_X$ and $\partial^\alpha = \partial_1^{\alpha_1}\cdots \partial_n^{\alpha_n}$.

A $\D_X$-modules gives a canonical description of systems of differential equations.
Consider a system of differential equations
$$\sum_{j=1}^k P_{ij}(x,\partial) f_j = 0; \qquad i= 1,\ldots,m$$
with unknown functions $f_j$ of $\O_X$ and differential operators $P_{ij}$.
The functions $f_j$ are somewhat arbitrary in the description of this system.
For instance, take $g_j=\lambda_j f_j$ for certain non-zero functions $\lambda_j$.
There is then a associated system of equations for $g_j$.
A solution of the $g_j$-system corresponds uniquely to a solution of the $f_j$-system.

Consider the cokernel $\M$ of the map
$$P:\D_X^k \to \D_X^m : \left(Q_1,\ldots, Q_k\right) \mapsto \left(\ldots, \sum_{j=1}^k Q_j P_{ij},\ldots\right).$$
This map is left $\D_X$-linear so $\M$ is a left $\D_X$-module.
Note that it is necessary to distinguish between left and right modules because differential operators form a non-commutative ring.
Direct verification shows that the solutions of the system of differential equations are encoded in $\Hom_{\D_X}(\M,\O_X)$.
More generally, for any $\D_X$-module $\N$ the solutions in $\N$ are encoded by $\Hom_{\D_X}(\M,\N)$.
This shows that $\D_X$-modules provide a canonical description of differential equations.

The goal of this section is to summarise some of the results and definitions which are common knowledge in the field of $\D_X$-modules.
This builds up to the Riemann-Hilbert correspondence which states in very general terms that a system of differential equations is equivalent to it's solutions.
This result is powerful because it yields a connection between algebraic geometry and topology.
A particular instantiation of this correspondence is the connection between Bernstein-Sato polynomials and monodromy.
Finally, we include the estimation of the roots of Bernstein-Sato polynomials due to Kashiwara and Lichtin.

Detailed treatments of the theory of $\D_X$-modules may be found in  \cite{bjork1979rings}, \cite{kashiwara2003d}, \cite{hotta2007d} or \cite{borel1987algebraic}.
\section{$\D_X$-modules}
As stated in the introduction $X$ may denote a smooth algebraic variety or a complex manifold.
\begin{definition}
  The sheaf of differential operators $\D_X$ is the subsheaf of rings in $\mathcal{E}nd_{\O_X}(\O_X,\O_X)$ generated by $\O_X$ and the vector fields $\Theta_X$.
\end{definition}
Observe that $\D_X$ is a sheaf of non-commutative rings.
Given local coordinates $x_1,\ldots, x_n$ on $X$ it holds that
$$\partial_i x_j - x_j\partial_i = \delta_{ij} $$
where $\delta$ denotes the Kronecker delta.
\begin{lemma}
  For any $x\in X$ the stalk $\D_{X,x}$ is left and right Noetherian. Moreover, in the algebraic case $\D_X$ is a left and right Noetherian sheaf of rings.
\end{lemma}
\begin{proof}
  The algebraic and analytic cases may be found in chapters 1 and 4 of \cite{hotta2007d} respectively.
\end{proof}
Giving a left $\D_X$-module is equivalent to giving a $\O_X$-module $\M$ with $\Theta_X$-action such that
$\xi\cdot (fm) = f (\xi \cdot m)  + \xi(f)\ m  $
for any sections $f$ of $\O_X$ and $\xi$ of $\Theta_X$. Similarly, giving a right $\D_X$-module is equivalent to giving a $\O_X$-module $\M$ with $\Theta_X$-action such that
$(mf)\cdot\xi = (m\cdot\xi)f - m\ \xi(f) $ for any sections $f$ of $\O_X^R$ and $\xi$ of $\Theta_X$.

Translation between left and right-modules is possible.
Denote $\omega_X$ for the top-level differential forms.
Then $\omega_X$ comes equipped with the structure of a right $\D_X$-module where vector fields act by the Lie derivative.

For any left $\D_X$-module $\M$ a right $\D_X$-module structure on $\M\otimes_{\O_X}\omega_X$ may be defined by
$$m\otimes \omega \cdot \xi = m\otimes \omega\xi - \xi f \otimes \omega.$$
For any right $\D_X$-module $\M$ a left $\D_X$-module structure on $\Hom_{\O_X}(\omega_X, \M)$ may be defined by
$$(\xi\cdot \varphi)(\omega) = \varphi(\omega \cdot \xi) - \varphi(\omega)\cdot \xi. $$
The following lemma follows by a direct computation.
\begin{lemma}
  The functor $\blank\otimes_{\O_X}\omega_X$ is a equivalence of categories with pseudoinverse  $\Hom_{\O_X}(\omega_X,\blank)$.
\end{lemma}
\subsection{Filtrations}
This non-commutativity exits the typical domain of algebraic geometry.
This can be resolved by consideration of a graded structure.
The essential observation here is that differential operators commute up to a element of lower order.
\begin{definition}
  The order filtration on $\D_X$ is defined inductively to be given by the sheaves of $\O_X$-submodules $F_i \D_X$ such that $F_0\D_X = \O_X$ and $[F_i\D_X, F_i \D_X] \subseteq F_{i-1}\D_X$.
\end{definition}
The term $F_i \D_X$ in the order filtration can be described as containing all differential operators of order less than or equal to $i$.
Indeed, given local coordinates $x_1,\ldots, x_n$ one can show that $F_i\D_X$ is the $\O_X$-module locally generated by $\partial^\alpha = \partial_1^{\alpha_1}\cdots \partial_n^{\alpha_n}$ where $\alpha$ is a multi-index with $\abs{\alpha}\leq n$.
The following observations are immediate.
\begin{lemma}
  The $F_i\D_X$ are coherent $\O_X$-modules and form a exhaustive filtration. This is to say that $\cup_{i\geq 0}F_i\D_X = \D_X$
  and that for any $i,j\geq 0$ it holds that $F_i\D_X \cdot F_j \D_X \subseteq F_{i+j}\D_X$.
\end{lemma}
There is a similar notion of filtrations on $\D_X$-modules $\M$.
Without any harm let's assume that $\M$ is a left $\D_X$-module, the case for right modules is analogous.
A filtration consists of $\O_X$-submodules $F_i\M$ of $\M$ such that $\cup_i F_i \M = \M$ and $F_i\D_X \cdot F_j\M \subseteq F_{i+j}\M$.
Stepping over to the graded object has the advantage that $\gr \D_X$ is commutative by definition of the order filtration whence the classical methods of algebraic geometry are applicable.
One can view $\gr\D_X$ as a subsheaf of $\O_{T^*X}$.
The symplectic structure of $T^* X$ captures part of the non-commutativity.
Indeed, given two differential operators $P, Q$.
Pick local coordinates $x_1,\ldots, x_n$ and decompose $P = \sum_{\alpha} p_\alpha \partial^\alpha $ and $Q = \sum_{\beta} q_\beta \partial^\beta$.
Let $m_1,m_2$ be the maximal values of $\abs{\alpha}$ and $\abs{\beta}$ with non-zero coefficients.
Then the induced elements of $P$ and $Q$ in $\gr \D_X$ are of the form $p = \sum_{\abs{\alpha} = m_1} p_\alpha \xi^\alpha$ and $q = \sum_{\abs{\beta} = m_2}q_\beta \xi^\beta $ where $\xi_i$ is the induced element of $\partial_i$.
On the other hand, the induced element of $PQ - QP$ is $\sum_{i=1}^n\frac{\partial p}{\partial \xi_i}\frac{\partial q}{\partial x_i} - \frac{\partial q}{\partial \xi_i}\frac{\partial p}{\partial x_i}$.
This is precisely $\{p,q\}$ where $\{\blank,\blank\}$ is the Poisson bracket.

Denote $\pi$ for the projection of $T^*X \to X$.
Then any $\gr\D_X$-module $\mathcal{M}$ has a corresponding module on $T^* X$ defined by $\O_{T^*X} \otimes_{\pi^{-1}\gr \D_X} \mathcal{M}$.
By abuse of notation this module is still denoted $\mathcal{M}$ and it will always be implicitly assumed that $\gr \D_X$-modules live on $T^*X$ unless it is explicitly mentioned otherwise.
In particular, for a filtration of the $\D_X$-module $\M$ the graded object $\gr \M = \oplus_i F_i \M / F_{i-1}\M$ is a $\gr \D_X$-module.

\begin{proposition}\label{prop: GoodFiltration}
  A $\D_X$-module $\M$ is coherent if and only if it locally admits a filtration such that $\gr \M$ is a coherent $\gr \D_X$-module. Such a filtration is called a good filtration. Moreover, this filtration can be taken globally if $X$ is a variety.
\end{proposition}
\begin{proof}
  A proof of this result may be found in \cite[Chapter 2]{hotta2007d}.
\end{proof}
\begin{proposition}
  Let $\M$ be a coherent $\D_X$-module, then the support of $\grrel\M$ in $T^* X$ is a independent of the chosen good filtration. It is called the characteristic variety of $\M$ and denoted $\Ch \M$.
\end{proposition}
\begin{proof}
  A proof of this result may be found in \cite[Chapter 2]{hotta2007d}.
\end{proof}
\begin{proposition}\label{prop: StructureChVar}
  Let $\M$ be a coherent $\D_X$-module, then $\Ch \M$ is a homogeneous and involutive closed subset of $T^* X$.
\end{proposition}
\begin{proof}
  These results may be found in \cite[Chapter 2]{kashiwara2003d}.
\end{proof}
Characteristic varieties behave well with respect to quotients and submodules.
\begin{proposition}\label{prop: SESBehaviourChar}
  Consider a short exact sequence of coherent $\D_X$-modules
  $$0 \to \M_1 \to \M_2 \to \M_3 \to 0 $$
  then it holds that
  $$\Ch \M_2 = \Ch \M_1 \cup \Ch \M_3. $$
\end{proposition}
\begin{proof}
  A good filtration on $\M_2$ induces good filtrations on $\M_1$ and $\M_3$ and one has a short exact sequence
  $$0\to \gr \M_1 \to \gr \M_2 \to \gr \M_3 \to 0 $$
  whence the result follows.
\end{proof}
Let $f$ be a non-constant function of $\O_X$ and $s$ a new variable.
For the purposes of the study of Bernstein-Sato polynomials the coherent $\D_X$-module $\D_X f^s$ is essential.
The corresponding characteristic variety is understood and may provide some intuition for general characteristic varieties.
\begin{proposition}\label{prop: Charfs}
  The characteristic variety of the coherent $\D_X$-module $\D_Xf^s$ is the closure of
  $$W_f = \{(x,sf^{-1}df(x)); \qquad f(x)\neq 0, s\in \C \}$$
  in $T^*X$.
\end{proposition}
\begin{proof}
  This proof of this result may be found in \cite{kashiwara1976b}.
\end{proof}
\begin{proposition}\label{prop: IsotropicAndDominate}
  One can write $\Ch \D_X f^s = \Lambda \cup W$ for some isotropic variety $\Lambda \subseteq T^*X$ and a irreducible $(n+1)$-dimensional variety $W$ which dominates $X$.
\end{proposition}
\begin{proof}
  The proof of this result may be found in \cite{kashiwara1976b}. It follows from \cref{prop: Charfs} by establishing that the part of the closure of $W_f$ above $f=0$ is isotropic.
\end{proof}

\subsection{Direct Image}
In this section we describe the direct image of $\D_Y$-modules.
Let $\mu:Y\to X$ be some morphism of smooth algebraic varieties or complex manifolds.

A-priori, it is not even clear what $\D_X$-module should correspond to $\D_Y$ since there is no natural push forward of vector fields.
For example consider the case where $\mu$ is the embedding of a curve in $X$.
This issue may be resolved by use of the transfer $(\D_Y,\mu^{-1}\D_X)$-bimodule $\D_{Y\to X}:= \O_Y \otimes_{\mu^{-1}\O_X}\mu^{-1}\D_X$.
Here, the right $\mu^{-1}\D_X$-module structure is just the action on the second component and the left $\D_Y$-module structure is defined by
$$f\cdot (g\otimes \mu^{-1}h_X) = fg \otimes \mu^{-1}h_X; \qquad \xi\cdot (g\otimes\mu^{-1}h_X) = \xi g \otimes \mu^{-1}h_X + g \otimes T\mu(\xi)\mu^{-1}h_X $$
for any sections $f$ of $\O_Y$ and $\xi$ of $\Theta_Y$.
Here $T\mu(\xi)$ is a local section of $\O_Y\otimes_{\mu^{-1}\O_X} \mu^{-1}\Theta_X$.
\begin{definition}
  The direct image functor $\int_\mu$ from $D^{b,r}(\D_Y)$ to $D^{b,r}(\D_X)$ is defined to be $R\mu_* (\blank\otimes_{\D_Y}^L \D_{Y\to X})$.
  For any $\D_Y$ module $\M$ the $j$-th direct image is the $\D_X$-modules $\int_\mu^j \M = \H^j \int_\mu \M$.
  The subscript $\mu$ will be suppressed whenever there is no ambiguity.
\end{definition}
Let us remark that a explicit free resolution for the transfer module is known.
This involves the Spencer complex $\Sp_Y^\bullet(\M)$ of a $\D_Y$-module $\M$ with $\Sp_Y^{-k}(\M) = \M\otimes_{\O_Y}\wedge^k \Theta_k$.
The details may be found in \cite{sabbah2011introduction}.
A direct image functor for left $\D_Y$-modules is induced as  $$\int\M := R\Hom_{\O_X}(\omega_X,\int\IntMinspace  \M\otimes_{\O_Y} \omega_Y)).$$

The definition for the direct image functor is somewhat subtle due to passing through derived categories but many nice properties follow.
Most notably, it is immediate from the derived definition that one gets a long exact sequence.
\begin{proposition}
    For any short exact sequence of $\D_Y$-modules
    $$0 \to \M_1 \to \M_2 \to \M_3 \to 0 $$
    there is a long exact sequence in direct images
    $$0 \to \int^0\M_1 \to \int^0 \M_2 \to \int^0 \M_3 \to \int^1 \M_1 \to \cdots.$$
\end{proposition}
\begin{proposition}
  Let $\mu:Z\to Y$ and $\nu:Y\to X$ be a morphisms of smooth varieties. Then there is a isomorphism of functors $\int_{\nu\circ \mu} \cong \int_\nu \int_\mu$.
\end{proposition}
\begin{proof}
  A proof may be found in \cite[Chapter 6]{borel1987algebraic}.
\end{proof}
A similar theorem applies to complex manifolds provided $\mu$ is proper.
Denote $D_{coh}^{*,*}(\D_X)$ for the full subcategory of $D^{*,*}(\D_X)$ consisting of those complexes of $\D_X$-modules with coherent cohomology.
The coherence properties of the direct image in the analytic case require the following notion.
\begin{definition}
  A $\D_Y$-module $\M$ is said to be $\mu$-good if there exists a open cover $\{V_j\}_{j\in J}$ of $X$ such that $\M$ admits a good filtration on $\mu^{-1}(V_j)$ for any $j\in J$.
\end{definition}
Note that, by \cref{prop: GoodFiltration} any coherent $\D_Y$-module on a algebraic variety is $\mu$-good.
\begin{theorem}\label{thm: MuGoodCoherent}
  Let $\M$ be a $\mu$-good $\D_Y$-module and suppose that $\mu$ is proper on the support of $\M$.
  Then, $\int \M$ has coherent cohomology.
\end{theorem}
\begin{proof}
  A proof may be found in \cite[Chapter 3]{sabbah2011introduction}
\end{proof}
Consider the following cotangent diagram.
$$
\begin{tikzcd}
    & \mu^* T^* X\arrow[swap]{ld}{T^* \mu} \arrow{rd}{\widetilde{\mu}} & \\
    T^* Y & & T^*X
\end{tikzcd}
$$
\begin{proposition}\label{prop: EstimateProper}
  Let $\M$ be a $\mu$-good $\D_Y$-module and suppose that $\mu$ is proper on the support of $\M$. Then, for any $j\geq 0$
  $$\Ch\left(\IntJ{j} \M \right)\subseteq  \widetilde{\mu}\left((T^*\mu)^{-1}(\Ch \M) \right).$$
\end{proposition}
\begin{proof}
  See remark 2.5.2 in \cite[Chapter 2]{hotta2007d} for the algebraic case or \cite[Chapter 3]{sabbah2011introduction} for the analytic case.
\end{proof}
\section{Riemann-Hilbert Correspondence}\label{sec: Riemann-Hilbert}
This section concerns the Riemann-Hilbert correspondence which states that a system of differential equations is equivalent to it's system of solutions.
The systems of differential equations are encoded in regular holonomic $\D_X$-modules.
The solutions are given by perverse sheaves.
\subsection{Holonomic Modules}
A particularly nice class of $\D_X$-modules are given by maximally overdetermined systems of differential equations.
This is to say that there are many relations for $\M$ or equivalently that $\Ch\M$ is small.
Observe that by the involutive part of \cref{prop: StructureChVar} it follows that $\dim \Ch\M \geq n$.
\begin{definition}
  A coherent $\D_X$-module $\M$ is called holonomic if
  $\dim \Ch \M  = n.$
\end{definition}
The full subcategory of $D^{*,*}(\D_X)$ consisting of complexes with holonomic cohomology is denoted $D_h^{*,*}(\D_X)$.
For technical purposes it is mostly important that holonomic modules have finiteness properties.
\begin{proposition}\label{prop: FiniteLength}
  Let $\M$ be a holonomic $\D_X$-module. Then, for any $x\in X$, the stalk $\M_x$ is a $(\D_X)_x$-module of finite length.
\end{proposition}
\begin{proof}
  This result may be found in \cite[Chapter 4]{kashiwara2003d}.
\end{proof}
\begin{proposition}\label{prop: HomAlgebraic}
  Let $\M$ be a holonomic $\D_X$-module. Then $\Hom_{\D_X}(\M,\M)$ is $\C$-algebraic. This is to say that for any $\varphi \in\Hom_{\D_X}(\M,\M) $ there exists some polynomial $b$ with coefficients in $\C$ such that $b(\varphi)=0$.
\end{proposition}
\begin{proof}
  This result may be found in \cite[Chapter 5]{bjork1979rings}.
\end{proof}
\begin{proposition}
  Let $\M$ be a holonomic $\D_X$-module and suppose that $\mu:Y\to X$ is proper on the support of $\M$. Then $\int \M$ has holonomic cohomology. Moreover, the assumption that $\mu$ is proper may be removed if in the algebraic case.
\end{proposition}
\begin{proof}
  This result may be found in \cite[Chapter 4]{sabbah2011introduction} or \cite[Chapter 3]{hotta2007d} in the algebraic case.
\end{proof}
When $\mu$ is proper this may be established by combining \cref{prop: EstimateProper} with the following results.
\begin{lemma}
  Let $\M$ be a holonomic $\D_X$-module. Then $\M$ has a globally defined good filtration.
\end{lemma}
\begin{proof}
  This result may be found in \cite[Chapter 4]{sabbah2011introduction}.
\end{proof}
\begin{lemma}\label{lem: IsotropicDirectImage}
  Let $\mu:Y\to X$ be a proper morphism and $V\subseteq T^*Y$ an isotropic subvariety. Then $\widetilde{\mu}((T^*\mu)^{-1}(\Chrel \M) )$ is also isotropic.
\end{lemma}
\begin{proof}
  This result may be found in \cite{kashiwara1976b}.
\end{proof}
\subsection{Regular singularities}
This section is based on \cite[Chapter 5]{kashiwara2003d} and \cite[Chapter 6]{hotta2007d}.
Let $X=\C$ considered with it's analytical topology and consider a ordinary differential operator
$P(x,\partial) = \sum_{k=0}^m a_k(x)\partial^k.$
Suppose that $a_m(x)\neq 0$ for any $x\neq 0$.
Then $\M :=\D_X/\D_X P$ is locally of the form $\O_X^m$ as a $\D_X$-module near any point $x\neq 0$.
In particular the solutions $\Hom_{\D_X}(\M,\O_X)$ form a locally constant sheaf of rank $m$ outside of $0$.
The solutions near zero may be more subtle due to monodromy.

Observe that $\Ch \M\subseteq \{(x,\xi): x\xi = 0\}$.
Hence, for any filtration on $\M$ there exists some $N>0$ such that
$(x\xi)^N \gr\M = 0 $.
\begin{proposition}
  The following conditions are equivalent.
  \begin{enumerate}
    \item There exists a filtration on $\M$ such that $x\xi\gr\M = 0$.
    \item The equation $P(x,\partial)u$ has $m$ linearly independent solutions of the form $x^\lambda \sum_{j=0}^s u_j \log(x)^j $
    near $0$ for some $s\geq 0$, $\lambda \in \C$ and holomorphic $u_j$ if and only $P$ has a regular singularity in $0$.
  \end{enumerate}
\end{proposition}
\begin{proof}
  This result may be found in \cite[Chapter 5]{kashiwara2003d}.
\end{proof}
If these two equivalent conditions are satisfied one calls $0$ a regular singularity of $\M$.
This has the following generalisation to higher dimensions.
\begin{definition}
   Let $\M$ be a holonomic $\D_X$-module on a complex manifold $X$ with characteristic variety determined by some ideal sheaf $\mathcal{I}$. Then $\M$ is called regular holonomic if it it admits a filtration such that $\mathcal{I} \gr(\M) = 0$.
\end{definition}
Denote $D^{**}_{rh}(\D_X)$ for the full subcategory of $D^{**}(\D_X)$ consisting of complexes with regular holonomic cohomology.

It appears that these definitions should generalise directly to the algebraic situation.
However, this has unintended consequences for the Riemann-Hilbert correspondence.
For a example, let $X = \C$ as before and consider the regular holonomic $\D_X$-modules $\O_X$ and $\M := \D_X/\D_X(\partial - 1)$.
These are analytically isomorphic by the map which sends $f$ to $f\exp(x)$.
In particular the Riemann-Hilbert correspondence shows that they have isomorphic systems of solutions.
However, $\O_X$ and $\M$ are not algebraically isomorphic.
This seems to suggest that the equivalence between differential equations and their systems of solutions would not hold in the algebraic case.
The problem is that $\M$ is not regular at infinity.

There are a number of equivalent definitions for regularity in the algebraic case.
The following definition expresses that the analytic definition may be used provided one adds the points at infinity.
This uses the analytification functor on coherent sheaves which is provided by the GAGA principle.
\begin{definition}
  Let $\M$ be a holonomic $\D_X$-module on a smooth variety $X$. Denote $\iota:X \to \overline{X}$ for the smooth completion of $X$. Then $\M$ is called regular if $(\int_\iota \M)^{an}$ is regular holonomic on the complex manifold $\overline{X}^{an}$.
\end{definition}
\subsection{Perverse Sheaves}
  Classically, the solutions to a differential equation on a vector bundle produce a local system.
  It is unreasonable to expect local systems in the case of general $\D_X$-modules since their support could be a proper subvariety.
  \begin{definition}
    Let $X$ be a complex manifold. A stratification of $X$ consists of a locally finite partition $X  = \sqcup_{j\in J} X_j$ into connected locally closed subsets, called strata, such that
    \begin{enumerate}
      \item[(i)] For any $j\in J$ the fronteer $\partial X_j = \overline{X}_j\setminus X_j$ is a union of strata.
      \item[(ii)] For any $j\in J$ the spaces $\overline{X}_j$ and $\partial X_j$ are closed complex analytic subspaces.
    \end{enumerate}
  \end{definition}
  The same definition applies on algebraic varieties by replacing the analytic subspaces by subvarieties.
  \begin{definition}
     A $\C_X$-module $\mathcal{F}$ is called a constructible sheaf on $X$ if there exists a stratification $X = \sqcup_{\alpha\in A}X_\alpha$ such that $\mathcal{F}\vert_{X_\alpha}$ is a local system of finite rank on $X_\alpha$ for any $\alpha \in A$.
  \end{definition}
  Denote $D^b_c(X)$ for the full subcategory of $D^b(\C_X)$ consisting of complexes with constructible cohomology.
  Such complexes are called constructible.

  For a constructible complex $\mathcal{F}^\bullet$ in $D^b_c(X)$ the supports and cosupports are defined dually by
  $$\supp^m\mathcal{F}^\bullet = \supp \H^m \mathcal{F}^\bullet; \qquad \operatorname{cosupp}^m\mathcal{F}^\bullet = \supp^{-m}\mathbb{D}\mathcal{F}^\bullet.$$
  The support $\supp \mathcal{F}^\bullet$ is the closure of the union of the $\supp^m\mathcal{F}^\bullet$.
  \begin{theorem}
    Let $\mathcal{F}^\bullet$ be a constructible complex on $Y$ and consider a morphism $\mu:Y\to X$ which is proper on $\supp \mathcal{F}^\bullet$. Then $Rf_*(\mathcal{F}^\bullet)$ is constructible on $X$.
  \end{theorem}
  \begin{proof}
    This result may be found in \cite[Chapter 4]{dimca2004sheaves}.
  \end{proof}
  \begin{theorem}
    Let $\mathcal{F}^\bullet$ be a complex of $D^b(\C_X)$. Then $\mathcal{F}^\bullet$ is constructible if and only if the dual $\mathbb{D}\mathcal{F}^\bullet := R \Hom_\C(\mathcal{F}^\bullet,\C_X)$ is constructible.
  \end{theorem}
  \begin{proof}
    This result may be found in \cite[Chapter 4]{dimca2004sheaves}.
  \end{proof}
  Let $D^{\leq 0}(X)$ denote the full subcategory of $D^b_c(X)$ consisting of complexes with $\dim \supp^{-m} \mathcal{F}^\bullet < m$ and $\dim \supp^m \mathcal{F}^\bullet = 0$ for all $m\geq 0$.
  Dually $D^{\geq 0}(X)$ consists of complexes with $\dim \op{cosupp}^{-m}\mathcal{F}^\bullet <m$ and $\dim \op{cosupp}^m \mathcal{F}^\bullet = 0$ for all $m\geq 0$.
  \begin{proposition}
    The pair $(D^{\leq 0}(X), D^{\geq 0}(X))$ is a non-degenerated $t$-structure on the triangulated category $D^{b}_c(X)$.
  \end{proposition}
  \begin{proof}
    This result may be found in \cite[Chapter 5]{dimca2004sheaves}.
  \end{proof}
  \begin{definition}
    The heart of the $t$-structure on $D^b_c(X)$ are called the perverse sheaves $\operatorname{Perv}(X) = D^{\leq 0}(X)\cap D^{\geq 0}(X).$
  \end{definition}
  Observe that the objects in $\operatorname{Perv}(X)$ are not sheaves but complexes.
  The reason for the terminology perverse sheaves is that the functor $U\mapsto \operatorname{Perv}(U)$ has the gluing property of sheaves.
  More precisely, it is a stack.
  Perverse sheaves still capture the local systems.
  \begin{theorem}
    Let $X$ be a complex manifold of dimension $n$. Then $\mathcal{L}[n]$ is a perverse sheaf on $X$ for any local system $\mathcal{L}$ on $X$.
  \end{theorem}
  \begin{proof}
    This result may be found in \cite[Chapter 5]{dimca2004sheaves}.
  \end{proof}
  The following are immediate from \cref{prop: HeartExtension} and \cref{prop: tStructCohomD}.
  \begin{proposition}
    A constructible complex $\mathcal{F}^\bullet$ is a perverse sheaf if and only if $^pH(\mathcal{F}^\bullet) = 0$ for all $k\neq 0$.
  \end{proposition}
  \begin{proposition}
    For any distinguished triangle in $D_c^b(X)$
    $$\mathcal{F}^\bullet \to \mathcal{G}^\bullet \to \mathcal{H}^\bullet \xrightarrow{+1} $$
    it holds that if two terms are perverse sheaves then so is the third.
  \end{proposition}
\subsection{Riemann-Hilbert Correspondence}
  Recall that $\Hom_{\D_X}(\M, \O_X)$ encodes the solutions of a system of differential equations.
  More generally, the solutions complex is the functor $\operatorname{Sol}(\blank) := R \Hom_{\D_X}(\blank,\O_X)[n]$ from $D^{b,\ell}(\D_X)^{opp}$ to $D^{b}(\C_X)$.
  The contravariance may be fixed using the duality functor
  $$\mathbb{D} = R\Hom_{\D_X}(\blank,\D_X)\otimes_{\O_X}^L\omega_X^{-1}[n]$$
  from $D^{b,*}(\D_X)^{opp}$ to $D^{b,*}(\D_X)$.
  The de Rham complex of $\M^\bullet$ is defined by
  $$\DR(\M^\bullet):=\Omega_X^\bullet \otimes_{\D_X} \M^\bullet[n]. $$
  \begin{proposition}
    There is a natural isomorphism $\Sol(\blank) \cong \DR(\mathbb{D}\blank)$.
  \end{proposition}
  \begin{proof}
    This result may be found in \cite[Chapter 5]{dimca2004sheaves}.
  \end{proof}
  \begin{proposition}
    For any holonomic complex $\M^\bullet$ in $D_h^{b,\ell}(\D_X)$ the complexes $\Sol(\M^\bullet)$ and $\DR(\M^\bullet)$ are constructible.
  \end{proposition}
  \begin{proof}
    This result may be found in \cite[Chapter 5]{dimca2004sheaves}.
  \end{proof}
  We are finally ready to state the Riemann-Hilbert correspondence on the equivalence between differential equations and their solutions.
  \begin{theorem}[Riemann-Hilbert Correspondence]
    The de Rham functor $\DR:D_{rh}^{b,\ell}(\D_X) \to D_c^b(X)$ is a $t$-exact equivalence of categories and commutes with direct images.
  \end{theorem}
  \begin{proof}
    This result may be found in \cite[Chapter 5]{dimca2004sheaves}.
  \end{proof}
  \begin{corollary}
    The de Rham functor is a equivalence of categories between the category of regular holonomic $\D_X$-modules and $\operatorname{Perv}(X)$.
  \end{corollary}
  \begin{proof}
    Follows from the Riemann-Hilbert correspondence and \cref{prop: FunctorHeart}.
  \end{proof}
\section{Monodromy and Bernstein-Sato Polynomials}\label{sec: MonodromyBS}
The goal of this section is to motivate Bernstein-Sato polynomials by their connection to monodromy.
Further, we include Kashiwara and Lichtin's proof for the estimation of the roots of the Bernstein-Sato polynomial.
This proof is a important framework for the generalised proof in \cref{ch: ChapterRelHol}.

We focus on the local analytic case.
The algebraic case will be discussed in detail in the next chapter.
Let $X$ be a complex manifold and consider a holomorphic function $f:X \to \C$.
Let $x\in X$ with $f(x) = 0$, we study the function germ $f:(X,x)\to (\C,0)$.
\subsection{Monodromy}
\begin{theorem}
  Let $B\subseteq X$ be a small ball and pick $t\in \C^\times$ close to zero. The diffeomorphism class of $F_{f,x} := f^{-1}(t)\cap B $  is independent of the choice of $t$.
\end{theorem}
\begin{proof}
  The isolated singularity case is established in \cite{milnor1968singular} and the general case was established by Hamm-Le.
\end{proof}
\begin{definition}
  The diffeomorphism class of $F_{f,x}$ is called the Milnor fiber.
\end{definition}
Going over a loop around the origin in $T$ induces a endomorphism $M^*$ on the cohomology $H^j(F_{f,x},\C_{F_{f,x}})$ for every $j\in \Z$.
This is called the monodromy action and only depends on the local singularity $f^{-1}(0)\cap B$.
In particular, this means that the eigenvalues of $M^*$ on $H^j(F_{f,x},\C_{F_{f,x}})$ are invariants of the singularity.
If $\lambda \in \C$ is a eigenvalue of $M^*$ on some $H^j(F_{f,x},\C_{F_{f,x}})$ it is called a eigenvalue of monodromy.


\begin{definition}
  Let $s$ be a new variable.
  The local Bernstein-Sato polynomial $b_{f,x}(s)\in \C[s]$ is the monic polynomial of minimal degree such that there exists some differential operator $P(x,\partial,s)$ in $\D_{X,x}\otimes_{\C}\C[s]$ with
  $$P(x,\partial,s) f^{s+1} = b_{f,x}(s) f^s$$
  in the stalk at $x$.
\end{definition}
The existence of Bernstein-Sato polynomials was originally proved by I.N. Berntein.
The following section gives a alternative method to prove the existence.
In fact, it will even be established that all roots of $b_{f,x}(s)$ are negative rational numbers.

Let $Z(b_{f,x})$ denote the set of zeros of the Bernstein-Sato polynomial.
These are connected to monodromy by the following classical theorem due to Malgrange and Kashiwara.
\begin{theorem}\label{thm: EigMonodromy}
  The set $\op{Exp}(Z(b_{f,x}))$ is equal to the set of eigenvalues of monodromy where  $\operatorname{Exp}(x) = \exp(2\pi ix).$
\end{theorem}
\begin{proof}
  This theorem may be found in \cite{budur2015bernstein}.
\end{proof}
Monodromy is a topological notion whereas the Bernstein-Sato polynomial is defined in terms of $\D_X$-modules.
This suggest that the Riemann-Hilbert correspondence is involved.
Indeed, monodromy of the Milnor fiber can be encoded in a constructible complex so that the Riemann-Hilbert correspondence is applicable.

Let $\widetilde{\C}^\times$ denote the universal cover of $\C^\times$ and consider the projection $p:X\times \widetilde{\C}^\times \to X$.
Denote $\iota:f^{-1}(0)\to X$ for the inclusion map.
\begin{definition}
  Deligne's nearby cycle functor from $D^b_c(X)$ to $D_c^b(f^{-1}(0))$ is given by $\psi_f:= L\iota^* \circ Rp_*\circ Lp^*$.
\end{definition}
Denote $\iota_x:\{x\}\to f^{-1}(0)$ for the inclusion map.
The following theorem is due to Deligne.
\begin{theorem}
  There is a isomorphism
  $$H^i(F_{f,x}, \C_{F_{f,x}}) \cong \mathbb{H}^i(L\iota_x^* (\psi_f \C_X)) $$
  and the monodromy action on the cohomology of the Milnor fiber corresponds with the action of the covering transformations $\widetilde{\C}^\times \to \widetilde{\C}^\times$ on the nearby cycles.
\end{theorem}
\begin{proof}
  This theorem may be found in \cite{budur2015bernstein}.
\end{proof}
To describe the $\D_X$-theoretic counterpart of this constructible complex requires the technical notion of $V$-filtrations.
The interested reader may find these concepts in \cite{budur2015bernstein}.
\subsection{Estimation of $Z(b_{f,0})$}
Estimation of $Z(b_{f,x})$ is easy if $f$ is a monomial.
For instance, if $f = x_1^{r_1}\cdots x_n^{r_n}$ then set $P = \partial_1^{r_1}\cdots \partial_n^{r_n}$ to get
$$P f^{s+1} = \prod_{i=1}^n (r_i s + r_i)\cdots (r_i s + 1) f^{s}. $$
The main idea employed in the proof by \cite{kashiwara1976b} is that one can reduce to this situation by resolution of singularities and $\D_X$-module direct images.
\begin{definition}
  Let $D$ be a divisor on $X$. A strong log resolution of $(X,D)$ consists of a projective morphism $\mu:Y\to X$ which is a isomorphism over the complement of $D$ such that $Y$ is smooth and $\mu^*D$ is a simple normal crossings divisor.
\end{definition}
Let $D$ be the divisor determined by $f$.
By Hironaka's resolution of singularities one can find a resolution $\mu:Y\to X$ for $(X,D)$.
Let $g = f\circ \mu$ denote the pullback of $f$ to $Y$, then $\mu^*D = \sum_{i=1}^r \operatorname{mult}_{E_i}(g) E_i$ where $\operatorname{mult}_{E_i}(g)$ denotes the order of vanishing of $g$ on the irreducible hypersurface $E_i$.
Using that $g$ is locally a monomial Kashiwara was able to establish the following estimate by consideration of the direct image of the $\D_Y$-module $\D_Y g^s$.
\begin{theorem}
  Every root of $b_{f,x}(s)$ is of the form $ -c_i/\operatorname{mult}_{E_i}(g) $ with $c_i\in \mathbb{Z}_{>0}$.
  In particular $Z(b_{f,x})\subseteq \mathbb{Q}_{<0}$.
\end{theorem}
\begin{proof}
  This is proved in \cite{kashiwara1976b}.
\end{proof}
Combining this estimate with \cref{thm: EigMonodromy} one gets the following theorem.
\begin{theorem}
  The eigenvalues of monodromy are roots of unity.
\end{theorem}
\cite{lichtin1989poles} improved the estimate by similar computations for the right $\D_Y$-module $\M$ spanned by $ g^s \mu^*(dx)$ inside $\D_Yg^s\otimes_{\O_Y} \omega_Y$.
The advantage of this approach is that $\mu^*(dx)$ involves the local behaviour of $\mu$ which can detect redundant divisors in $\mu^*D$.
The vanishing of $\mu^*(dx)$ is measured in the relative canonical divisor $K_{Y/X} = \sum k_i E_i$.
\begin{theorem}\label{thm: LichtinEstimate}
  Every root of $b_{f,x}(s)$ is of the form $-(k_i + c_i)/\operatorname{mult}_{E_i}(g)$ with $c_i\in \mathbb{Z}_{>0}$.
\end{theorem}
We will now reproduce the proof for this improved estimate following Lichtin and Kashiwara.
\\

One can ensure that multiplication by $s$ stays inside $\D_X f^s$ with the following trick.
Introduce a new coordinate $x_{n+1}$ and set $\widetilde{f}= x_{n+1}f$ on the space $X\times \C$.
Then $x_{n+1}\partial_{n+1}$ acts like $s$ on $\widetilde{f}^s$.
The induced map $Y\times \C \to X\times \C$ is a strong log resolution for the divisor determined by $\widetilde{f}$.
Now suppose we can prove \cref{thm: LichtinEstimate} for $\widetilde{f}$.
Then, the theorem also follows for $f$ due to the following result.
\begin{lemma}
  The Bernstein-Sato polynomial $b_{f,x}(s)$ is a divisor of $b_{\widetilde{f},x}(s)$.
\end{lemma}
\begin{proof}
  Take local coordinates $x_1,\ldots, x_{n+1}$ and let $P$ be in the stalk $\D_{X\times \C,x}$ such that $b_{\widetilde{f},x} \widetilde{f}^s = P \widetilde{f}^{s+1}$.
  Expand $P = \sum_{j=1}^N P_j  \partial_{n+1}^{j}$
  with coefficients $P_j$ in $\D_{X,x}$.
  Then
  $$x_{n+1}^N b_{\widetilde{f},x}(s) \widetilde{f}^s = \left(\sum_{j=1}^k (s + 1)^{j} \sum_\alpha Q_{\alpha} \partial_1^{\alpha_1}\cdots \partial_n^{\alpha_n} \right)\widetilde{f}^{s+1}$$
  where the $P_j$ were expanded as polynomials in $\partial_1,\ldots,\partial_n$ with coefficients $Q_{\alpha}$ in $\O_{X,x}$.

  Observe that $\partial_1,\ldots, \partial_n$ act on the formal symbol $\widetilde{f}^{s+1}$ the same as they act on the formal symbol $f^{s+1}$.
  Expand the $Q_{\alpha}$ as power series in $x_1,\ldots,x_{n+1}$ and identify powers of $x_{n+1}$ on both sides for the desired functional equation.
\end{proof}
For notational simplicity we will simply write $f$ instead of $\widetilde{f}$ and $X$ instead of $X\times \C$.
The dimension of $X$ will still be denoted with $n$.

Let $t$ be a new variable.
The sheaf of rings $\D_X\langle s,t\rangle$ is found from $\D_X$ by adjoining $s$ and $t$ subject to $ts -st = 1$ where $s,t$ commute with $\D_X$.
The notation with the angle brackets is to emphasise that $s$ and $t$ do not commute with each other.
One can view $\D_X f^s$ as a $\D_X\langle s,t\rangle$-module by the action $t P(s)f^s = P(s+1) f^{s+1}$ for any differential operator $P$ in $\D_X[s]$.
In this notation the functional equation for $b_{f,x}$ means that $b_{f,x}\in \operatorname{Ann}_{\C[s]} (\D_X f^s / t \D_X f^s)_x$.

There is a $\O_X$-linear isomorphism between any left $\D_X$-module $\N$ and it's right version $ \N \otimes_{\O_X}\omega_X$.
Concretely, any section $u$ of $\N$ gives rise to the section $u^* := u dx$.
Further, for any operator $P$ of $\D_X$ there is a adjoint $P^*$ such that
$$(P\cdot u)^* =   u^* \cdot P^*$$
for any section $u$ of $\N$.
For a vector field $\xi := \sum_i\xi_i \partial_i$ comparison of the definitions shows that $\xi^* := \sum_i\partial_i\xi_i$ satisfies this equality and this extends to $\D_X$ by iterating.
By this procedure the functional equation $P f^{s+1} = b(s) f^s$ may equivalently be stated as the equation
$$f^{s+1}dx \cdot P^* = b(s) f^s dx $$
in $\D_X f^s \otimes_{\O_X}\omega_X$.
The corresponding module $\M$ on $Y$ will be the submodule of $\D_Y g^s \otimes_{\O_Y}\omega_Y$ spanned by $g^s \mu^*(dx)$.
Observe that $\M$ may also be equipped with a $\D_X\langle s,t\rangle$-module structure.
\begin{lemma}\label{lem: UpstairsB}
  The polynomial $b(s) = \prod_{i=1}^r \prod_{j=1}^{\mult_{E_i}(g)} (\mult_{E_i}(g)s + k_i + j)$ annihilates $\M / t\M$.
\end{lemma}
\begin{proof}
  This may be checked locally.
  If the chosen point is on none of divisors $E_i$ then $g$ is invertible so that $\M/t\M$ is trivial.
  Now suppose we are working near a point $y\in Y$ which is on $E_i$ if and only if $i\in I$ with $I$ non-empty.
  Then one can pick local coordinates $y_i$ such that
  $$ g = \prod_{i\in I} y_i^{m_i}; \qquad \mu^*(dx) =u \prod_{i\in I} y_i^{k_i}$$
  where $u$ is a local unit.
  Now set $P = u^{-1}(\prod_{i\in I}\partial_i^{m_i})u$ to get
  $$g^{s+1} \mu^*(dx)\cdot P = q(s) g^s \mu^*(dx)$$
  where $q(s) = \prod_{i\in I}\prod_{j=1}^{\mult_{E_i}(g)} (\mult_{E_i}s +  m_i + j ). $
\end{proof}
Observe that $s,t$ can be viewed as $\D_X$-linear injective endomorphisms on $\M$.
The associated the long exact sequence yields a $\D_X\langle s,t\rangle$-module structure on the direct image $\Int\M$ where the functorial nature of the direct image is used to ensure that $ts -st = 1$.
Similarly, the polynomial $ b(s)$ provided by \cref{lem: UpstairsB} annihilates $\Int\M/t\Int\M$.

Consider the surjection $\D_Y\to \M$ induced by $1 \mapsto g^s \mu^*(dx)$.
The associated long exact sequence includes a morphism $\int^0\D_Y \to \Int\M$.
Observe that $\int^0\D_Y = R^0 \mu_*(\D_{Y\to X})$ contains a global section corresponding to the section $1$ of $\D_{Y\to X}$.
Let $u$ be the image of this section in $\Int\M$ and denote $\U$ for the right $\D_X\langle s,t\rangle$-module generated by $u$.
\begin{lemma}\label{lem: SurjectiveUf}
  There is a surjective morphism of right $\D_X\langle s,t\rangle$-modules $\U \to \D_X f^s \otimes_{\O_X} \omega_X$ sending $u$ to $f^s dx$.
\end{lemma}
\begin{proof}
  The resolution of singularities $Y\to X$ is a isomorphism on the complement of $\{f = 0\}$. Hence, a isomorphism $\U \cong \Int \M \cong  \D_X f^s  \otimes_{\O_X}\omega_X$ holds outside of $\{f = 0\}$.

  Pick some open set $V\subseteq X$. To show this yields a well-defined morphism of $\D_X$-modules it must be show that $(f^s dx)P = 0$ whenever $uP = 0$ in $\U(V)$.
  Due to the isomorphism it is certainly the case that $(f^s dx) P = 0$ outside of $\{f = 0\}$.
  Hence, the support of the coherent sheaf of $\O_V$-modules $\O_V (f^s dx) P $ lies in $\{f = 0\}$.
  The Nullstellen Satz now yields that $f^N (f^s dx) P  = 0$ for some sufficiently large $N\geq 0$.
  Note that $f$ is a non-zero divisor of $(\D_Xf^s\otimes_{\O_X} \omega_X)(V)$.
  Hence, it follows that $(F^s dx) P= 0$ on $V$ as desired.

  Finally, observe that $tu = fu$ so that this morphism of $\D_X$-modules also commutes with the actions by $t,s$.
\end{proof}
Due to \cref{lem: UpstairsB} there is a suitable $b$-polynomial for $\Int \M$.
By \cref{lem: lem: SurjectionUf} it remains to compare $\Int\M$ and $\U$.
\begin{lemma}
  The quotient $\Int\M / \U$ is a holonomic $\D_X$-module.
\end{lemma}
\begin{proof}
  By \cref{prop: SESBehaviourChar} the characteristic variety of $\M$ is a subset of the characteristic variety of $\D_Yg^s \otimes_{\O_Y} \omega_Y$.
  This has the same characteristic variety as $\D_Y g^s$ due to the behaviour of the $\O_Y$-linear isomorphism between $\D_Yg^s$ and $\D_Yg^s \otimes_{\O_Y} \omega_Y$.
  By \cref{prop: IsotropicAndDominate} it follows that $\Ch \M \subseteq W \cup \Lambda$ for some isotropic $\Lambda\subseteq T^*Y$ and a irreducible $(n+1)$-dimensional variety $W$ which dominates $Y$.

  Observe that $\M$ is certainly $\mu$-good since it admits a global good filtration $F_i\M := F_i\D_X \cdot g^s \mu^*(dx)$.
  Hence, \cref{prop: EstimateProper} is applicable and yields that
  $$\Ch \Int \M \subseteq \widetilde{\mu}\left((T^*\mu)^{-1} (\Lambda \cup W)\right).$$
  By \cref{lem: IsotropicDirectImage} the set $\widetilde{\mu}((T^*\mu)^{-1}(\Lambda))$ is still isotropic and will not form any obstruction to $\Int\M/\U$ being holonomic.
  Further, observe that $\widetilde{\mu}((T^*\mu)^{-1}(W))$ remains a irreducible $(n+1)$-dimensional variety which dominates $X$.
  On the other hand $\mu$ is a isomorphism outside of $\{f = 0\}$ so $\Int\M/\U$ is only supported on $\{f = 0\}$.
  Intersecting $\widetilde{\mu}((T^*\mu)^{-1}(W))$ with $\{f = 0\}$ yields a $n$-dimensional variety whence the desired result follows.
\end{proof}
\begin{proposition}\label{prop: StableZero}
  For sufficiently large $N$ it holds that $t^N (\Int \M)_x / \U_x = 0$.
\end{proposition}
\begin{proof}
  The sequence $t^n \Int\M / \U$ forms a decreasing sequence of holonomic $\D_X$-modules.
  By \cref{prop: FiniteLength} the induced sequence of $\D_{X,x}$ modules in the stalk at $x$ must stabilise.
  Let $N$ be sufficiently large such that $t^N(\Int\M)_x/\U_x$ attains the stable value.

  Applying \cref{prop: HomAlgebraic} to the $(\D_X)_x$-linear automorphism $s$ produces a non-zero polynomial $q(s)\in \C[s]$ which annihilates $t^N(\Int\M)_x/\U_x$.
  Let $q(s)$ be of minimal degree with this property.
  Observe that $q(s+1)t = tq(s)$ so, using the stabilisation, it follows that
  $$q(s+1)t^N\left(\Int\M\right)_x/\U_x = t q(s) t^N\left(\Int\M\right)_x/\U_x = 0.$$
  This means that $q(s) - q(s+1)$ also annihilates $t^N(\Int\M)_x/\U_x$.
  By the minimality of the degree of $q(s)$ it follows that $q(s) - q(s+1)=0$ which is to say that $q(s)$ is a non-zero constant.
  This means that $t^N(\Int\M)_x/\U_x = 0 $ as desired.
\end{proof}
Putting all these facts together yields the proof of \cref{thm: LichtinEstimate}.
\begin{proof}
  Let $N$ be as in \cref{prop: StableZero} and denote $b(s)$ for the polynomial provided by \cref{lem: UpstairsB}.
  Set $B(s) = b(s+N+1)b(s+N)\cdots b(s)$ and observe that $B(s)\M_x \subseteq t^{N+1}\M_x \subseteq t\U_x$.
  In particular this means that $B(s) \in \operatorname{Ann}_{\C[s]} \U_x / t\U_x$.

  The $\D_X\langle s,t\rangle$-linear surjection $\U \to \D_X f^s \otimes_{\O_X} \omega_X$ from \cref{lem: SurjectiveUf} now implies that $b_{f,x}(s)$ divides $B(s)$.
  This yields the desired estimate for $Z(b_{f,x})$.
\end{proof}
