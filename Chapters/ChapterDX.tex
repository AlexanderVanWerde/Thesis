\chapter{$\D_X$-modules and the Riemann-Hilbert Correspondence}\label{Ch: ChapterDX}
The subject of this chapter are modules over rings of differential operators.
Throughout $X$ can be a smooth algebraic variety or a complex manifold.
The ring of differential operators $\D_X$ will be defined formally in the next section.
For the purpose of this section it's sufficient to note that local sections of $\D_X$ are of the form $\sum c_{\alpha } \partial^\alpha$ with $c_{\alpha}$ local sections of $\O_X$ and $\partial^\alpha = \partial_1^{\alpha_1}\cdots \partial_n^{\alpha_n}$.

A $\D_X$-modules gives a canonical description of systems of differential equations.
Consider a system of differential equations
$$\sum_{j=1}^k P_{ij}(x,\partial) f_j = 0; \qquad i= 1,\ldots,m$$
with unknown functions $f_j$ of $\O_X$ and differential operators $P_{ij}$.
The functions $f_j$ are somewhat arbitrary in the description of this system.
For instance, take $g_j=\lambda_j f_j$ for certain non-zero functions $\lambda_j$.
There is then a associated system of equations for $g_j$.
A solution of the $g_j$-system corresponds uniquely to a solution of the $f_j$-system.

Consider the cokernel $\M$ of the map
$$P:\D_X^k \to \D_X^m : \left(Q_1,\ldots, Q_k\right) \mapsto \left(\ldots, \sum_{j=1}^k Q_j P_{ij},\ldots\right).$$
This map is left $\D_X$-linear so $\M$ is a left $\D_X$-module.
Note that it is necessary to distinguish between left and right modules because differential operators form a non-commutative ring.
Direct verification shows that the solutions of the system of differential equations are encoded in $\Hom_{\D_X}(\M,\O_X)$.
More generally, for any $\D_X$-module $\N$ the solutions in $\N$ are encoded by $\Hom_{\D_X}(\M,\N)$.
This shows that $\D_X$-modules provide a canonical description of differential equations.
\section{$\D_X$-modules}
From now on let $X$ be a smooth algebraic variety over $\C$ and denote $n= \dim X$.
The properties discussed in this section are common knowledge in the field of $\D_X$-modules.
For detailed references see \cite{bjork1979rings}, \cite{kashiwara2003d} or \cite{hotta2007d}.

\begin{definition}
  The sheaf of differential operators $\D_X$ is the subsheaf of rings in $\mathcal{E}nd_{\O_X}(\O_X,\O_X)$ generated by $\O_X$ and the vector fields $\Theta_X$.
\end{definition}
\subsection{Filtrations}
Observe that $\D_X$ is a sheaf of non-commutative rings.
Given local coordinates $x_1,\ldots, x_n$ on $X$ it holds that
$$\partial_i x_j - x_j\partial_i = \delta_{ij} $$
where $\delta$ denotes the Kronecker delta.

This non-commutativity exits the typical domain of algebraic geometry.
This can be resolved by consideration of a graded structure.
The essential observation here is that differential operators commute up to a element of lower order.
\begin{definition}
  The order filtration on $\D_X$ is defined inductively to be given by the sheaves of $\O_X$-submodules $F_i \D_X$ such that $F_0\D_X = \O_X$ and $[F_i\D_X, F_i \D_X] \subseteq F_{i-1}\D_X$.
\end{definition}
The term $F_i \D_X$ in the order filtration can be described as containing all differential operators of order less than or equal to $i$.
Indeed, given local coordinates $x_1,\ldots, x_n$ one can show that $F_i\D_X$ is the $\O_X$-module locally generated by $\partial^\alpha = \partial_1^{\alpha_1}\cdots \partial_n^{\alpha_n}$ where $\alpha$ is a multi-index with $\abs{\alpha}\leq n$.
The following observations are immediate.
\begin{lemma}
  The $F_i\D_X$ are coherent $\O_X$-modules and form a exhaustive filtration. This is to say that $\cup_{i\geq 0}F_i\D_X = \D_X$
  and that for any $i,j\geq 0$ it holds that $F_i\D_X \cdot F_j \D_X \subseteq F_{i+j}\D_X$.
\end{lemma}
There is a similar notion of filtrations on $\D_X$-modules $\M$.
Without any harm let's assume that $\M$ is a left $\D_X$-module, the case for right modules is analogous.
A filtration consists of $\O_X$-submodules $F_i\M$ of $\M$ such that $\cup_i F_i \M = \M$ and $F_i\D_X \cdot F_j\M \subseteq F_{i+j}\M$.

Stepping over to the graded object has the advantage that $\gr \D_X$ is commutative by definition of the order filtration whence the classical methods of algebraic geometry are applicable.
The symplectic structure of $T^* X$ captures part of the non-commutativity.
Indeed, given two differential operators $P, Q$.
Pick local coordinates $x_1,\ldots, x_n$ and decompose $P = \sum_{\alpha} p_\alpha \partial^\alpha $ and $Q = \sum_{\beta} q_\beta \partial^\beta$.
Let $m_1,m_2$ be the maximal values of $\abs{\alpha}$ and $\abs{\beta}$ with non-zero coefficients.
Then the induced elements of $P$ and $Q$ in $\gr \D_X$ are of the form $p = \sum_{\abs{\alpha} = m_1} p_\alpha \xi^\alpha$ and $q = \sum_{\abs{\beta} = m_2}q_\beta \xi^\beta $ where $\xi_i$ is the induced element of $\partial_i$.
On the other hand, the induced element of $PQ - QP$ is $\sum_{i=1}^n\frac{\partial p}{\partial \xi_i}\frac{\partial q}{\partial x_i} - \frac{\partial q}{\partial \xi_i}\frac{\partial p}{\partial x_i}$.
This is precisely $\{p,q\}$ where $\{\blank,\blank\}$ is the Poisson bracket.

One can view $\gr\D_X$ as a subsheaf of $\O_{T^*X}$.
Denote $\pi$ for the projection of $T^*X \to X$.
Then any $\gr\D_X$-module $\mathcal{M}$ has a corresponding module on $T^* X$ defined by $\O_{T^*X} \otimes_{\pi^{-1}\gr \D_X} \mathcal{M}$.
By abuse of notation this module is still denoted $\mathcal{M}$ and it will always be implicitly assumed that $\gr \D_X$-modules live on $T^*X$ unless it is explicitly mentioned otherwise.
In particular, for a filtration of the $\D_X$-module $\M$ the graded object $\gr \M = \oplus_i F_i \M / F_{i-1}\M$ is a $\gr \D_X$-module.

\begin{proposition}
  A $\D_X$-module $\M$ is coherent if and only if it admits a filtration such that $\gr \M$ is a coherent $\gr \D_X$-module. Such a filtration is called a good filtration.
\end{proposition}
\begin{proof}
  A proof of this result may be found in \cite[Chapter 2]{hotta2007d}.
\end{proof}
\begin{proposition}
  Let $\M$ be a coherent $\D_X$-module, then the support of $\grrel\M$ in $T^* X$ is a independent of the chosen good filtration. It is called the characteristic variety of $\M$ and denoted $\Ch \M$.
\end{proposition}
\begin{proof}
  A proof of this result may be found in \cite[Chapter 2]{hotta2007d}.
\end{proof}
\begin{proposition}\label{prop: StructureChVar}
  Let $\M$ be a coherent $\D_X$-module, then $\Ch \M$ is a homogeneous and isotropic closed subset of $T^* X$.
\end{proposition}
\begin{proof}
  These results may be found in \cite[Chapter 2]{kashiwara2003d}.
\end{proof}
\begin{proposition}
  Consider a short exact sequence of coherent $\D_X$-modules
  $$0 \to \M_1 \to \M_2 \to \M_3 \to 0 $$
  then it holds that
  $$\Ch \M_2 = \Ch \M_1 \cup \Ch \M_3. $$
\end{proposition}
\begin{proof}
  A good filtration on $\M_2$ induces good filtrations on $\M_1$ and $\M_3$ and one has a short exact sequence
  $$0\to \gr \M_1 \to \gr \M_2 \to \gr \M_3 \to 0 $$
  whence the result follows.
\end{proof}
\subsection{Holonomicity}
A particularly nice class of $\D_X$-modules are given by maximally overdetermined systems of differential equations.
This is to say that there are many relations for $\M$ or equivalently that $\Ch\M$ is small.
Observe that by isotropic part of \cref{prop: StructureChVar} it follows that $\dim \Ch\M \geq n$.
\begin{definition}
  A coherent $\D_X$-module $\M$ is called holonomic if
  $\dim \Ch \M  = n.$
\end{definition}
For technical purposes it is mostly important that holonomic modules have finiteness properties.
\begin{proposition}
  Let $\M$ be a holonomic $\D_X$-module. Then, for any $x\in X$, the stalk $\M_x$ is a $(\D_X)_x$-module of finite length.
\end{proposition}
\begin{proof}
  This result may be found in \cite[Chapter 4]{kashiwara2003d}.
\end{proof}
Recall from the introduction that $\Hom_{\D_X}(\M,\N)$ encodes the solutions in $\N$ of a system of differential equations.
\begin{proposition}
  Let $\M,\N$ be holonomic $\D_X$-modules. Then, for any $x\in X$, the stalk $\Hom_{\D_X}(\N,\M)_x$ is a finite-dimensional vector space over $\C$.
\end{proposition}
\begin{proof}
  This result may be found in \cite[Chapter 4]{kashiwara2003d}.
\end{proof}
\begin{corollary}
  Let $\M$ be a holonomic $\D_X$-module. Then $\Hom_{\D_X}(\M,\M)$ is $\C$-algebraic. This is to say that for any $\varphi \in\Hom_{\D_X}(\M,\M) $ there exists some polynomial $b$ with coefficients in $\C$ such that $b(\varphi)=0$.
\end{corollary}
\section{Regular singularities}
This section is based on \cite[Chapter 5]{kashiwara2003d}.
Let $X=\C$ considered with it's analytical topology and consider a ordinary differential operator
$P(x,\partial) = \sum_{k=0}^m a_k(x)\partial^k.$
Suppose that $a_m(x)\neq 0$ for any $x\neq 0$.
Then $\M :=\D_X/\D_X P$ is locally of the form $\O_X^m$ as a $\D_X$-module near any point $x\neq 0$.
In particular the solutions $\Hom_{\D_X}(\M,\O_X)$ form a locally constant sheaf of rank $m$ outside of $0$.
The solutions near zero may be more subtle due to monodromy.

Observe that $\Ch \M\subseteq \{(x,\xi): x\xi = 0\}$.
Hence, for any filtration on $\M$ there exists some $N>0$ such that
$(x\xi)^N \gr\M = 0 $.
\begin{proposition}
  The following conditions are equivalent.
  \begin{enumerate}
    \item There exists a filtration on $\M$ such that $x\xi\gr\M = 0$.
    \item The equation $P(x,\partial)u$ has $m$ linearly independent solutions of the form $x^\lambda \sum_{j=0}^s u_j \log(x)^j $
    near $0$ for some $s\geq 0$, $\lambda \in \C$ and holomorphic $u_j$ if and only $P$ has a regular singularity in $0$.
  \end{enumerate}
\end{proposition}
If these two equivalent conditions are satisfied one calls $0$ a regular singularity of $\M$.
This has the following generalisation to higher dimensions.
\begin{definition}
   Let $\M$ be a holonomic $\D_X$-module on a complex manifold $X$ with characteristic variety determined by some ideal sheaf $\mathcal{I}$. Then $\M$ is called regular holonomic if it it admits a filtration such that $\mathcal{I} \gr(\M) = 0$.
\end{definition}
It appears that these definitions should generalise directly to the algebraic situation.
However, this has unintended consequences for the Riemann-Hilbert correspondence which states that a system of differential equations should be equivalent to the system of solutions.
Concretely, the systems of differential equations are encoded in regular holonomic $\D_X$-modules $\M$.

For a example, let $X = \C$ as before and consider the regular holonomic $\D_X$-modules $\O_X$ and $\M := \D_X/\D_X(\partial - 1)$.
These are analytically isomorphic by the map which sends $f$ to $f\exp(x)$.
In particular the Riemann-Hilbert correspondence shows that they have isomorphic systems of solutions.
However, $\O_X$ and $\M$ are not algebraically isomorphic.
This seems to suggest that the equivalence between differential equations and their systems of solutions would not hold in the algebraic case.
The problem is that $\M$ is not regular at infinity.

 The adjusted definition proceeds in two steps which we sketch.
 The precise details may be found in \cite[Chapter 7]{borel1987algebraic}.
 Firstly, one induces a module corresponding to $\M$ on the smooth completion $\overline{X}$ of the smooth algebraic variety $X$.
 Hereafter, the old definition may be applied on the analytification of $\overline{X}$ by use of the GAGA principle.
 The smooth completion ensures that the regularity also holds at infinity.

\section{Perverse Sheaves}
The Riemann-Hilbert correspondence states in great generality that there is a equivalence between a system of differential equations and the system of solutions.
Philosophically, this is a significant result because it yields a connection between the algebraic/analytic world of differential equations and the topological world associated to their solutions.
\section{Riemann-Hilbert Correspondence}

\section{Monodromy}
