\chapter{$\D_X$-modules and the Riemann-Hilbert Correspondence}\label{Ch: Chapter1}
The subject of this chapter are modules over rings of differential operators.
Throughout $X$ can be a smooth algebraic variety or a complex manifold.
The ring of differential operators $\D_X$ will be defined formally in the next section.
For the purpose of this section it's sufficient to note that local sections of $\D_X$ are of the form $\sum c_{\alpha } \partial^\alpha$ with $c_{\alpha}$ local sections of $\O_X$ and $\partial^\alpha = \partial_1^{\alpha_1}\cdots \partial_n^{\alpha_n}$.

A $\D_X$-modules gives a canonical description of systems of differential equations.
Consider a system of differential equations
$$\sum_{j=1}^k P_{ij}(x,\partial) f_j = 0; \qquad i= 1,\ldots,m$$
with unknown functions $f_j$ of $\O_X$ and differential operators $P_{ij}$.
The functions $f_j$ are somewhat arbitrary in the description of this system.
For instance, take $g_j=\lambda_j f_j$ for certain non-zero functions $\lambda_j$.
There is then a associated system of equations for $g_j$.
A solution of the $g_j$-system corresponds uniquely to a solution of the $f_j$-system.

Consider the cokernel $\M$ of the map
$$P:\D_X^k \to \D_X^m : \left(Q_1,\ldots, Q_k\right) \mapsto \left(\ldots, \sum_{j=1}^k Q_j P_{ij},\ldots\right).$$
This map is left $\D_X$-linear so $\M$ is a left $\D_X$-module.
Note that it is necessary to distinguish between left and right modules because differential operators form a non-commutative ring.
Direct verification shows that the solutions of the system of differential equations are encoded in $\Hom_{\D_X}(\M,\O_X)$.
More generally, for any $\D_X$-module $\N$ the solutions in $\N$ are encoded by $\Hom_{\D_X}(\M,\N)$.
This shows that $\D_X$-modules provide a canonical description of differential equations.
\section{$\D_X$-modules}
From now on let $X$ be a smooth algebraic variety over $\C$ and denote $n= \dim X$.
The properties discussed in this section are common knowledge in the field of $\D_X$-modules.
For detailed references see \cite{bjork1979rings}, \cite{kashiwara2003d} or \cite{hotta2007d}.

\begin{definition}
  The sheaf of differential operators $\D_X$ is the subsheaf of rings in $\mathcal{E}nd_{\O_X}(\O_X,\O_X)$ generated by $\O_X$ and the vector fields $\Theta_X$.
\end{definition}
\subsection{Filtrations}
Observe that $\D_X$ is a sheaf of non-commutative rings.
Given local coordinates $x_1,\ldots, x_n$ on $X$ it holds that
$$\partial_i x_j - x_j\partial_i = \delta_{ij} $$
where $\delta$ denotes the Kronecker delta.

This non-commutativity exits the typical domain of algebraic geometry.
This can be resolved by consideration of a graded structure.
The essential observation here is that differential operators commute up to a element of lower order.
\begin{definition}
  The order filtration on $\D_X$ is defined inductively to be given by the sheaves of $\O_X$-submodules $F_i \D_X$ such that $F_0\D_X = \O_X$ and $[F_i\D_X, F_i \D_X] \subseteq F_{i-1}\D_X$.
\end{definition}
The term $F_i \D_X$ in the order filtration can be described as containing all differential operators of order less than or equal to $i$.
Indeed, given local coordinates $x_1,\ldots, x_n$ one can show that $F_i\D_X$ is the $\O_X$-module locally generated by $\partial^\alpha = \partial_1^{\alpha_1}\cdots \partial_n^{\alpha_n}$ where $\alpha$ is a multi-index with $\abs{\alpha}\leq n$.
The following observations are immediate.
\begin{lemma}
  The $F_i\D_X$ are coherent $\O_X$-modules and form a exhaustive filtration. This is to say that $\cup_{i\geq 0}F_i\D_X = \D_X$
  and that for any $i,j\geq 0$ it holds that $F_i\D_X \cdot F_j \D_X \subseteq F_{i+j}\D_X$.
\end{lemma}
There is a similar notion of filtrations on $\D_X$-modules $\M$.
Without any harm let's assume that $\M$ is a left $\D_X$-module, the case for right modules is analogous.
A filtration consists of $\O_X$-submodules $F_i\M$ of $\M$ such that $\cup_i F_i \M = \M$ and $F_i\D_X \cdot F_j\M \subseteq F_{i+j}\M$.

Stepping over to the graded object has the advantage that $\gr \D_X$ is commutative by definition of the order filtration whence the classical methods of algebraic geometry are applicable.
The symplectic structure of $T^* X$ captures part of the non-commutativity.
Indeed, given two differential operators $P, Q$.
Pick local coordinates $x_1,\ldots, x_n$ and decompose $P = \sum_{\alpha} p_\alpha \partial^\alpha $ and $Q = \sum_{\beta} q_\beta \partial^\beta$.
Let $m_1,m_2$ be the maximal values of $\abs{\alpha}$ and $\abs{\beta}$ with non-zero coefficients.
Then the induced elements of $P$ and $Q$ in $\gr \D_X$ are of the form $p = \sum_{\abs{\alpha} = m_1} p_\alpha \xi^\alpha$ and $q = \sum_{\abs{\beta} = m_2}q_\beta \xi^\beta $ where $\xi_i$ is the induced element of $\partial_i$.
On the other hand, the induced element of $PQ - QP$ is $\sum_{i=1}^n\frac{\partial p}{\partial \xi_i}\frac{\partial q}{\partial x_i} - \frac{\partial q}{\partial \xi_i}\frac{\partial p}{\partial x_i}$.
This is precisely $\{p,q\}$ where $\{\blank,\blank\}$ is the Poisson bracket.

One can view $\gr\D_X$ as a subsheaf of $\O_{T^*X}$.
Denote $\pi$ for the projection of $T^*X \to X$.
Then any $\gr\D_X$-module $\mathcal{M}$ has a corresponding module on $T^* X$ defined by $\O_{T^*X} \otimes_{\pi^{-1}\gr \D_X} \mathcal{M}$.
By abuse of notation this module is still denoted $\mathcal{M}$ and it will always be implicitly assumed that $\gr \D_X$-modules live on $T^*X$ unless it is explicitly mentioned otherwise.
In particular, for a filtration of the $\D_X$-module $\M$ the graded object $\gr \M = \oplus_i F_i \M / F_{i-1}\M$ is a $\gr \D_X$-module.

\begin{proposition}
  A $\D_X$-module $\M$ is coherent if and only if it admits a filtration such that $\gr \M$ is a coherent $\gr \D_X$-module. Such a filtration is called a good filtration.
\end{proposition}
\begin{proof}
  A proof of this result may be found in \cite[Chapter 2]{hotta2007d}.
\end{proof}
\begin{proposition}
  Let $\M$ be a coherent $\D_X$-module, then the support of $\grrel\M$ in $T^* X$ is a independent of the chosen good filtration. It is called the characteristic variety of $\M$ and denoted $\Ch \M$.
\end{proposition}
\begin{proof}
  A proof of this result may be found in \cite[Chapter 2]{hotta2007d}.
\end{proof}
\begin{proposition}\label{prop: StructureChVar}
  Let $\M$ be a coherent $\D_X$-module, then $\Ch \M$ is a homogeneous and isotropic closed subset of $T^* X$.
\end{proposition}
\begin{proof}
  These results may be found in \cite[Chapter 2]{kashiwara2003d}.
\end{proof}
\begin{proposition}
  Consider a short exact sequence of coherent $\D_X$-modules
  $$0 \to \M_1 \to \M_2 \to \M_3 \to 0 $$
  then it holds that
  $$\Ch \M_2 = \Ch \M_1 \cup \Ch \M_3. $$
\end{proposition}
\begin{proof}
  A good filtration on $\M_2$ induces good filtrations on $\M_1$ and $\M_3$ and one has a short exact sequence
  $$0\to \gr \M_1 \to \gr \M_2 \to \gr \M_3 \to 0 $$
  whence the result follows.
\end{proof}
\subsection{Holonomicity}
A particularly nice class of $\D_X$-modules are given by maximally overdetermined systems of differential equations.
This is to say that there are many relations for $\M$ or equivalently that $\Ch\M$ is small.
Observe that by isotropic part of \cref{prop: StructureChVar} it follows that $\dim \Ch\M \geq n$.
\begin{definition}
  A coherent $\D_X$-module $\M$ is called holonomic if
  $\dim \Ch \M  = n.$
\end{definition}
For technical purposes it is mostly important that holonomic modules have finiteness properties.
\begin{proposition}
  Let $\M$ be a holonomic $\D_X$-module. Then, for any $x\in X$, the stalk $\M_x$ is a $(\D_X)_x$-module of finite length.
\end{proposition}
\begin{proof}
  This result may be found in \cite[Chapter 4]{kashiwara2003d}.
\end{proof}
Recall from the introduction that $\Hom_{\D_X}(\M,\N)$ encodes the solutions in $\N$ of a system of differential equations.
\begin{proposition}
  Let $\M,\N$ be holonomic $\D_X$-modules. Then, for any $x\in X$, the stalk $\Hom_{\D_X}(\N,\M)_x$ is a finite-dimensional vector space over $\C$.
\end{proposition}
\begin{proof}
  This result may be found in \cite[Chapter 4]{kashiwara2003d}.
\end{proof}
\begin{corollary}
  Let $\M$ be a holonomic $\D_X$-module. Then $\Hom_{\D_X}(\M,\M)$ is $\C$-algebraic. This is to say that for any $\varphi \in\Hom_{\D_X}(\M,\M) $ there exists some polynomial $b$ with coefficients in $\C$ such that $b(\varphi)=0$.
\end{corollary}
\subsection{Regular singularities}
This subsection is based on \cite[Chapter 5]{kashiwara2003d}.
Let $X=\C$ considered with it's analytical topology and consider a ordinary differential operator
$P(x,\partial) = \sum_{k=0}^m a_k(x)\partial^k.$
Suppose that $a_m(x)\neq 0$ for any $x\neq 0$.
Then $\M :=\D_X/\D_X P$ is locally of the form $\O_X^m$ as a $\D_X$-module near any point $x\neq 0$.
In particular the solutions $\Hom_{\D_X}(\M,\O_X)$ form a locally constant sheaf of rank $m$ outside of $0$.
The solutions near zero may be more subtle due to monodromy.

Observe that $\Ch \M\subseteq \{(x,\xi): x\xi = 0\}$.
Hence, for any filtration on $\M$ there exists some $N>0$ such that
$(x\xi)^N \gr\M = 0 $.
\begin{proposition}
  The following conditions are equivalent.
  \begin{enumerate}
    \item There exists a filtration on $\M$ such that $x\xi\gr\M = 0$.
    \item The equation $P(x,\partial)u$ has $m$ linearly independent solutions of the form $x^\lambda \sum_{j=0}^s u_j \log(x)^j $
    near $0$ for some $s\geq 0$, $\lambda \in \C$ and holomorphic $u_j$ if and only $P$ has a regular singularity in $0$.
  \end{enumerate}
\end{proposition}
If these two equivalent conditions are satisfied one calls $0$ a regular singularity of $\M$.
This has the following generalisation to higher dimensions.
\begin{definition}
   Let $\M$ be a holonomic $\D_X$-module on a complex manifold $X$ with characteristic variety determined by some ideal sheaf $\mathcal{I}$. Then $\M$ is called regular holonomic if it it admits a filtration such that $\mathcal{I} \gr(\M) = 0$.
\end{definition}
It appears that these definitions should generalise directly to the algebraic situation.
However, this has unintended consequences for the Riemann-Hilbert correspondence which states that a system of differential equations should be equivalent to the system of solutions.
Concretely, the systems of differential equations are encoded in regular holonomic $\D_X$-modules $\M$.

For a example, let $X = \C$ as before and consider the regular holonomic $\D_X$-modules $\O_X$ and $\M := \D_X/\D_X(\partial - 1)$.
These are analytically isomorphic by the map which sends $f$ to $f\exp(x)$.
In particular the Riemann-Hilbert correspondence shows that they have isomorphic systems of solutions.
However, $\O_X$ and $\M$ are not algebraically isomorphic.
This seems to suggest that the equivalence between differential equations and their systems of solutions would not hold in the algebraic case.
The problem is that $\M$ is not regular at infinity.

 The adjusted definition proceeds in two steps which we sketch.
 The precise details may be found in \cite[Chapter 7]{borel1987algebraic}.
 Firstly, one induces a module corresponding to $\M$ on the smooth completion $\overline{X}$ of the smooth algebraic variety $X$.
 Hereafter, the old definition may be applied on the analytification of $\overline{X}$ by use of the GAGA principle.
 The smooth completion ensures that the regularity also holds at infinity.
\section{Category Theory}
The Riemann-Hilbert correspondence in it's full generality also gives information on the cohomology of $\D_X$-modules.
A uniform framework for these concepts may be provided by derived category theory which can be defined for general abelian categories.
\todo{Also mention further}

The contents of this section are common knowledge.
More details may be found in \cite[Chapters 1 and 5]{dimca2004sheaves}.
\todo{Add more?}

\subsection{Derived Categories}
Fix a abelian category $\mathcal{A}$ and denote $C(\AA)$ for the category of chain complexes in $\AA$.
The category $C(\AA)$ contains full subcategories $C^*(\AA)$ with $*= +, -, b$ denoting that the complexes in $\AA$ are bounded below, above or bounded on both sides respectively.
For example $C^+(\AA)$ may contain complexes of the form $\cdots \to 0\to \cdots X^{-1} \to X^0 \to \cdots$.
For a complex $X^\bullet$ and $k\in \Z$ one has a shifted complex $X^\bullet[k]$ with $(X^\bullet[k])^s = X^{k+s}$.
Further, denote $\operatorname{Hom}^k(X^\bullet, Y^\bullet) :=  \op{Hom}(X^\bullet, Y^\bullet[k])$ which are the chain maps that change the grading by $k$.
\begin{definition}
  Two complex morphisms $u,v:X^\bullet\to Y^\bullet$ are called homotopic if there exists $h\in \operatorname{Hom}^{-1}(X^\bullet, Y^\bullet)$ such that $u-v = d_Y h + hd_X$.
\end{definition}
\begin{definition}
  A morphism $u:X^\bullet\to Y^\bullet$  of complexes in $C^*(\AA)$ is called a quasi-isomorphism if the induced morphism in cohomology $H^k(u):H^k(X^\bullet) \to H^k(Y^\bullet)$ is a isomorphism for all $k$. This may be denoted $u\sim v$.
\end{definition}
The derived category $D^*(\AA)$ is defined as the category obtained from $C^*(\AA)$ by localising with respect to the multiplicative system formed by the quasi-isomorphisms.
This means that $D^*(\AA)$ has the same objects as $C^*(\AA)$ but the quasi-isomorphisms of $C^*(\AA)$ have been turned into isomorphisms.
This definition can be made more concrete provided the category has enough injectives.
\begin{definition}
  A abelian category $\AA$ has enough injectives if for any object $X$ in $\AA$ there is an exact sequence $0\to X \to I$ in $\AA$ with $I$ injective.
\end{definition}
\begin{definition}
  Let $\AA$ be a abelian category.
  The homotopical category of complexes $K^*(\AA)$ of $\AA$ has the same objects as $C^*(\AA)$ and as morphisms
  $$\operatorname{Hom}_{K^*(\AA)}(X^\bullet,Y^\bullet):= \operatorname{Hom}_{C^*(\AA)}(X^\bullet,Y^\bullet)/\sim.$$
\end{definition}
Observe that two homotopic maps induce the same morphism in cohomology.
It follows that there is a well-defined functor $p_\AA:K^*(\AA)\to D^*(\AA)$.
\begin{proposition}
  Let $\AA$ be a abelian category with enough injectives and denote $I(\AA)$ for the full subcategory from the injective objects.
  Then the natural functor
  $$K^+(I(\AA))\to D^+(\AA) $$
  is a equivalence of categories.
\end{proposition}

\subsection{Triangulated Categories}
The categories $K^*(\AA)$ and $D^*(\AA)$ may fail to be exact.
In particular, the notion of short exact sequences no longer makes sense.
Instead, $K^*(\AA)$ and $D^*(\AA)$ may be viewed as triangulated categories which is to say that they come equipped with a notion of exact triangles.
\begin{definition}
  Let $u:X^\bullet \to Y^\bullet$ be a morphism of complexes in $C^*(\AA)$.
  The mapping cone of $u$ is the complex in $C^*(\A)$ given by
  $$C_u^\bullet := Y^\bullet \oplus (X^\bullet[1]) $$
  with $d_u(y,x)= (dy + u(x) , -dx)$.
\end{definition}
The concept of a mapping cone originated in a construction from algebraic topology which explains the name.
From the perspective of $X$ the $j$-th cohomology of the mapping cone measures the $x\in X^j$ which have a primitive in $Y^{j-1}$ but not in $X^{j-1}$.
Observe that the mapping cone gives rise to a triangle
$$T_u:X^\bullet \xrightarrow{u} Y^\bullet \to C_u^\bullet \to X^\bullet[1]$$
which may be denoted more intuitively as
$$
  \begin{tikzcd}
    X^\bullet \arrow{rr}{u}& & Y^\bullet \arrow{dl}{}\\
    & C_u^\bullet\arrow{lu}{+1}
  \end{tikzcd}
$$
Such triangles behave like short exact sequences.
Firstly, one can show that the composition of any two consecutive morphisms in $T_u$ is homotopic to $0$.
This already means that $T_u$ behaves like a chain complex in $K^*(\A)$.
Further, they give rise to a long exact sequence in cohomology
$$\cdots \to H^k(X^\bullet) \xrightarrow{u} H^k(Y^\bullet) \to H^k(C_u^\bullet) \xrightarrow{\delta} H^{k+1}(X^\bullet) \to \cdots$$
where the connecting morphism $\delta$ comes from the map $C_u^\bullet \to X^\bullet[1]$.

Further investigation of the properties of $T_u$ gives rise to the concept of a triangulated category.
These definitions and properties are pleasant in their own right so we go into some detail.

The distinguish triangles $\mathcal{T}$ in $K^*$ are the family of triangles which are isomorphic to a triangle of the form $T_u$.
Observe that $K^*(\AA)$ has a shift functor $T$ given by $TX^\bullet = X^\bullet[1]$.
\begin{definition}
  An additive category $\mathcal{D}$ equipped with a self-equivalence $T$ and family of distinguished triangles $\mathcal{T}$ is called a triangulated category if the following axioms are satisfied.
  \begin{enumerate}
    \item[(Tr1)] Any triangle isomorphic to a distinguish triangle is distinguished. For any object $X$ the triangle $X\to X \to 0 \to TX$ is distinguish where the first morphism is the identity.
    Any morphism $u:X\to Y$ is part of some distinguished triangle $X\xrightarrow{u} Y \to Z \to TX$.
    \item[(Tr2)] A triangle $X\xrightarrow{u} Y \xrightarrow{v} Z \xrightarrow{w} TX$ is distinguished if and only if the triangle $Y \xrightarrow{v} Z \xrightarrow{w} TX \xrightarrow{-Tu} TY$ is distinguished.
    \item[(Tr3)] A commutative diagram of the following from whose rows are distinguished triangles gives rise to a morphism of triangles
    $$
      \begin{tikzcd}
        X\arrow{d}\rar& Y\arrow{d}\rar & Z\rar & TX\\
        A\rar& B\rar & C\rar & TA
      \end{tikzcd}
     $$
    \item[(Tr4)] For any triple of distinguished triangles
    $$
      \begin{tikzcd}[row sep=small]
        X \arrow{r}{u} & Y \arrow{r}{x}& A \arrow{r}{} & TX \\
        Y \arrow{r}{v} & Z \arrow{r}{}& B \arrow{r}{y} & TY\\
        X \arrow{r}{vu} & Z\arrow{r}{} & C\arrow{r}{} & TX
      \end{tikzcd}
    $$
    there is a distinguished triangle
    $$\begin{tikzcd}
      A \arrow{r}{a} & C \arrow{r}{b}& B \arrow{r}{(Tx)y} & TA \\
    \end{tikzcd} $$
    such that $(id_X, v,a)$ and $(u,id_Z,b)$ are morphisms of triangles.
    \end{enumerate}
\end{definition}
\begin{proposition}
  Let $\mathcal{A}$ be a abelian category. Then $K^*(\A)$ and $D^*(\A)$ are triangulated categories.
\end{proposition}
A triangle $X\to Y \to Z \to TX$ will also be denoted $X\to Y \to Z \xrightarrow{+1} X$ and $T^m X$ may be denoted with $X[m]$.
Now the data of the final axiom can be organised as follows.
Correspondingly, $(Tr4)$ is also referred to as the octahedral axiom.
\begin{equation*}
\xymatrix@=1.6em{
& C \ar@{->}[dr] \ar[ddl] & \\
A \ar@{->}[ur] \ar[d]& &
    B\ar[ll]|(0.25)\hole|(0.75)\hole
       \ar[ddl] \\
X \ar[rr]^{v \circ u} \ar[dr]_u & &
    Z \ar[u] \ar[uul]\\
& Y \ar[ur]_v \ar[uul]&
}
\end{equation*}
\begin{definition}
  Let $\mathcal{D}$ be a triangulated category. A subcategory $\mathcal{C}$ of $\mathcal{D}$ is said to be stable under extensions if any distinguished triangle in $\mathcal{D}$ with two vertices in $\mathcal{C}$ also has it's third vertex in $\mathcal{D}$.
\end{definition}
\begin{definition}
  Let $\mathcal{C}$ be a full additive subcategory of a triangulated category $\mathcal{D}$. Then $\mathcal{C}$ is a triangulated subcategory if $\mathcal{C}$ is stable under extensions and $T\mathcal{C}\subseteq \mathcal{C}$.
\end{definition}

\begin{definition}
  Let $\mathcal{D}$ be a triangulated category and $\AA$ a abelian category.
  An additive functor $F:\mathcal{D} \to \AA$ is a cohomological functor if for any distinguished triangle in $\mathcal{D}$
  $$X \to Y \to Z\xrightarrow{+1} X $$
  the induced sequence $F(X) \to F(Y) \to F(Z) $
  is a exact in $\AA$. If $F$ is a cohomological functor one sets $F^i = F\circ T^i$.

  The family of functors $F^i$ is conservative if for any distinguished triangle
  $$X \to Y \to Z \xrightarrow{+1} X$$
  the induced long sequence
  $$\cdots \to F^i(X) \to F^i(Y) \to F^i(Z) \to F^{i+1}(X) \to \cdots $$
  is exact.
\end{definition}
The key example for the above definition is given by the cohomological functor $H^0: K^*(\A) \to \A$ and the conservative system of functors $H^k$.
\begin{definition}
  Let $\mathcal{D}, \mathcal{D}'$ be triangulated categories.
  A functor $F:\mathcal{D} \to \mathcal{D}'$ is called a functor of triangulated categories if it is compatible with the shift functor and transforms distinguished triangles in $\mathcal{D}$ into distinguished triangles of $\mathcal{D}'$.
\end{definition}
\subsection{Derived Functors}
Given abelian categories $\AA, \mathcal{B}$ and a functor of triangulated categories $F:K^*(\AA)\to K^*(\mathcal{B})$ one may wonder if there is a natural lift to $D(\AA)$.
\begin{definition}
  Let $F$ be as above. The right derived functor of $F$ is a couple $(R^*F,\xi_F)$ where $F^*F:D^*(\AA) \to D^*(\mathcal{B})$ is a functor of triangulated categories and $\xi_F:p_B\circ F \to R^*F \circ p_\A^*$
\end{definition}
\subsection{$t$-structures}
A generalisation of positive and negatively supported complexes to is given triangulated categories by the concept of a $t$-structure.
\begin{definition}
  A $t$-structure on a triangulated category $\mathcal{D}$ consists of two strictly full subcategories $\mathcal{D}^{\leq 0}$ and $\mathcal{D}^{\geq 0}$ such that, setting $\mathcal{D}^{\leq n} := \mathcal{D}^{\leq 0}[-n]$ and $\mathcal{D}^{\geq n} := \mathcal{D}^{\geq 0} [-n]$, the following properties hold.
  \begin{enumerate}
    \item[(i)] It holds that $D^{\leq 0}$ is a subcategory of $D^{\leq 1}$ and $D^{\geq 1}$ is a subcategory of $D^{\geq 0}$.
    \item[(ii)] For any objects $X$ in $\mathcal{D}^{\leq 0}$ and $Y$ of $\mathcal{D}^{\geq 1}$ it holds that $\operatorname{Hom}(X,Y) = 0$.
    \item[(iii)] For any object $X$ of $\mathcal{D}$ there is a distinguished triangle
    $$A \to X \to B \xrightarrow{+1} A $$
    with $A$ in $\mathcal{D}^{\leq 0}$ and $B$ in $\mathcal{D}^{\geq 1}$.
  \end{enumerate}
\end{definition}
\begin{definition}
  Let $\mathcal{D}$ be a triangulated category with a $t$-structure. Then $\mathcal{D} = \mathcal{D}^{\leq 0} \cap \mathcal{D}^{\geq 0}$ is called the heart of the $t$-structure.
\end{definition}
In the motivating case of $K^*(\AA)$ and $D^*(\AA)$ the heart of the $t$-structure recovers the original abelian category $\AA$.
\begin{proposition}
  The heart $\mathcal{D}$ of a $t$-structure is an abelian category which is stable by extensions.
\end{proposition}
Observe that $D^*(\AA)$ comes equipped with a truncation functors
$\tau_{\leq m}:D^*(\AA)\to D^-(\AA)$ which sends a complex $X^\bullet$ to
$$\tau_{\leq m}X^\bullet : \cdots \to X^{m-1} \to \ker d \to 0 \to 0 \to \cdots$$
and similarly a truncation functor $\tau_{\geq m}$ is defined by
$$\tau_{\geq m}X^\bullet: \cdots \to 0 \to 0 \to \operatorname{coim} d \to A^{m+1}\to \cdots.$$
This generalises to $t$-structures.
\begin{proposition}
  Let $\mathcal{D}$ be a triangulated category with a $t$-structure.
  Then the inclusion of $\mathcal{D}^{\leq}$ in $\mathcal{D}$ has a right adjoint functor $\tau_{\leq n}$.
  Similarly, the inclusion of $\mathcal{D}^{\geq n}$ in $\mathcal{D}$ has a left adjoint $\tau_{\geq n}$.
\end{proposition}
Observe that in the example of $D^*(\AA)$ one has that $\tau_{\geq 0} \tau_{\leq 0} X^\bullet$ is the complex with a single entry $H^0(X^\bullet)$.
This generalises to $t$-structures by viewing $\tH^0:= \tau_{\geq 0}\tau_{\leq 0}$ as a functor from $\mathcal{D}$ to it's heart $\mathcal{C}$.
Further let $\tH^i:= H^0 \circ T^i$.
\begin{definition}
  A $t$-structure is said to be non-degenerated if $\cap \mathcal{D}^{\leq n} = \cap \mathcal{D}^{\geq n}= \operatorname{Null}$ where $\operatorname{Null}$ denotes the family of objects which are isomorphic to the zero object in $\mathcal{D}$.
\end{definition}
\begin{proposition}
  Let $\mathcal{D}$ be a triangulated category with a $t$-structure. Then $\tH^0:\mathcal{D}\to \mathcal{C}$ is a cohomological functor.
\end{proposition}
\begin{proposition}
  Let $\mathcal{D}$ be a triangulated category with a non-degenerated $t$-structure. Then the system of functors $\tH^i$ is conservative and $X\in \mathcal{D}^{\leq 0}$ if and only if $\tH^i(X) = 0$ for $i>0$. Similarly $X\in \mathcal{D}^{\geq 0}$ if and only if $\tH^i(X)= 0$ for $i<0$.
\end{proposition}
\begin{definition}
  Let $\mathcal{D}_1,\mathcal{D}_2$ be triangulated categories equipped with $t$-structures. A functor of triangulate categories $F:\mathcal{D}_1\to \mathcal{D}_2$ is called left or right $t$-exact if $F(\mathcal{D}_1^{\geq 0}) \subseteq \mathcal{D}_2^{\geq 0}$ or $F(\mathcal{D}_1^{\leq 0}) \subseteq \mathcal{D}_2^{\leq 0}$ respectively.
  The functor $F$ is called $t$-exact if it is left and right $t$-exact.
\end{definition}
\begin{definition}
  Let $\mathcal{D}_1,\mathcal{D}_2$ be triangulated categories equipped with $t$-structures and let $F:\mathcal{D}_1 \to \mathcal{D}_2$ be a functor of triangulated categories.
  The perverse functor $^pF$ associated to $F$ is the induced functor on the hearts $^pF = \tH^0 \circ F \circ j_1$ where $j_1$ denotes the inclusion functor $j_1:\mathcal{C}_1\to \mathcal{C}_2$.
\end{definition}
\begin{proposition}

\end{proposition}
\subsection{Derived Functors}
\section{Riemann-Hilbert Correspondence}

\subsection{Perverse Sheaves}
The Riemann-Hilbert correspondence states in great generality that there is a equivalence between a system of differential equations and the system of solutions.
Philosophically, this is a significant result because it yields a connection between the algebraic/analytic world of differential equations and the topological world associated to their solutions.
\section{Monodromy}
\section{Spectral Sequences}
