\chapter{$\D_X$-modules and the Riemann-Hilbert Correspondence}\label{Ch: Chapter1}
The subject of this chapter are modules over rings of differential operators.
For the purpose of this introduction $X$ can be a algebraic variety or smooth manifold.
The ring of differential operators $\D_X$ will be defined formally in the next section.
For the purpose of this section it's sufficient to note that local sections of $\D_X$ are of the form $\sum c_{\alpha \beta} \partial^\alpha$ with $c_{\alpha}$ local sections of $\O_X$ and $\partial^\alpha = \partial_1^{\alpha_1}\cdots \partial_n^{\alpha_n}$.

A $\D_X$-modules gives a canonical description of systems of differential equations.
Consider a system of differential equations
$$\sum_{j=1}^k P_{ij}(x,\partial) f_j = 0 $$
with unknown functions $f_j$ of $\O_X$ and differential operators $P_{ij}$.
The functions $f_j$ are somewhat arbitrary in the description of this system.
For instance, take $g_j=\lambda_j f_j$ for certain nonzero functions $\lambda_j$.
There is then a associated system of equations for $g_j$.
A solution of the $g_j$-system corresponds uniquely to a solution of the $f_j$-system.

Consider the cokernel $\M$ of the map
$$P:\D_X^k \to \D_X^m : \left(Q_1,\ldots, Q_k\right) \mapsto \left(\ldots, \sum_{j=1}^k Q_j P_{ij},\ldots\right).$$
This map is left $\D_X$-linear so $\M$ is a left $\D_X$-module.
Note that it is necessary to distinguish between left and right modules because differential operators form a non-commutative ring.
Direct verification shows that the solutions of the system of differential equations are encoded in $\Hom_{\D_X}(\M,\O_X)$.
\section{$\D_X$-modules}
Throughout this section let $X$ be a smooth algebraic variety over $\C$.
The properties discussed in this section are common knowledge in the field of $\D_X$-modules.
For detailed references see \cite{bjork1979rings}, \cite{kashiwara2003d} or \cite{hotta2007d}.

\begin{definition}
  The sheaf of differential operators $\D_X$ is the subsheaf of $\mathcal{E}nd_{\O_X}(\O_X,\O_X)$ generated by $\O_X$ and the vector fields $\Theta_X$.
\end{definition}
Observe that $\D_X$ is a sheaf of non-commutative rings.
Given local coordinates $x_1,\ldots, x_n$ on $X$ it holds that
$$\partial_i x_j - x_j\partial_i = \delta_{ij} $$
where $\delta$ denotes the Kronecker delta.

This non-commutativity exits the typical domain of algebraic geometry.
This can be resolved by consideration of a graded structure.
The essential observation here is that differential operators commute up to a element of lower order.
\begin{definition}
  The order filtration on $\D_X$ is defined inductively to be given by the sheaves of $\O_X$-submodules $F_i \D_X$ such that $F_0\D_X = \O_X$ and $[F_i\D_X, F_i \D_X] \subseteq F_{i-1}\D_X$.
\end{definition}
The term $F_i \D_X$ in the order filtration can be described as containing all differential operators of order less than or equal to $i$.
Indeed, given local coordinates $x_1,\ldots, x_n$ one can show that $F_i\D_X$ is the $\O_X$-module locally generated by $\partial^\alpha = \partial_1^{\alpha_1}\cdots \partial_n^{\alpha_n}$ where $\alpha$ is a multi-index with $\abs{\alpha}\leq n$.
The following observations are immediate.
\begin{lemma}
  The $F_i\D_X$ are coherent $\O_X$-modules and form a exhaustive filtration. This is to say that $\cup_{i\geq 0}F_i\D_X = \D_X$
  and that for any $i,j\geq 0$ it holds that $F_i\D_X \cdot F_j \D_X \subseteq F_{i+j}\D_X$.
\end{lemma}
There is a similar notion of filtrations on $\D_X$-modules $\M$.
Namely a collection of $\O_X$-submodules $F_i\M$ of $\M$ such that $\cup_i F_i \M = \M$ and $F_i\D_X \cdot F_j\M \subseteq F_{i+j}\M$.

Observe that, by definition of the order filtration, the graded object $\gr \D_X = \oplus_{i}F_i\D_X / F_{i-1}\D_X$ is a commutative $\O_X$-algebra.
It can be viewed as a subsheaf of $\O_{T^*X}$.
Denote $\pi$ for the projection of $T^*X \to X$.
Then any $\gr\D_X$-module $\mathcal{M}$ has a corresponding module on $T^* X$ defined by $\O_{T^*X} \otimes_{\pi^{-1}\gr \D_X} \mathcal{M}$.
By abuse of notation this module is still denoted $\mathcal{M}$ and it will always be implicitly assumed that $\gr \D_X$-modules live on $T^*X$ unless it is explicitly mentioned otherwise.
In particular, for a filtration of the $\D_X$-module $\M$ the graded object $\gr \M = \oplus_i F_i \M / F_{i-1}\M$ is a $\gr \D_X$-module.
\begin{lemma}
  A $\D_X$-module $\M$ is coherent if and only if it admits a filtration such that $\gr \M$ is a coherent $\gr \D_X$-module. Such a filtration is called a good filtration.
\end{lemma}
\begin{proof}
  A proof of this result may be found in \cite[Chapter 2]{hotta2007d}.
\end{proof}
\begin{lemma}
  Let $\M$ be a coherent $\D_X$-module, then the support of $\grrel\M$ in $T^* X$ is a independent of the chosen good filtration. It is called the characteristic variety of $\M$ and denoted $\Ch \M$.
\end{lemma}
\begin{proof}
  A proof of this result may be found in \cite[Chapter 2]{hotta2007d}.
\end{proof}
\begin{lemma}
  Let $\M$ be a coherent $\D_X$-module, then $\Ch \M$ is a homogeneous and isotropic closed subset of $T^* X$.
\end{lemma}
\begin{proof}
  This result may be found in \cite[Chapter 2]{kashiwara2003d}.
\end{proof}
\begin{lemma}
  Consider a short exact sequence of coherent $\D_X$-modules
  $$0 \to \M_1 \to \M_2 \to \M_3 \to 0 $$
  then it holds that
  $$\Ch \M_2 = \Ch \M_1 \cup \Ch \M_3. $$
\end{lemma}
\section{Characteristic Variety}
\section{Holonomic $\D_X$-modules}
\section{Derived Category Theory}
\section{Spectral Sequences}
\section{Riemann-Hilbert Correspondence}
