\chapter{Conclusion}
The main result in this master's thesis was a multivariate generalisation of Lichtin's upper bound for the collection of roots of Bernstein-Sato polynomials.
The proof of this result reduced to the case of local computations with monomials.
To relate the result on the resolution of singularities to the desired result on the original space the $\D$-module direct image functor was employed. 
The main new difficulty, unique to this multivariate case, was that the associated $\D_X$-modules are no longer holonomic. 
Instead the condition of relative holonomicity was employed and homological algebra was used to reduce back to the classical case. 

The use of homological algebra and spectral sequences is definitely the most technical part of the argument. 
An alternative approach has been suggested by van der Veer and will appear in a forthcoming joint paper. 
This approach is less technical and paints the same geometric picture. 

We also derived lower bounds for the Bernstein-Sato zero locus in terms of the jumping loci of mixed multiplier ideals and log-canical threshold polytopes. 
It is interesting to note that these proofs employ the poles of certain functions defined in terms of integrals. 
These functions are precisely the $\zeta_g$-functions discussed in the summary which were the original motivation for the study of Bernstein-Sato relations following a question by Gelfand. 

Finally, we considered how the notion of holonomicity may be applied to derive recursions for the moments of random variables. 
Combined with asymptotic results, which are available for holonomic sequences, this provided an alternative approach to derive bounds on the tails of a random variable. \\ 

\noindent
Let us conclude this thesis by discussing some related questions which remain open and may provide an interesting topic for future research. 

Lichtin's estimate provides a lower bound on the distance of the roots of the Bernstein-Sato  polynomial to the origin.
A upper bound on the distance to the origin is also known due to \cite{saito2006introduction}.
It would be interesting to also establish a multivariate generalisation of this result. 
The argument in this thesis invokes Noetherianity in the proof of \cref{thm: EstimateBernsteinSatoZeroLocust}. 
This can add an arbitrarily large number of shifts to the $b$-polynomial so that the current argument is unable to establish an upper bound on the distance to the origin. 

The alternative approach due to van der Veer, mentioned above, employs the so-called characteristic cycles instead. 
These are a refinement of the characteristic variety which also keeps track of the multiplicities. 
It follows from this argument that one can establish an upper bound on the number of shifts provided the multiplicities of the direct image can be controlled. 
The interaction of the direct image functor with characteristic cycles is known but appears to not to be written down explicitly anywhere.
It would be useful to write down the proof for this fundamental result. 
In particular, this then also produce the desired upper bound for the distance of the hyperplanes in the Bernstein-Sato zero locus to the origin. 

This thesis was mainly concerned with estimates for the zero locus of the Bernstein-Sato ideal. 
It is of course also interesting to estimate the Bernstein-Sato ideal itself. 
It has been conjectured by \cite{budur2015bernsteinB} that the Bernstein-Sato ideal is generated by linear polynomials. \\


\noindent
Regarding the application of holonomicity in probability theory we have mentioned that \cite{bitoun2019feynman} used the recursions in the moments to recover the parameters of a compactly supported probability distribution given a finite number of moments. 
It would be interesting to investigate similar questions in the case of non-compact distributions with finite moments.

Further, it would be significant if holonomicity can be used to improve algorithms for problems in probability. 
One possible example of this is the case of random matrix theory. 
Free probability theory allows one to compute the limiting eigenvalue distribution of a sum of large random matrices $A_n+B_n$ given the individual distributions of $A_n,B_n\in \C^{n\times n}$.  
Moreover, if certain functions associated to $A_n$ and $B_n$ are algebraic then this can be done in an algorithmic fashion due to \cite{rao2008polynomial}. 
It has been observed in {\it loc. cit.} that this algebraic assumption is not always satisfied. 
Given that algebraic functions are an example of holonomic functions it is natural to wonder whether a generalisation with holonomic functions is possible. 






