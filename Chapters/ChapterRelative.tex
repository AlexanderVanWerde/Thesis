\chapter{Relative Holonomic Modules}\label{ch: ChapterRelHol}
\section{Introduction}\label{sec: IntoductionChapterRelative}
Let $X$ be a smooth algebraic variety and $F:X\to \C^r$ a morphism with coordinate functions $f_1,\ldots, f_r$.
Introduce new variables $s_1,\ldots, s_r$ and abbreviate $F^s = f_1^{s_1}\cdots f_r^{s_r}$.
The local Bernstein-Sato ideal $B_{F,x}$ of $F$ at $x\in X$ is the collection of all $b(s_1,\ldots,s_r)\in \C[s]$ such that there exists some differential operator $P\in \D_{X,x}\otimes_{\C}\C[s]$ with
$$b(s_1,\ldots,s_r) F^s =  P\cdot F^{s + 1}$$
in the stalk at $x$.
The local Bernstein-Sato ideal only depends on the local singularity at $x$, in particular there are only finitely many different $B_{F,x}$.
The global Bernstein-Sato ideal $B_F$ is the intersection of all local Bernstein-Sato ideals.
The local Bernstein-Sato ideal is also a topic of interest when $F$ is a analytic function germ.

In the case $p=1$ the ring $\C[s]$ is a principal ideal domain and the monic generator of $B_{F,x}$ recovers the Bernstein-Sato polynomial.
Recall from \cref{sec: MonodromyBS} that the Bernstein-Sato polynomial encodes the eigenvalues of monodromy.
A generalisation is possible for the Bernstein-Sato zero locust $Z(B_F)$.

Let $D$ be the divisor determined by $\prod f_i$ and set $U= X\setminus D$.
For any $\lambda \in (\C^*)^r$ let $L_\lambda$ denote the line bundle on $U$ found by the pullback over $F$ of the line bundle on $(\C^*)^r$ with monodromy $\lambda_i$ around the $i$-th missing coordinate hyperplane.
The eigenvalues of monodromy are generalised by the set
$$S(F) = \{\lambda \in (\C^*)^r \ \vert\  \exists j\in \Z\  \exists x\in D: H^j(U_x,L_\lambda) \neq 0\} $$
where $U_x$ is the intersection of $U$ with a arbitrarily small ball centred at $x$.
If $F$ is a analytic function germ $X$ should be replaced by a arbitrarily small neighbourhood of $x$ and we denote the resulting set $S_x(F)$.
\begin{theorem}
  If $F$ is a morphism of smooth affine algebraic varieties or then $\op{Exp}(Z(B_F)) = S(F)$. If $F$ is a holomorphic germ at $x$ then $\operatorname{Exp}(Z(B_{F,x})) = S_x(F)$.
\end{theorem}
\begin{proof}
  This is proved in \cite{budur2019zero}.
\end{proof}
\begin{theorem}
  Every irreducible component of $Z(B_{F,x})$ of codimension $>1$ can be translated by an element of $\Z^r$ into a component of codimension one.
\end{theorem}
\begin{proof}
  This result is due to \cite{maisonobe2016filtration}.
\end{proof}
This shows that $Z(B_{F})$ is a natural topic of study.
%Since $Z(B_F)$ is the union of the $Z(B_{F,x})$ it is sufficient to study the local Bernstein-Sato zero loci and it may be assumed that $X$ is a affine algebraic variety.
The rationality of the roots of the Bernstein-Sato polynomial admits the following generalisation of Kashiwara's estimate.
Denote $\mu:Y\to X$ for the strong log resolution of $(X,D)$.
Then $\mu^*D = \sum a_i E_i$ is in normal crossings form and we let $G = F\circ \mu$ with coordinate functions $g_1,\ldots, g_r$.
\begin{theorem}
  Every irreducible component of $Z(B_{F,x})$ of codimension $1$ is a hyperplane of the form
  $$\operatorname{mult}_{E_i}(g_1) s_1 + \cdots + \operatorname{mult}_{E_i}(g_r)s_r + c_i $$
  with $c_i\in \Z_{>0}.$
\end{theorem}
\begin{proof}
  The rationality of the hyperplanes was established by \cite{sabbah1987proximite} and \cite{gyoja1993bernstein}.
  The concrete estimate for the slopes was established by \cite{budur2019zero}.
\end{proof}

The goal of this chapter is to generalise Lichtin's estimate in \cref{thm: LichtinEstimate} using the canonical divisor $K_{Y/X} = \sum k_i E_i$.
\begin{theorem}\label{thm: EstimateBernsteinSatoZeroLocust}
  Every irreducible component of $Z(B_{F,x})$ of codimension $1$ is a hyperplane of the form
  $$\operatorname{mult}_{E_i}(g_1) s_1 + \cdots + \operatorname{mult}_{E_i}(g_r)s_r + k_i + c_i=0$$
  with $c_i \in \mathbb{Z}_{> 0 }$.
\end{theorem}
The main new ingredient in our proof is a induction argument which reduces the problem to the monovariate case.
This induction process involves a tensor product to reduce the number of variables.
The non-exactness of the tensor product causes error-terms in the form of the $\Tor$-functor.
These error-terms may be controlled using homological results which are set up in \cref{sec: NonComHomological}

In fact, it is known that analytical and algebraic local Bernstein-Sato ideals agree.
Hence, the algebraic version follows from the analytical version.
Nonetheless, for notational simplicity the proof will be done in the algebraic case and we discuss the necessary modifications for the local analytic case at the end.
\begin{theorem}\label{thm: AnalyticAlgebraic}
  The local Bernstein-Sato ideal of $B_{F,x}$ of a function germ $F$ on $X$ equals the local Bernstein-Sato ideal on the analytification $X^{an}$.
\end{theorem}
\begin{proof}
  The original proof is in the unpublished work \cite{brianccon2002remarques}.
  We were unable to find a copy of this work.
  The proof may also be found in chapter 7 of \cite{MasterThesisBSLocal}.
\end{proof}
\section{Relative Notions}
For any regular commutative $\C$-algebra integral domain $R$ we define a sheaf of rings on $X\times \Spec R$ by
$$\D_{X}^R = \D_X \otimes_{\C} \O_{\Spec R}.$$
It will also be convenient to use the abbreviations $\O_{X}^R := \O_{X \times \Spec R}$.

The order filtration $F_p\D_X$ extends to a filtration $F_p\D^R_X = F_p\D_X \otimes_{\C} \O_R$ on $\D^R_X$ which is called the relative filtration.
The associated graded objects are denoted by $\grrel$. Denote $\pi:T^*X\times \Spec R \to X\times \Spec R$ for the projection map.
As in the case of $\D_X$-modules one can view $\pi^{-1}(\grrel\D_X^R)$ as a subsheaf of $\O_{T^*X}^R$ and for any $\grrel\D_X^R$-module $\mathcal{M}$ there is a corresponding module on $T^*X\times \Spec R$ defined by $\O_{T^*X}^R\otimes_{\pi^{-1}\grrel\D_X^R}\pi^{-1}\mathcal{M}$.
By abuse of notation the corresponding module on $T^*X\times \Spec R$ is still denoted with $\mathcal{M}$ and we adopt the perspective that $\grrel\D_X^R$-modules always live on $T^*X\times \Spec R$ unless explicitly mentioned otherwise.

Similarly to the case of $\D_X$ it holds that $\D_X^R$ is the sheaf of rings generated by $\O_X^R$ and $\Theta_X$ inside of $\mathcal{E}nd_{\C}(\O_X^R)$. Giving a left $\D_X^R$-module is equivalent to giving a $\O_X^R$-module $\M$ with $\Theta_X$-action such that
$\xi\cdot (fm) = f (\xi \cdot m)  + \xi(f)\ m  $
for any sections $f$ of $\O_X^R$ and $\xi$ of $\Theta_X$. Similarly, giving a right $\D_X^R$-module is equivalent to giving a $\O_X$-module $\M$ with $\Theta_X$-action such that $(mf)\cdot\xi = (m\cdot\xi)f - m\ \xi(f) $ for any sections $f$ of $\O_X^R$ and $\xi$ of $\Theta_X$.

The following results are analogous to those in \cref{Ch: ChapterDX}.
They follow from general results in \cite[Appendix III]{bjork1993analytic}.
\begin{proposition}\label{prop: CoherentAndGoodFiltration}
    A quasi-coherent $\D_X^R$-module $\M$ is coherent if and only if it admits a filtration such that $\grrel \M$ is coherent over $\grrel \D_X^R$. Such a filtration is called a good filtration.
\end{proposition}
\begin{proposition}
    Let $\M$ be a coherent $\D_X^R$-module, then the support of $\grrel \M$ in $T^*X \times \Spec R$ is independent of the chosen good filtration. It is called the characteristic variety of $\M$ and denoted $\Chrel \M$.
\end{proposition}
\begin{lemma}\label{lem: SESBehaviourA}
    Consider a short exact sequence of coherent $\D_X^R$-modules
    $$0\to \M_1 \to \M_2 \to\M_3 \to 0 $$
    then it holds that
    $$\Chrel \M_2 = \Chrel \M_1 \cup \Chrel \M_3. $$
\end{lemma}

A coherent $\D_X^R$-module $\M$ is said to be relative holonomic over $R$ if $\Chrel \M = \cup_w \Lambda_w \times S_w$
for irreducible conic Lagrangian subvarieties $\Lambda_w\subseteq T^*X$ and irreducible closed subvarieties $S_w\subseteq \Spec R$.
\begin{lemma}\label{lem: RelHolGs}
  The sheaf of $\D_X^{\C[s]}$-modules $\D_X^{\C[s]} F^s$ is relative holonomic.
\end{lemma}
\begin{proof}
  This result may be found as proposition 13 in \cite{maisonobe2016filtration}. The proof applies in both the analytic and algebraic cases.
\end{proof}

\begin{lemma}\label{lem: InclusionCharVar}
  Let $\M$ be a finitely generated $\D_Y^R$-module. Suppose that $\Chrel \M \subseteq \Lambda \times \Spec R$ for some, not necessarily irreducible, conic Lagrangian subvariety $\Lambda\subseteq T^*X$. Then $\M$ is relative holonomic.
\end{lemma}
\begin{proof}
  This result may be found in \cite{maisonobe2016filtration} in the analytical case and \cite{budur2019zero} in the algebraic case.
\end{proof}
The Bernstein-Sato ideal may be defined more generally for any $\D_X^R$-module $\M$ as $B_\M := \Ann_{R}\M$. To see how this generalises $B_F$ one considers $\D_X^R F^s$ as a $\Xt$-module.
Here $t$ is a new variable which commutes with sections of $\D_X$ and satisfies $ts_i - s_it = 1$ for any $i=1,\ldots,
p$.  The $\Xt$-module structure on $\D_X^R F^s$ is then defined by $tP(s)F^s = P(s+1)F^{s + 1}$ for any section $P$ of $\D_X^R$. From this point of view $B_F = B_{\D_X^R F^s / t\D_X^R F^s}$.

The Bernstein-Sato ideal may be recovered from the characteristic variety.
\begin{proposition}\label{prop: ProjectionBernsteinSatoRelativeChar}
  Let $\M$ be a relative holonomic $\D_X^R$ module. Then  $Z(B_\M) = \pi_2(\Ch^{rel}(\M))$ where $\pi_2:T^*X\times \Spec R \to \Spec R$ is the projection on the second coordinate.
\end{proposition}
\begin{proof}
  This result may be found in \cite{maisonobe2016filtration} in the analytical case and \cite{budur2019zero} in the algebraic case.
\end{proof}
\section{Direct Image Functor for $\D_X^R$-modules}
    In this section we state the natural generalisation of the direct image functor for $\D_X$-modules to the relative case of $\D_X^R$-modules. As with $\D$-modules this is the most natural for right-modules.

    Consider some morphism $\mu:Y\to X$ and denote $\mu^R$ for the induced map from $Y\times \Spec R$ to $X\times \Spec R$.
    A transfer $(\D_Y^R,(\mu^R)^{-1}\D_X^R)$-bimodule is defined by $\D_{Y\to X}^R:= \D_{Y\to X}\otimes_{\C} R .$ Written out, this means that $\D_{Y\to X}^R = \O_Y^R \otimes_{(\mu^R)^{-1}\O_X^R}(\mu^R)^{-1}\D_X^R$.
    \begin{definition}
      The direct image functor $\int_{\mu^R}$ from $D^{b,r}(\D_Y^R)$ to $D^{b,r}(\D_X^R)$ is defined to be $R\mu^R_* (\blank\otimes_{\D_Y^R}^L \D_{Y\to X}^R)$.
      For any $\D_Y^R$ module $\M$ the $j$-th direct image is the $\D_X^R$-modules $\int^j_{\mu^R} \M = H^j \int_{\mu^R} \M$.
      The subscript $\mu^R$ will be suppressed whenever there is no ambiguity.
    \end{definition}
    Observe that a free $\D_Y^R$-resolution for a complex $\M^\bullet$ is also a free $\D_Y$-resolution.
    Hence, the following isomorphism holds in $D^{b,r}(\D_X)$ on $Y\times \Spec R$
    $$\M^\bullet \otimes_{\D_Y^R}^L \D_{Y\to X}^R \cong \M^\bullet \otimes_{\D_Y}^L \D_{Y\to X}.$$
    Denote $p_X:X\times \Spec R \to X$ for the projection map.
    Due to \cref{prop: GrothendieckIsomorphism} one has a isomorphism
    $$R(p_X)_* \circ R\mu^R_* \cong R(p_X \circ \mu^R)_* \cong R\mu^* \circ R(p_Y)_*.$$
    Combining the right-hand-sides of these isomorphisms yields the $\D_X$-module direct image.
    This is to say that the $\D_Y^R$-module direct image computes the $\D_Y$-module direct image with additional structure.

    By definition as a derived functor a long exact sequence is immediate.
    \begin{proposition}
        For any short exact sequence of $\D_Y^R$-modules
        $$0\to \M_1 \to \M_2 \to \M_3 \to 0$$
        there is a long exact sequence in direct images
        $$0\to \int^0 \M_1 \to \int^0 \M_2 \to \int^0 \M_3 \to \int^1 \M_1 \to \cdots. $$
    \end{proposition}
    Analogously to the case of $\D_X$-modules the direct image has the following conservation properties.
  \begin{definition}
      A $\D_Y^R$-module $\M$ is said to be $\mu$-good if $X$ admits a open cover $\{V_j\}_{j\in J}$ such that $\M$ has a good filtration on $\mu^{-1}(V_j)$ for every $j\in J$.
  \end{definition}
  \begin{theorem}
      Suppose that $\mu$ is proper and let $\M$ be a $\mu$-good $\D_Y^R$-module. Then $\int \M$ has coherent cohomology.
  \end{theorem}
  \begin{proof}
    The proof for $\D_X$-modules in chapter 3 of \cite{sabbah2011introduction} is still applicable provided Grauert's coherence theorem is replaced with the algebraic version. This is theorem 3.2.1 of \cite{EGAIII}.
  \end{proof}
  We want to establish that direct images also conserve holonomicity.
  A analytical version of this result is found in theorem 1.17 of \cite{monteiro2016riemann}.
  The analytic analogue of $\D_{Y}^R$ is given by $\D_{Y\times U/U}:= \D_Y\otimes_{\O_Y}\O_{Y\times U}$ where $U$ denotes the analytification of $\Spec R$.
  It is clear how notions such as the relative characteristic variety may be analytified.

  Since analytification is a exact functor it is compatible with good filtrations and $(\grrel\M)^{an} \cong \grrel \M^{an}$.
  In particular $\Chrel \M^{an}$ is the analytification of $\Chrel\M$.
  Using \cref{prop: GrothendieckIsomorphism} the exactness of analytification also implies that it commutes with the direct image.

  Now consider the following cotangent diagram.
  $$
  \begin{tikzcd}
      & \mu^* T^* X\times \Spec R\arrow[swap]{ld}{T^* \mu \times \operatorname{Id}_{\Spec R}} \arrow{rd}{\widetilde{\mu}\times \operatorname{Id}_{\Spec R}} & \\
      T^* Y \times \Spec R& & T^*X\times \Spec R
  \end{tikzcd}
  $$
  \begin{theorem}\label{thm: UpperBoundChrel}
    Suppose that $\mu$ is proper and let $\M$ be a $\mu$-good $\D_Y^R$-module. Then, for any $j\geq 0$
    $$\Chrel\left(\IntJ{j} \M \right)\subseteq  \widetilde{\mu}\times \operatorname{Id}_{\Spec R}\left((T^*\mu \times \operatorname{Id}_{\Spec R})^{-1}(\Chrel \M) \right).$$
  \end{theorem}
  \begin{proof}
    The analytic version of this statement is provided by corollary 4.3 in \cite{schapira1994index}.
    This implies the algebraic version due to all involved notions commuting with analytification.
  \end{proof}
  \begin{theorem}\label{thm: RelHolConserved}
      Suppose that $\mu$ is proper and let $\M$ be a $\mu$-good relative holonomic $\D_Y^R$-module. Then $\int \M$ has relative holonomic cohomology.
  \end{theorem}
  \begin{proof}
    Use \cref{lem: IsotropicDirectImage} on the upper bound of \cref{thm: UpperBoundChrel}. And apply \cref{lem: InclusionCharVar} for the desired result.
  \end{proof}
\section{Non-commutative Homological Notions}\label{sec: NonComHomological}
In this section we discuss homological notions associated to the $\Ext$-functor over the noncommutative sheaf of rings $\D_X^R$.
These notions are particularly well-behaved for relatively holonomic modules.
The results are sheaf-theoretic rewordings of the similar results in \cite{budur2019zero} which are themselves derived from the appendices of \cite{bjork1993analytic}.
\begin{definition}
  For a non-zero coherent sheaf of $\D_X^R$-modules $\M$ the smallest integer $k\geq 0$ such that $\Ex{\D_X^R}{k}(\M,\D_X^R)\neq 0$ is called the grade of $\M$ and is denoted $j(\M)$.
\end{definition}


The following proposition gives geometrical meaning to grades.
\begin{proposition}\label{prop: AXRBehaviourGradesJAndChrel}
  For coherent $\D_X^R$-modules $\M$ it holds that
  $$j(\M) + \dim \Chrel \M = 2n + \dim R $$
  where $\dim R$ denotes the Krull dimension of the ring $R$.
\end{proposition}
\begin{proof}
  This is lemma 3.2.2 in \cite{budur2019zero}.
\end{proof}
\begin{corollary}\label{cor: GradeIFFBernsteinIdeal}
  Let $\M$ be a relative holonomic $\D_X^R$-module. Then $\M$ has grade strictly greater than $n$ if and only if $B_\M$ is non-zero.
\end{corollary}
\begin{proof}
  This is immediate from \cref{prop: ProjectionBernsteinSatoRelativeChar} and \cref{prop: AXRBehaviourGradesJAndChrel}.
\end{proof}
\begin{definition}
  A non-zero coherent sheaf of $\D_X^R$-modules $\M$ is called $j$-pure if $j(\N)=j(\M)=j$ for every non-zero submodule $\N$.
\end{definition}
\begin{lemma}\label{lem: ExtGrade}
  Let $\M$ be a non-zero coherent $\D_X^R$-module of grade $j$. Then $\Ex{\D_X^R}{k}(\M,\D_X^R)$ has grade greater than or equal to $k$ for any $k\geq 0$ and $\Ex{\D_X^R}{j}(\M,\D_X^R)$ is a $j$-pure $\D_X^R$-module.

  Moreover $\M$ is $j$-pure if and only if $\Ex{\D_X^R}{j}(\Ex{\D_X^R}{k}(\M,\D_X^R),\D_X^R)=0$ for every $k\neq j$.
\end{lemma}
\begin{proof}
  This is lemma 4.3.5 in \cite{budur2019zero}.
\end{proof}
\begin{lemma}
  Let $\M$ be a relative holonomic $\D_X^R$-module of grade $j$. Then $\Ex{\D_X^R}{j}(\M,\D_X^R)$ is a relative holonomic $\D_X^R$-module and
  $$\Ch^{rel}\Ex{\D_X^R}{j}(\M,\D_X^R)\subseteq \Chrel \M. $$
\end{lemma}
\begin{proof}
  This is lemma 3.2.4 in \cite{budur2019zero}.
\end{proof}
\begin{lemma}
  Let $P \subseteq R$ be a prime ideal and let $\M$ be a coherent $\D_X^{R/P}$-module. If $\M$ is relative holonomic as a $\D_X^R$-module then it is also relative holonomic over $\D_X^{R/P}$.
\end{lemma}
\begin{proof}
  That $\M$ is relative holonomic over $\D_X^{R}$ means that it admits a good filtration such that
  $$\supp \gr_{\D_X^R}^{rel} \M  = \bigcup \Lambda \times S_\Lambda $$
  for Lagrangian subvarieties $\Lambda \subseteq T^* X \times \Spec R$ and algebraic varieties $S_\Lambda \subseteq \Spec R$.
  This filtration descends to a good filtration over $\D_{X}^{R/P}$ and it holds that
  $$\supp \gr_{\D_X^{R/P}}^{rel}\M = (\Id_{T^*X} \times \Delta)^{-1}(\supp \gr_{\D_X^R}^{rel}\M)$$
  where $\Delta:\Spec R/P\to \Spec R$ is the closed embedding.
  This yields the desired result.
\end{proof}
\begin{lemma}\label{lem: TorRelHol}
  Let $\M$ be a relative holonomic $\D_X^R$-module and let $P\subseteq R$ be a prime ideal. Then, for any $k\geq 0$, $\Tr{\D_X^R}{k}(\M, \D_X^{R/P})$ is a relative holonomic $\D_X^{R/P}$-module.
 \end{lemma}
\begin{proof}
  Compute $\Tr{\D_X^R}{k}(\M, \D_X^{R/P})$ with a locally free $\D_X^R$-resolution of $\D_X^{R/P}$.
  Then \cref{lem: SESBehaviourA} and \cref{lem: InclusionCharVar} show that it is a relative holonomic $\D_X^R$-module.
  The claim follows by the foregoing lemma.
\end{proof}
\begin{lemma}\label{lem: NotBernsteinInjectiveAutomorphism}
  Let $\M$ be a relative holonomic $\D_X^R$-module which is $(n+k)$-pure for some $0\leq k \leq \dim R$. If $b\in R$ is not contained in any minimal prime ideal containing $B_\M$ then multiplication by $b$ induces injective automorphisms on $\M$ and $\Ex{\D_X^R}{n+k}(\M,\D_X^R)$. Moreover, there exists a good filtration on $\M$ such that $b$ induces a injection on $\grrel \M$.
\end{lemma}
\begin{proof}
  This is lemma 3.4.2 in \cite{budur2019zero}.
\end{proof}
The proof of the following lemma is a slight modification on the proof of proposition 3.4.3 in \cite{budur2019zero}.
\begin{lemma}\label{lem: RestrictToCM}
  Let $\M$ be a non-zero relative holonomic $\D_X^R$-module of grade $j(\M) = n$  then, for any irreducible $b\in R$, it holds that $\M \otimes_R R/(b)$ is a non-zero relative holonomic  $\D_X^{R/(\ell)}$-module of grade $n$.
\end{lemma}
\begin{proof}
  Applying \cref{lem: TorRelHol} with $k=0$ yields that $\M \otimes_R R/(b)$ is a relative holonomic $\D_X^{R/(\ell)}$-module.

  It remains to establish that $\M \otimes_R R/(b)$ is non-zero of grade $n$.
  By taking a free resolution of $\M$ one has that
  $$R\Hom_{\scaleto{\D_X^R}{8pt}}(\M,\D_X^R)\otimes_{\scaleto{\D_X^R}{8pt}}^L \D_X^{R/(b)} \cong R \Hom_{\scaleto{\D_X^{R/(b)}}{8pt}}(\M\otimes_{\scaleto{\D_X^R}{8pt}}^L \D_X^{R/(b)}, \D_X^{R/(b)}) $$
  where we note that $\D_X^{R/(b)}$ is a $\D_X^{R}$-bimodule so that both tensor products are well-defined.
  We compare the Grothendieck spectral sequences\todo{Applicability Grothendieck: \cite{budur2019zero}} of both sides of this isomorphism.

  The spectral sequence associated with the right-hand-side has $E_2$-sheet
  $$E^{pq}_2 = \Ex{\D_X^{R/(b)}}{p}(\Tr{\D_X^R}{-q}(\M, \D_X^{R/(b)}), \D_X^{R/(b)}).$$
  Recall from \cref{lem: ExtGrade} that terms with $p>n$ have grade greater than $n$ and due to \cref{prop: AXRBehaviourGradesJAndChrel} there are no non-zero terms with $p<n$.
  Hence, the only term with $p+q = n$ which could potentially have degree $n$ is $E^{n0}_2$.
  If we can show that the total cohomology of degree $n$ on the left-hand-side has grade $n$ then it follows that $\Ex{\D_X^{R/(b)}}{n}(\M\otimes_{\scaleto{\D_X^R}{8pt}} \D_X^{R/(b)}, \D_X^{R/(b)}) \neq 0$ which is the desired result.

  The spectral sequence associated to the left-hand-side has $E_2$-sheet given by
  $$E_2^{pq} =\Tr{\D_X^R}{-p}( \Ex{\D_X^R}{q}(\M,\D_X^R), \D_X^{R/(b)}).$$
  Note that there are no non-zero differentials out of $E_j^{0n}$ for $j\geq 2$.
  Further, the differentials into $E^{0n}_j$ stem from $E^{-j(n+j-1)}_j$ which is a subquotient of $\Tr{\D_X^R}{j}(\Ex{\D_X^R}{n+j-1}(\M,\ab \D_X^R)\ab , \D_X^{R/(b)})$.
  Observe that $\D_X^R \xrightarrow{b} \D_X^R $ yields a free resolution for $\D_X^{R/(b)}$.
  It follows that $E^{-j(n+j-1)}_j=0$ for $j\geq 2$ whence $E_j^{0n} = E_2^{0n}$ for all $j\geq 2$.
  It remains to show that that $E_2^{0n}$ has grade $n$, then the the total cohomology of degree $n$ has grade $n$ and this concludes the proof.

  Denote $\E^n := \Ex{\D_X^R}{n}(\M,\D_X^R)$, by \cref{lem: ExtGrade} it holds that $\E^n$ is a $n$-pure relative holonomic $\D_X^R$-module.
  By \cref{lem: NotBernsteinInjectiveAutomorphism} it follows that $b$ induces injections on $\E^{n}$ and $\gr^{rel} \E^{n}$ for some appropriate filtration.
  In particular the vertical maps in the following diagram are injective
  $$\begin{tikzcd}
    0 \arrow{r} & F_{i-1} \E^n\arrow{r} \arrow{d}{b}& F_i \E^{n}\arrow{r} \arrow{d}{b}& \gr^{rel}_i \E^n\arrow{r}\arrow{d}{b} & 0\\
    0 \arrow{r} & F_{i-1} \E^n\arrow{r} & F_i \E^{n}\arrow{r} & \gr^{rel}_i \E^n\arrow{r} & 0
  \end{tikzcd} $$
  so the snake lemma yields a short exact sequence
  $$\begin{tikzcd}
    0 \arrow{r} & F_{i-1} \E^n\otimes_R R/(b)\arrow{r} & F_i \E^{n}\otimes_R R/(b)\arrow{r} & \gr^{rel}_i \E^n\arrow{r}\otimes_R R/(b) & 0.
  \end{tikzcd} $$
  The injectivity of $b$ on $\gr^{rel}\E^n$ implies that $b$ is also injective on $\E^n/F_i\E^n$. A similar application of the snake lemma now yields a short exact sequence
  $$\begin{tikzcd}
    0 \arrow{r} & F_{i} \E^n\otimes_R R/(b)\arrow{r} & \E^{n}\otimes_R R/(b)\arrow{r} & (\E^n/F_i \E^n )\arrow{r}\otimes_R R/(b) & 0.
  \end{tikzcd} $$
  A filtration on $\E^n \otimes_R R/(b)$ is induced by the image of $F_i\E^n$. By the injectivity of the short exact sequences one now has isomorphisms
  \begin{align*}
    F_i(\E^n \otimes_R R/(b)) \cong F_i\E^n / (F_i \E^n \cap b\E^n) \cong F_i \E^n / b F_i \E^n \cong  (F_i\E^n) \otimes_R R/(b)
  \end{align*}
  combined with the surjectivity of the first short exact sequence it follows that
  $$\gr^{rel}(\E^n \otimes_R R/(b)) \cong \gr^{rel}\E^n \otimes_R R/(b). $$
  It follows that
  $$\Ch^{rel}(\E^n \otimes_{\D_X^R} \D_X^{R/(b)})  = (\Id_{T^*X} \times \Delta)^{-1}(\Chrel \M)$$
  with $\Delta: \Spec R/(b) \to \Spec R$ the closed embedding as before.
  Since $\M$ has grade $n$ this equality and \cref{prop: AXRBehaviourGradesJAndChrel} imply that $\Ch^{rel}(\E^n \otimes_{\D_X^R} \D_X^{R/(b)})$ has dimension $n + \dim R - 1$.
  In particular it follows that $\E^n \otimes_{\D_X^R} \D_X^{R/(b)}$ is non-zero and has grade $n$. This concludes the proof.
\end{proof}
By \cref{lem: ExtGrade} the following definition gives a class of $j$-pure modules.
\begin{definition}
  A coherent $\D_X^R$-module $\M$ is said to be $j$-Cohen-Macaulay for some $j\geq 0$ if $\Ex{\D_X^R}{k}(\M,\D_X^R) = 0$ for any $k\neq j$.
\end{definition}
The property of being $j$-pure is not stable when restricting to a subscheme of $\Spec R$.
For the subclass of $j$-Cohen-Macaulay modules the restriction is more well-behaved.
\begin{lemma}\label{lem: CMLemmaBudur}
  Let $\M$ be a relative holonomic and $(n+k)$-Cohen-Macaulay $\D_X^R$-module. Let $b\in R$ be irreducible and non-vanishing on every irreducible component of $Z(B_\M)$. Then it holds that $\M\otimes_R R/(b)$ is a relative holonomic $(n+k)$-Cohen-Macaulay $\D_X^{R/(b)}$-module or zero.
\end{lemma}
\begin{proof}
  This is shown in the proof of proposition 3.4.3 in \cite{budur2019zero}. This proof is similar to the proof of \cref{lem: RestrictToCM} which was based on \cite{budur2019zero}.
\end{proof}
\begin{lemma}\label{lem: GradeToCMOverOpen}
  Let $\M$ be a relative holonomic $\D_X^R$-module of grade $n+k$. Then there exists a open $\Spec R' \subseteq \Spec R$ such that $\M\otimes_R R'$ is a relative holonomic and $(n+k)$-Cohen-Macaulay $\D_X^{R'}$ module. Moreover it may be assumed that the complement of $\Spec R'$ in $\Spec R$ has codimension $>k$.
\end{lemma}
\begin{proof}
  This is established in the proof of lemma 3.5.2 in \cite{budur2019zero}.
\end{proof}
Relative holonomic $\D_X^R$-modules are not necessarily of finite length.
This seems to be a obstruction for generalising the proof of \cref{prop: StableZero}.
The following lemmas provide the necessary modifications and provide the same line of thought as the proof by \cite{kashiwara1976b} for \cref{prop: StableZero}.
\begin{lemma}
  For any $(n+k)$-Cohen-Macaulay $\D_X^R$-module $\M$ it holds that  $$\Ex{\D_X^R}{n+k}(\Ex{\D_X^R}{n+k}(\M,\D_X^R),\D_X^R) \cong \M.$$
\end{lemma}
\begin{proof}
  By taking a free resolution one sees that
  $$R\Hom_{\D_X^R}(R\Hom_{\D_X^R}(\M, \D_X^R),\D_X^R) \cong \M.$$
  Due to the $(n+k)$-Cohen-Macaulay assumption the only non-zero term in the Grothendieck spectral sequence of the left-hand-side is $\Ex{\D_X^R}{n+k}(\Ex{\D_X^R}{n+k}(\M,\D_X^R),\D_X^R)$ whence the desired result follows.
\end{proof}

\begin{lemma}\label{lem: StabilisationChains}
  Let $\M$ be a relative holonomic $\D_X^R$-module of grade $j(\M) = n+k$.
  Consider a chain of submodules $\cdots \to \M_2\to \M_1 \to \M_0 = \M $
  such that the quotients $\M_i/\M_{i+1}$ are isomorphic.
  Then there exists a open $\Spec R' \subseteq \Spec R$ and some sufficiently large $N$ such that $\M_i' = \M_i\otimes_R R'$ stabilises for $i\geq N$. Moreover, it may be assumed that $\Spec R \setminus \Spec R'$ has codimension strictly greater than $k$.
\end{lemma}
\begin{proof}
  By use of \cref{lem: GradeToCMOverOpen} it may be assumed that $\Spec R'$ is such that $\M_i'$, $\M'/\M_i'$ and $\M'/\M_1'$ are zero or $(n+k)$-Cohen-Macauley for any $i=0,\ldots,N$.
  For notational simplicity we abbreviate $\E^{k}(\M') :=\Ex{\D_X^{R'}}{k}(\M',\D_X^{R'})$.

  Observe that $ \M_i'/\M_{i+1}'$ is $(n+k)$-Cohen-Macaulay or zero for any $i\geq 0$.
  This may be used to establish that $\M'/\M_i$ and $\M_i$ are actually $(n+k)$-Cohen-Macaulay or zero for arbitrary $i\geq 0$.

  The injection $ \M_{i+1}'\to \M_i'$ induces exact sequences
  $$\E^{n+k+j}\left(\frac{\M_i'}{ \M_{i+1}'}\right) \to  \E^{n+k+j}\left(\M_i'\right) \to \E^{n+k +j}\left( \M_{i+1}'\right)\to 0.$$
  By induction on $i$ it follows that $\M_i'$ is $(n+k)$-Cohen-Macaulay or zero for any $i\geq 0$.
  Similarly the long exact sequence induced by the surjection $\M'/\M_{i+1}' \twoheadrightarrow \M'/\M_i'$ yields that $\M'/\M_i'$ is $(n+k)$-Cohen-Macaulay or zero for any $i\geq 0$.


  The fact that $\M_i'/\M_{i+1}'$ and $\M'/\M_{i}'$ are $(n+k)$-Cohen-Macauley implies that the morphisms $\E^{n+k}(\M_i') \to  \E^{n+k}(\M_{i+1}')$ and $\E^{n+k}(\M') \to \E^{n+k} (\M_i')$ are surjective.
  Note that $\E^{n+k}(\M)$ is a coherent sheaf over the Noetherian sheaf of rings $\D_X^{R'}$.
  Hence the kernels of $\E^{n+k}(\M') \to \E^{n+k} (\M_i')$ stabilise.
  This means that the maps $\E^{n+k}(\M_i') \to  \E^{n+k}(\M_{i+1}')$ are isomorphisms for sufficiently large $i$.
  This establishes that $\E^{n+k}(\M_{i+1}')$ stabilises.
  By $ \M_i'$ being $(n+k)$-Cohen-Macaulay this means that $\E^{n+k}(\E^{n+k}(\M_i')) \cong \M_i'$ stabilises.
\end{proof}
\begin{lemma}\label{lem: StabilisationtN}
  Let $\M$ be a relative holonomic $\D_X^R$-module of grade $j(\M) = n+k$ which comes equipped with the structure of a $\Xt$-module. Suppose that $k\geq 1$ and that the quotients $t^i\M/ t^{i+1}\M$ are isomorphic.
  Then there exists a open $\Spec R' \subseteq \Spec R$ and some $N\geq 1$ such that $\M' = \M \otimes_{R} R'$ is a relative holonomic $\D_X^{R'}$-module with $t^N \M' = 0$.
  Moreover, it may be assumed that $\Spec R \setminus \Spec R'$ has codimension strictly greater than $k$.
\end{lemma}
\begin{proof}
  By \cref{lem: StabilisationChains} a open $\Spec R'$ may be found such that $t^i \M'$ stabilises for $i\geq N$.
  By \cref{cor: GradeIFFBernsteinIdeal} there exists some non-zero $b(s_1,\ldots,s_r) \in B_{\M'}$ which is necessarily also in $B_{t^N\M}$.
  Note that one has the commutation relation
  $$tb(s_1,\ldots,s_r) = b(s_1+1,\ldots, s_r + 1)t.$$
  Since $t^{N+1}\M' = t^N\M'$ it follows by iteration that $b(s_1+n,\ldots, s_r + n)\in B_{t^N \M'}$ for any $n\geq 0$.
  This implies that $Z(B_{t^N\M'}) = \emptyset$ which means that $t^N\M' = 0$ due to \cref{prop: ProjectionBernsteinSatoRelativeChar}.
\end{proof}
\section{Estimation of the Bernstein-Sato Zero Locust}
This section contains the main result of this chapter, namely a proof of the improved estimate for the Bernstein-Sato zero locust which was announced in \cref{thm: EstimateBernsteinSatoZeroLocust}.
We use the same notation as \cref{sec: IntoductionChapterRelative}.
This proof is similar to the method employed by \cite{lichtin1989poles} and \cite{kashiwara1976b} but a new induction argument is required in the proof of \cref{lem: GradeNPlusOne}.\\

\noindent
For the estimation of $Z(B_{F,x})$ it may be assumed that $X$ is affine and admits global coordinates $x_1,\ldots,x_n$.
Similarly to \cref{sec: MonodromyBS} these global coordinates allow to restate the functional equation $P F^{s+1} = b F^s$ as the equation
$$F^{s+1}dx \cdot P^* = b F^s dx $$
in $\D_X^{\C[s]} F^s \otimes_{\O_X}\omega_X$.
The corresponding module $\M$ on $Y$ will be the submodule of $\omega_Y\otimes_{\O_Y} \D_Y^{\C[s]} G^s$ spanned by $G^s \mu^*(dx)$.
Observe that $\omega_Y\otimes_{\O_Y} \D_Y^{\C[s]} G^s$ is relative holonomic due to \cref{lem: RelHolGs} so the submodule $\M$ is certainly also relative holonomic.
One can equip $\M$ with the structure of a $\XtR{Y}{}$-module by the action $t G^s \mu^*(dx) = G^{s+1} \mu^*(dx)$.
The following statement and proof are analogous to \cref{lem: UpstairsB}.
\begin{lemma}\label{lem: BernsteinSatoPolynomialUpstairs}
  In the notation of \cref{sec: IntoductionChapterRelative} a polynomial of the form
  $$b(s)=\prod_{i=1}^r \prod_{j=1}^N(\operatorname{mult}_{E_i}(g_1)\ab s_1 + \cdots + \operatorname{mult}_{E_i}(g_r)\ab s_r + k_i + j)$$
  belongs to the Bernstein-Sato ideal $B_{\M/t\M}$ if $N\geq 0$ is sufficiently large.
\end{lemma}
\begin{proof}
  This may be checked locally.
  Suppose we are working near a point $y\in Y$ which is on $E_i$ if and only if $i\in I$.
  Pick local coordinates $z_1,\ldots,z_n$ where $z_i$ determines $E_i$ if $i\in I$.

  In these local coordinates $G^s = \prod_{j=1}^r u_j^{s_j} \prod_{i\in I} z_i^{\sum_{j=1}^r \mult_{E_i}(g_j)s_j}$ and $\mu^*(dx) = v \prod_{i\in I} z_i^{k_i} dz$
  For any $i\in I$ set $\xi_i := \partial_i - \sum_{j=1}^{r}s_i\partial_i(u_i)u_i^{-1}$ and let $P = v^{-1}(\prod_{j=1}^r u_j^{-1}) (\prod_{i\in I}\xi_i^{\sum_{j=1}^r\mult_{E_i}(g_j)}) v$ then
  $$v\prod_{j=1}^r u_j^{s_j + 1}\prod_{i\in I}z_i^{\sum_{j=1}^r \mult_{E_i}(g_j)(s_j+1) + k_i}dz \cdot P =  q(s) v\prod_{j=1}^r u_j^{s_j}\prod_{i\in I}z_i^{\sum_{j=1}^r \mult_{E_i}(g_j)s_j + k_i} $$
  where
  $$q(s) = \prod_{i\in I}(\sum_{j=1}^r \mult_{E_i}(g_j)s_j + \mult_{E_i}(g_j) + k_i)\cdots(\sum_{j=1}^r \mult_{E_i}(g_j)s_j + 1 + k_i ).$$
\end{proof}
Recall that the $\D_Y^R$-module direct image computes the $\D_Y$-module direct image with additional structure.
In particular the $\D_X$-linear endomorphism $t$ induces a endomorphism on $\Int\M$.
By the functoriality of the direct image $ts_i - s_it = 1$ for any $i$ so $\Int\M$ is equipped with the structure of a $\Xt$-module.
Similarly, the functoriality and \cref{lem: BernsteinSatoPolynomialUpstairs} yields a $b$-polynomial of a desirable form for $\Int \M/t\Int\M$.

Consider the surjection $\D_Y^R\to \M$ induced by $1\mapsto G^s \mu^*(dx)$.
The associated long exact sequence includes a morphism $\int^0 \D_Y^R \to \Int\M$.
Observe that $\int^0 \D_Y^R = R^0\mu_*(\D_{Y\to X}^R)$ contains global section corresponding to $1$ in $\D_{Y\to X}^R$.
Write $u$ for the image of this section in $\Int \M$ and denote $\U$ for the right $\Xt$-module generated by $u$.
Similarly to \cref{lem: SurjectiveUf} the following lemma shows that the main difficulty is to transfer the $b$-polynomial for $\Int\M/t\Int\M$ into a $b$-polynomial for $\U/t\U$.
This will exploit \cref{lem: StabilisationtN} whence it is needed that $\Int\M /\U$ has grade at least $n+1$.\\
\begin{lemma}\label{lem: SurjectionUF}
  There is a morphism right $\D_\X^R$-modules $\U\to \D_\X^RF^s \otimes_{\O_\X}\omega_\X$ sending $u$ to $F^sdx$.
\end{lemma}
\begin{proof}
   The proof is identical to the proof of \cref{lem: SurjectiveUf} with $f$ replaced by $\prod f_i$.
\end{proof}

In what follows we want to consider the $\D_Y^{\C[s]}$-module $\M$ as a $\D_Y$-module.
This could disturb coherence.
To solve this one introduces new coordinates such that there are vector fields $\mathcal{S}_1,\ldots, \mathcal{S}_r$ which acts as $s_1,\ldots,s_r$ on the generator.

Note that there are finitely many codimension $1$ components in $Z(B_{F,x})$.
Hence, there exist $p$ independent linear polynomials $\sum_{i=1}^r d_{ij}s_i$ such that for any $j$ there is no hyperplane parallel to $\sum_{i=1}^r d_{ij}s_i = 0$ in $Z(B_{F,x})$.
Moreover, it may be assumed that the $d_{ij}$ are non-negative integers.
Introduce new coordinates $z_{n+1}, \ldots,z_{n+r}$ and set $\widetilde{f}_j = f_j\prod_{i=1}^r z_{n+i}^{d_{ij}}$ on $X\times \C^r$.
Note that the induced map $Y\times \C^r \to X\times \C^r$ is a resolution of singularities for $\prod \widetilde{f}_i$ and that $\widetilde{g}_j = g_j\prod_{i=1}^r z_{n+i}^{d_{ij}}$ is the pullback of $\widetilde{f}_i$.

For any $i=1,\ldots,p$ it holds that
$$\widetilde{G}^s \mu^*(dx)\cdot \partial_{n+i} = \sum_{j=1}^r d_{ij}s_j x_j^{-1} \widetilde{G}^s \mu^*(dx).$$
Since the linear polynomials are independent a appropriate $\C$-linear combination provides a vector field $\zeta_j$ with $\widetilde{G}^s \mu^*(dx)\cdot \zeta_j = s_{j}z_j^{-1}\widetilde{G}^s \mu^*(dx)$.
Set $\mathcal{S}_j = \zeta_jz_j$ so that $\widetilde{G}^s \mu^*(dx) \cdot \mathcal{S}_j = s_j \widetilde{G}^s \mu^*(dx) $.
This solves the coherence issue.
The following result and it's proof are analogous to \cref{lem: BernsteinTilde}.
\begin{lemma}\label{lem: ReplacementIsAllowed}
  For any $x\in X\times \{0\}^r$ it holds that if $b\in B_{\widetilde{F},x}$ then $b \in B_{F,x}$.
\end{lemma}
\begin{proof}
  Take local coordinates $x_1,\ldots, x_{n+r}$ near $x$ and let $P$ be in the stalk $\D_{X\times \C^r,x}^{\C[s]}$ such that $b \widetilde{F}^s = P \widetilde{F}^{s+1}$.
  Similarly to the above there is a $\C$-basis $\xi_1,\ldots,\xi_r$ for the span of $\partial_{n+1}, \ldots, \partial_{n+r}$ so that $\mathcal{S}_j := x_{n+j}\xi_j$ satisfies $\mathcal{S}_j \cdot \widetilde{F}^s = s_{j}\widetilde{F}^s$.
  Expand $P$ as a polynomials in $\xi_1,\ldots,\xi_r$
  $$P = \sum_{\alpha} P_\alpha \xi_{1}^{\alpha_1}\cdots \xi_{r}^{\alpha_r}$$
  where the coefficients $P_\alpha$ live in a stalk of $\O_{X\times \C^r}\otimes_{\O_X\times \C^r}\D_{X\times \C^r}^{\C[s]}$.

  Let $N$ be greater than the maximal value of $\abs{\alpha}$ then
  $$(x_{n+1}\cdots x_{n+r})^N b \widetilde{F}^s = \left(\sum_{\alpha} \prod_{i=1}^r (s_i + 1)^{\alpha_i} \sum_\beta Q_{\alpha\beta} \partial_1^{\beta_1}\cdots \partial_n^{\beta_n} \right)\widetilde{F}^{s+1}$$
  where the $P_\alpha$ were expanded as polynomials in $\partial_1,\ldots,\partial_n$ with coefficients $Q_{\alpha\beta}$ from $\O_{X\times \C^r}$.
  Observe that $\partial_1,\ldots, \partial_n$ act on the formal symbol $\widetilde{F}^{s+1}$ the same as they act on the formal symbol $F^{s+1}$.

  Now consider this functional equation on the analytification of $X\times \C^r$ and expand the $Q_{\alpha\beta}$ as power series at $x$.
  Identifying powers of $x_{n+1}\cdots x_{n+r}$ on both sides a functional equation with analytical coefficients for $F^s$ follows.
  This establishes that $b$ is in the analytic Bernstein-Sato ideal.
  It follows that $ b\in B_{F,x}$ since analytic and algebraic Bernstein-Sato ideals agree by \cref{thm: AnalyticAlgebraic}.
\end{proof}
Note that replacing $F$ by $\widetilde{F}$ leaves \cref{thm: EstimateBernsteinSatoZeroLocust} unchanged up to hyperplanes parallel to $\sum_{i=1}^r d_{ij}s_i = 0$.
These are not in $Z(B_{F,x})$ by assumption so, by \cref{lem: ReplacementIsAllowed}, it remains to prove the theorem for $\widetilde{F}$.
For notational simplicity we simply write $F$ instead of $\widetilde{F}$ and $X$ instead of $X\times \C^r$.
The dimensions of $X$ will be denoted $m = n+p$.

Let $\ell_1,\ldots,\ell_{r-1}\in \C[s]$ be degree one polynomials which will be fixed later.
For any $i=0\ldots,p$ let $L_i$ be the ideal of $\C[s]$ generated by $\ell_1,\ldots,\ell_i$.
Assume that the $\ell_i$ are chosen sufficiently generically so that $Z(L_{r-1})$ is a line.
\begin{lemma}\label{lem: CharVarEstimateW}
  The $\D_\Y$-module $\M\otimes_{\C[s]}\C[s]/L_{r-1}$ is coherent and it's characteristic variety satisfies $\Ch\M\otimes_{\C[s]}\C[s]/L_{r-1} \subseteq \Lambda \cup W $ where $\Lambda$ is isotropic and $W$ is a irreducible variety of dimension $m +1$ which dominates $\Y$.
\end{lemma}
\begin{proof}
  Recall that we ensured that $\M$ is a coherent $\D_\Y$-module.
  Hence, also $\M\otimes_{\C[s]}\C[s]/L_{r-1}$ will be a coherent $\D_\Y$-module.

  Take local coordinates $z_1,\ldots,z_n,z_{n+1},\ldots,z_{n+r}$ on $\Y$ as in the proof of \cref{lem: BernsteinSatoPolynomialUpstairs}.
  This is to say that locally
  $$G^s \mu^*(dx) = v \prod_{i=1}^nu_i^{s_j}z_i^{\sum_{j=1}^r M_{ij}s_j + m_i}\prod_{i=1}^n z_{n+i}^{\sum_{j=1}^r d_{ij}s_j} dz.$$
  Let $s_0$ denote a new variable so that $\C[s]/L_{r-1}\cong \C[s_0]$.
  Then $\M\otimes_{\C[s_0]}R/L_{r-1}$ may be viewed as the $\D_\Y$-module which is locally generated by a formal symbol
  $$[G^s \mu^*(dx)] =v \prod_{i=1}^{2n} u_i^{A_i s_0 + a_i}z_i^{B_i s_0 + b_i} dz $$
  where $A_i,B_i,a_i,b_i$ are complex numbers and we set $u_{n+i}=1$.
  Moreover, since the linear functions $\sum_{j=1}^r d_{ij}s_j$ on the final terms in $G^s\mu^*(dx)$ formed a basis for the linear polynomials there will be at least one $B_{i+n}$ which is non-zero.

  Denote $w = v\prod_{i=1}^n u_i^{a_i}$ and consider for any $j=1,\ldots,n+r$ the operation of $w^{-1}\partial_j wz_j$ on the generator
  $$[G^s \mu^*(dx)]\cdot w^{-1}\partial_j w =((B_j s_0 + b_j)z_j^{-1} + \sum_{i=1}^{n} A_i s_0 u_i^{-1}\partial_j(u_i) )[G^s \mu^*(dx)].$$
  Recall that the $s_1,\ldots,s_n$ could be produced by acting with a vector field.
  Since $s_0$ is found with affine relations it follows that there exists some differential operator $\mathcal{S}_0$ of degree $1$ such that $s_0 [G^s \mu^*(dx)] = [G^s \mu^*(dx)]\cdot \mathcal{S}_0$.
  Now we get a well-defined surjection $\D_\Y/I \twoheadrightarrow \M\otimes_{\C[s]}\C[s]/L_{r-1}$ where $I$ denotes the right ideal generated by $w^{-1}\partial_j wz_j - b_j - \mathcal{S} h_i$ with $h_j = B_j + z_j\sum_{i=1}^n A_iu_i^{-1}\partial_j(u_i)$ for $j=1,\ldots, n+r$.

  Note that $ z_j\sum_{i=1}^n A_iu_i^{-1}\partial_j(u_i) = 0$ for $j>n$.
  Hence, the $h_{n+j}$ are complex scalars and they are not all zero since there exists a non-zero $B_{n+j}$.
  After renumbering we now have that $h_1 \in \C^\times$.
  Denoting $\zeta_j, \sigma_0$ for the elements of $\gr \D_\X$ which correspond to $\partial_j, \mathcal{S}_0$ respectively it holds that $\gr I$ contains $z_j \zeta_j - h_j \sigma_0$ for any $j=1,\ldots, n+r$.
  Then also $h_1z_j \zeta_j - h_jz_1 \zeta_1$ is in $\gr I$ for any $j=2,\ldots, n+r$.
  This yields the desired bound for the characteristic variety.
\end{proof}
\begin{lemma}\label{lem: InjectiveEll}
  Any polynomial $b\in \C[s]$ which is not in $L_i$ induces a injective automorphisms on $\M \otimes_{\C[s]}\C[s]/L_i$.
\end{lemma}
\begin{proof}
  Observe that $\M$ has a trivial Bernstein-Sato ideal so that it has degree $m$ by \cref{cor: GradeIFFBernsteinIdeal}.
  By inductively applying \cref{lem: RestrictToCM} it holds that $\M \otimes_{\C[s]}\C[s]/L_i$ has degree $m$.
  In particular, it has a trivial Bernstein-Sato ideal.

  Similarly to the proof of \cref{lem: CharVarEstimateW} one can pick local coordinates $z_i$ such that $\M \otimes_{\C[s]}\C[s]/L_i$ is generated by some formal symbol $[G^s \mu^*(dx)]$.
  Further pick some isomorphism $\C[s]/L_i \cong \C[\widetilde{s}]$.
  By definition of the formal symbol $\partial_i$ acts on $[G^s \mu^*(dx)]$ as a polynomial in $\widetilde{s}$ with rational functions of the $z_i$ as coefficients.

  If $b$ is not injective it follows by clearing denominators that there is some non-zero polynomial $f = \sum_{\alpha,\beta} c_{\alpha,\beta} z^\alpha \widetilde{s}^\beta$ with $[G^s \mu^*(dx)] f = 0$.
  Further, it can be assumed that the degree of $f$ in $z$ is zero.
  Indeed, if $z_i$ occurs in $f$ then one can find a non-zero polynomial $g$ of lesser degree such that
  $$[G^s \mu^*(dx)] f\partial_1 = [G^s \mu^*(dx)]\partial_1 f + [G^s \mu^*(dx)] g.$$
  The left-hand-side of this equality vanishes and the term $[G^s \mu^*(dx)]\partial_1 f$ must also vanish since $\partial_i$ acts as a rational function.
  This means that $[G^s \mu^*(dx)] g = 0$.
  Repeating this procedure it may be assumed that $f$ is a non-zero polynomial in $\C[\widetilde{s}]$.

  Since $\C[\widetilde{s}]$ commutes with $\D_Y^{\C[\widetilde{s}]}$ it follows that $f$ is a non-zero Bernstein-Sato polynomial of $\M \otimes_{\C[s]}\C[s]/L_i$ which is a contradiction.
\end{proof}
The proof that $\Int\M/\U$ has a grade $\geq m+1$ involves a tensor product to reduce the number of variables.
The following lemma allows us to choose the $\ell_i$ appropriately such that the $\Tor$-terms associated to the tensor product can be controlled.
\begin{lemma}\label{lem: RelHolTorDegree}
  One can pick $\ell_1,\ldots,\ell_{r-1}$ such that $\Tr{\D_\X^{\C[s]/L_{i}} }{1}(K_i, \D_\X^{\C[s]/L_{r-1}})$ is a relative holonomic $\D_\X^{R/L_{r-1}}$-module of grade greater than or equal to $m+1$ for every $i=1,\ldots, r-1$.
  Here $K_i$ denotes the kernel of $\ell_i$ on $\int^1 (\M \otimes_{\C[s]}\C[s]/L_{i-1})$.
\end{lemma}
\begin{proof}
  By a taking a $\D_{\X}^{\C[s]/L_{i}}$-free resolution of $K_i$ one finds that
  $$R \Hom_{\scaleto{\D_\X^{ \C[s]/L_{r-1} }}{8pt} }(K_i\otimes^L_{\scaleto{ \D_\X^{\C[s]/L{i}} }{8pt} } \D_\X^{\C[s]/L_{r-1} },\D_\X^{ \C[s]/L_{r-1}} ) \cong R \Hom_{ \scaleto{\D_\X^{\C[s]/L_{i}}}{8pt} } (K_i,\D_\X^{\C[s]/L_{i}}) \otimes^L_{\scaleto{ \D_\X^{\C[s]/L_{i}}}{8pt} } \D_\X^{\C[s]/L_{r-1}}$$
  where we note that $\D_\X^{\C[s]/L_{r-1}}$ is a $\D_X^{\C[s]/L_{i}}$-bimodule so that both tensor products are defined.
  We compare the Grothendieck spectral sequences of both sides.

  The spectral sequence on the left-hand-side has terms
  $$E^{pq}_2 = \Ex{\D_\X^{\C[s]/L_{r-1}}}{p}(\Tr{\D_\X^{\C[s]/L_{i}} }{-q}(K_i,\D_X^{\C[s]/L_{r-1}}),\D_\X^{\C[s]/L_{r-1}}).$$
  Since $\Tr{\D_\X^{\C[s]/L_{i}} }{-q}(K_i,\D_X^{\C[s]/L_{r-1}})$ is a relative holonomic $\D_X^{\C[s]/L_{r-1}}$-module these terms are only non-zero for $p=m$ or $p = m+1$.
  In particular, the spectral sequence degenerates at $E_2$.
  By \cref{lem: ExtGrade} the statement that $\Tr{\D_\X^{\C[s]/L_i}}{1}(K_i, \D_\X^{\C[s]/L_{r-1}})$ has grade $\geq m+1$ is equivalent to the total cohomology of degree $p + q = m-1$ having grade $\geq m+1$.

  The spectral sequence on the right-hand-side has terms
  $$E_2^{pq} = \Tr{\D_\X^{\C[s]/L_{i}} }{-p}(\Ex{ \D_\X^{\C[s]/L_{i}} }{q} (K_i,\D_\X^{\C[s]/L_{i}}), \D_\X^{\C[s]/L_{r-1}}).$$
  The claim follows if we can ensure that that all terms with $p+q = m -1$ vanish on $\X\times\Spec R$ for some open subset $\Spec R\subseteq \C^p$.
  Indeed, then by \cref{cor: GradeIFFBernsteinIdeal} the terms have grade $m+1$ and it follows that the same must hold for the total cohomology.
  \\

  The $\ell_i$ and the open $\Spec R$ are constructed by induction on $i$. For any $i,j,k$ with $k\leq i$ denote $\E_{ik}^{n+j}:= \Ex{\D_\X^{R/L_{k}} }{n+j}(K_k, \D_\X^{R/L_k}) \otimes_{\scaleto{\D_\X^{R/L_{k}}}{8pt}} \D_\X^{R/L_{i}}$.
  In every induction step it is ensured that
  \begin{enumerate}[label=(\roman*)]
    \item $\E_{ii}^{n+j}$ is $(n+j)$-Cohen-Macaulay over $\D_X^{R/L_i}$ or zero for every $j\geq 0$.
    \item $Z(L_i)\cap \Spec R \neq 0$.
    \item $\ell_i$ induces a injection on $\E_{(i-1)k}^{n+j}$
    for every $j\geq 0$ and $k<i$.
  \end{enumerate}
  By abuse of notation $L_i$ may also denote the ideal of $R$ generated by $\ell_1,\ldots,\ell_i$.

  Take some arbitrary $\ell_1$ for the base-case and use \cref{lem: GradeToCMOverOpen} to find a open $\Spec R \subseteq \C^r$ such that $\E^{n+j}_{11}$  is $(n+j)$-Cohen-Macaulay for every $j\geq 0$.
  This only requires removing a strict closed subset of $\Spec \C[s]/L_1$ so $Z(L_1)\cap \Spec R = \Spec R/L_1$ is non-empty.
  The final property is vacuous for $i=1$.

  Now assume that $i>1$ and that $\ell_1,\ldots, \ell_{i-1}$ are already constructed.
  First let's ensure that $\ell_i$
  induces a injection on $\E^{n+j}_{(i-1)k}$ for every $j\geq 0$ and $k<i$.
  By iterative application of \cref{lem: CMLemmaBudur} it holds that $\E^{n+j}_{(i-1)k }$
  is $(n+j)$-Cohen-Macaulay over $\D_\X^{L_{i-1}}$.
  Take $\ell_i$ so that the induced element of $R/L_{i-1}$ is non-constant and does not vanish on any irreducible component of the Bernstein-Sato zero locust of $\E^{n+j}_{(i-1)k}$ for every $j\geq 0$ and $k<i$.
  Then, by \cref{lem: NotBernsteinInjectiveAutomorphism} the desired injectivity follows.
  As before, \cref{lem: GradeToCMOverOpen} can be used to to find a open $\Spec R' \subseteq \Spec R$ such that $\E^{n+j}_{ii}$  is $(n+j)$-Cohen-Macaulay for every $j\geq 0$ and  $Z(L_i)\cap \Spec R' = \Spec R'/L_i$ is non-empty.
  Note that replacing $\Spec R$ by $\Spec R'$ will conserve the induction hypothesis.
  This concludes the inductive construction of the $\ell_i$.\\


  Applying injectivity of $\ell_i$ on $\E_{(i-1)k}^{n+j}$ with the free resolution $\D_\X^{R/L_{i-1}}\xrightarrow{\ell_i}\D_\X^{R/L_{i-1}}$ for $\D_\X^{R/L_i}$ yields that $\Tr{\D_\X^{R/L_i}}{m}(\E_{(i-1)k}^{n+j}, \D_\X^{R/L_i}) = 0$ for all $m>0$.
  By taking a $\D_\X^{R/L_{i-1}}$-free resolution of $\mathcal{E}_{(i-1)k}^{n+j}$ it follows that
  $$\mathcal{E}_{(i-1)k}^{n+j} \otimes^L_{\scaleto{\D_\X^{R/L_{i-1}}}{8pt}} \D_\X^{R/L_{r-1}} \cong  \mathcal{E}_{ik}^{n+j}  \otimes^L_{\scaleto{\D_\X^{R/L_{i}}}{8pt}}\D_\X^{R/L_{r-1}}. $$


  Iterative application of the isomorphism yields $\E_{ii}^{n+j}\otimes^L_{\scaleto{\D_\X^{R/L_{i}}}{8pt}} \D_\X^{R/L_{r-1}} \cong \E_{(p-2)i}^{n+j}\otimes^L_{\scaleto{\D_\X^{R/L_{p-2}}}{8pt}} \D_\X^{R/L_{r-1}}$.
  This means that
  $$\Tr{\D_\X^{\C[s]/L_{i}}}{-r} (\Ex{\D_\X^{\C[s]/L_{i}} }{q} (K_i,\D_\X^{\C[s]/L_{i}}), \D_\X^{\C[s]/L_{r-1}})\cong \Tr{\D_\X^{\C[s]/L_{i}} }{-r}(\E_{(p-2)i}^{q}, \D_\X^{\C[s]/L_{r-1}}).$$
  The right right-hand-side of this isomorphism was already observed to vanish for any $r < 0$ and the left-hand-side is precisely the $E^{pq}_2$-term of the spectral sequence.
  This establishes that the $E^{pq}_2$-terms with $r+q = m -1$ vanish for $q>m-1$.
  The remaining term $E^{m -1,0}_2$ is zero regardless since it involves $\Ext^{m - 1}$ of a relative holonomic module.
  This concludes the proof.
\end{proof}
\begin{lemma}\label{lem: GradeNPlusOne}
  The relative holonomic $\D_\X^{\C[s]}$-module $\Int\M/\U$ has grade $j(\Int\M/\U)\geq m+1$.
\end{lemma}
\begin{proof}
  Let $\ell_1,\ldots,\ell_{r-1}$ be the degree one polynomials provided by \cref{lem: RelHolTorDegree}.
  For any $i=0,\ldots, r-1$ denote $\M_i =  \M \otimes_{\C[s]} \C[s]/L_{i}$.
  The first task is to relate $(\Int\M) \otimes_{\C[s]} \C[s]/L_{r-1}$ with $\Int\M_{r-1}$.

  Recall from \cref{lem: InjectiveEll} that $\ell_{i}$ is injective on $\M_{i-1}$.
  This implies that
  $\ell_{i}\Int \M_{i-1} = \int^0 \ell_{i}\M_{i-1}.$
  The injective automorphisms of $\ell_i$ on $\M_{i-1}$ induces a long exact sequence of $\D_X^{\C[s]/L_{i-1}}$-modules
  $$0 \to \Int\M_{i-1} \xrightarrow{\ell_i} \Int \M_{i-1} \to \Int \M_i \to \cdots $$
  whence $(\Int \M_{i-1})\otimes_{\C[s]/L_{i-1}} \C[s]/L_i$ is a submodule of $\Int\M_i$.
  The quotient is isomorphic to the kernel $K_i$ of $\ell_i$ on $\int^1\IntMinspace\M_{i-1}$.

  Applying a tensor product with $\C[s]/L_{r-1}$ to the inclusion $\Int \M_{i-1} \otimes_{\C[s]/L_{i-1}} \C[s]/L_i \hookrightarrow \Int\M_i$ yields a exact sequence
  $$\Tr{\C[s]/L_{i-1}}{1}\left(K_i, \frac{\C[s]}{L_{r-1}}\right) \to \left(\Int \M_{i-1}\right) \otimes_{\C[s]/L_{i-1}}\frac{\C[s]}{L_{r-1}} \to \left(\Int\M_i\right) \otimes_{\C[s]/L_{i-1}}\frac{\C[s]}{L_{r-1}}.$$
  By choice of the $\ell_i$ it holds that $\Tr{\D_\X^{\C[s]/L_i} }{1}(K_i, \D_\X^{\C[s]/L_{r-1}})$ has grade $\geq m+1$.
  This implies the existence of a non-zero polynomial $b_i\in \C[s]/L_{r-1}$ which annihilates $\Tr{\C[s]/L_{i}}{1}(K_i, \C[s]/\ab L_{r-1} ))$.

  Denote $B = \prod_{i=1}^{r-1}b_i$ and note that the kernels of the automorphisms induced by $B^N$ form a increasing sequence inside the coherent $\D_X^{\C[s]/L_{r-1}}$-module $(\Int \M_{i-1})\otimes_{\C[s]/L_{i-1}}\C[s]/L_{r-1}$.
  Such a increasing sequence must stabilise for sufficiently large $N$.
  Then it follows that the intersection of $\img \Tr{\C[s]/L_{i-1}}{1}(K_i, \C[s]/L_{r-1})$ and $ B^N((\Int \M_{i-1}) \otimes_{\C[s]/L_{i-1}}\C[s]/L_{r-1})$ is trivial.
  This means that there are injections $$B^N \left(\Int \M_{i-1}\right) \otimes_{\C[s]/L_{i-1}}\C[s]/L_{r-1} \hookrightarrow B^N \left(\Int\M_i\right)\otimes_{\C[s]/L_{i-1}}\C[s]/L_{r-1}.$$
  In particular $B^N(\Int\M)\otimes_{\C[s]}\C[s]/L_{r-1}$ is a submodule of $\Int\M_{r-1}$.


  Since $\mu$ is proper \cref{prop: EstimateProper} yields that $\Int \M_{r-1}$ is a coherent $\D_\X$-module with characteristic variety $\widetilde{\mu}((T^*\mu)^{-1}(\Lambda\cup W))$ with $\Lambda$ isotropic and $W$ irreducible of dimension $m+1$ dominating $\Y$.
  Observe that $B^N(\Int\M/ \U)\otimes_{\C[s]}\C[s]/L_{r-1}$ is a subquotient of $\Int\M_{r-1}$ with support in the divisor $D$.
  Hence, $B^N(\Int\M/ \U)\otimes_{\C[s]}\C[s]/L$ is a coherent $\D_\X$-module with
  $$\Ch\left( B^N\left(\Int\M/ \U\right)\otimes_{\C[s]}\C[s]/L_{r-1}) \right)\subseteq  \widetilde{\mu}\left((T^*\mu)^{-1}(\Lambda\cup W)\right) \cap \pi^{-1}(D)$$
  where $\pi:T^*X\to X$ is the projection map.

  This means $B^N(\Int\M/ \U)\otimes_{\C[s]}\C[s]/L_{r-1}$ is a holonomic $\D_\X$-module.
  Indeed, by \cref{lem: IsotropicDirectImage} $\widetilde{\mu}((T^*\mu)^{-1}(\Lambda))$ remains isotropic and forms no obstruction to the characteristic variety being Lagrangian.
  Moreover, $\widetilde{\mu}((T^*\mu)^{-1}(W))$ is irreducible of dimension $m+1$ and dominates $\X$.
  Intersecting with $\pi^{-1}(D)$ yields a closed strict subset which necessarily has lower dimension.
  Hence, it follows that $\dim \Ch B^N (\Int\M/ \U)\otimes_{\C[s]}\C[s]/L_{r-1} \leq m$.
  This means that $B^N(\Int\M/ \U)\otimes_{\C[s]}\C[s]/L_{r-1}$ is holonomic.
  By \cref{prop: HomAlgebraic} the Bernstein-Sato ideal of holonomic module is non-zero.
  This implies that the Bernstein-Sato ideal of $(\Int\M/ \U)\otimes_{\C[s]}\C[s]/L_{r-1}$ is non-zero which means that it has grade $\geq m+1$ by \cref{cor: GradeIFFBernsteinIdeal}.
  Then also $\Int\M/\U$ has grade $\geq m+1$ by \cref{lem: RestrictToCM}.
  \end{proof}
Now all ingredients are in place for the proof of \cref{thm: EstimateBernsteinSatoZeroLocust}.
\begin{theorem}
  With notation as in \cref{sec: IntoductionChapterRelative} every irreducible component of $Z(B_F)$ of codimension $1$ is a hyperplane of the form
  $$\operatorname{mult}_{E_i}(g_1) s_1 + \cdots + \operatorname{mult}_{E_i}(g_r)s_r + k_i + c_i=0$$
  with $c_i \in \mathbb{Z}_{\geq 0 }$.
\end{theorem}
\begin{proof}
  By \cref{lem: GradeNPlusOne} the $\D_X$-module $\M/\U$ has grade greater than or equal to $m + 1$. Hence \cref{lem: StabilisationtN} provides $N\geq 1$ such that $t^N\M/\U = 0$ on $\X\times\Spec R$ for some open $\Spec R\subseteq \C^r$ with complement of codimension strictly greater than $1$.

  Let $b(s_1,\ldots,s_r)$ denote the Bernstein-Sato polynomial for $\M/t\M$ provided by \cref{lem: BernsteinSatoPolynomialUpstairs}.
  Set $B := \prod_{i=0}^{N+1} b(s_1 + i, \ldots, s_r + i)$ then it follows that
  $B\M \subseteq t\U$ on $\X \times \Spec R$.
  In particular this means that $B$ is in the Bernstein-Sato ideal of $\U/t\U$ over $\Spec R$.
  By the surjection of \cref{lem: SurjectionUF} this means that $B\in B_{F,x}$ over $\Spec R$. This proves the theorem because the complement of $\Spec R$ has codimension strictly greater than 1.
\end{proof}
\subsection{Analytical Modifcations}
