\addcontentsline{toc}{chapter}{Summary} %%
\chapter*{Summary}\label{ch: Summary} %%
This thesis is concerned with the estimation of Bernstein-Sato zero loci.\\

\noindent
Historically, this problem arose when trying to extend a function to the entire complex plane.
Let $f\in \mathbb{R}[x_1,\ldots,x_n]$ be some fixed positive polynomial and let $g \in \mathcal{C}_c^\infty$ be some test function.
Gelfand asked if it is possible to find a meromorphic extension of the function
$$\zeta_g(s) = \int_{\mathbb{R}^n} g f^s dx; \qquad \operatorname{Re}(s)>0 $$
to the entire complex plane.
A proof by I.N. Bernstein relies on the existence of a differential operator $P(x,\partial, s)$ such that
$$P(x,\partial,s) f^{s+1} = b_f(s) f^s$$
for some polynomial $b_f(s)\in \C[s]$.
The monic polynomial of minimal degree is called the Bernstein-Sato polynomial.
Further, this method shows that any pole of $\zeta_g$ is a root of $b_f(s)$ up to a shift with a negative integer.

The relation between the roots of $b_f(s)$ and the poles of functions like $\zeta_g$ is the topic of a notable open problem called the monodromy conjecture.
Further, the roots of $b_f(s)$ relate to various invariants of singularities.
This shows that the roots of the Bernstein-Sato polynomial are a worthy topic of study.

Estimation of the roots of the Bernstein-Sato polynomial has been done in terms of data from a resolution of singularities.
The resolution reduces the problem to the case where $f$ is a monomial in which case the Bernstein-Sato polynomial can be computed explicitly.
The main difficulty is to connect the easier problem to the original problem.
This relies on the sheaf-theoretic framework of $\D_X$-modules and their direct images.
Here $\D_X$ denotes the sheaf of differential operators on a space $X$.

The original estimate due to Kashiwara establishes that the roots of $b_f(s)$ are negative rational numbers.
A lower bound for the distance between the largest root and $0$ was established by Lichtin.
\\

\noindent
In this thesis a multivariate generalization of the problem is considered.
Let $F= (f_1,\ldots,f_r)$ be a tuple of complex polynomials $f_i \in \C[x_1,\ldots,x_n]$ and introduce new variables $s_1,\ldots, s_r$.
It is then still known that, for any $a \in \mathbb{Z}_{\geq 0}^r$, there exists a differential operator $P(x,\partial, s)$ such that
$$P(x,\partial,s) f_1^{s_1 + a_1}\cdots f_r^{s_r + a_r} = b(s)f_1^{s_1}\cdots f_r^{s_r}$$
for some polynomial $b(s) \in \C[s_1,\ldots, s_r]$.
The collection of all possible polynomial $b(s)$ form an ideal of $\C[s_1,\ldots,s_r]$ which is called the Bernstein-Sato ideal and is denoted $B_{F}^a$.
The roots of Bernstein-Sato polynomials are generalised by the zero locus $Z(B_F^a)$.

The first purpose of this thesis was to provide a generalisation for the estimate due to Lichtin to this multivariate case.
The main new idea is a reduction argument to reduce the number of $s_i$ to one.
This involves a tensor product which gives rise to error terms in the shape of the $\Tor$-functor.
Homological algebra is used to control these error terms.

We also consider the relation of the Bernstein-Sato zero locus to some invariants of singularities.
Namely, the jumping walls of mixed multiplier ideals and $\LCT$-poltopes. \\

\noindent
The final chapter of this thesis explores how tools from $\D_X$-module theory could be used on certain problems from probability theory.
It is shown that $\D_X$-module theory may be used to derive recursions in the moments of a random variable.
These recursions can then be used to establish tail bounds for random variables.
