\addcontentsline{toc}{chapter}{Introduction} %% CHANGE BACK TO SUMMARY %%
\chapter*{Introduction} %% CHANGE BACK TO SUMMARY %%


%%      CHANGE BACK TO SUMMARY    %%


The purpose of this thesis is provide a improved estimate for the zero locust of Bernstein-Sato ideals.\\

Historically, this problem arose when trying to extend a function to the entire complex plane.
Let $f\in \mathbb{R}[x_1,\ldots,x_n]$ be some fixed positive polynomial and $g \in \mathcal{C}_c^\infty$ be some test function.
Gelfand asked if it is possible to find a meromorphic extension of the function
$$\Gamma_g(s) = \int_{\mathbb{R}^n} g f^s dx; \qquad \operatorname{Re}(s)>0 $$
to the entire complex plane.
A proof by Bernstein relies on the existence of a differential operator $P(x,\partial, s)$ such that
$$P(x,\partial,s) f^{s+1} = b(s) f^s$$
for some polynomial $b(s)\in \C[s]$.
The polynomial of minimal degree is called the Bernstein-Sato polynomial.
In particular, this method shows that any pole of $\Gamma_g$ is a root of $b(s)$ up a shift with a negative integer.\\

Suppose that $f(0)= 0$ and consider $f$ as a function germ $f:(\C^n,0)\to (\C,0)$.
The roots of the Bernstein-Sato polynomial measure the singularity of $f$ at $0$.
This comes from a relation to the eigenvalues of monodromy, which is to say the behaviour of $f^{-1}(t)$ as $t\in \C^\times$ twists around zero.
Further, the relation between roots of $b(s)$ and poles of functions like $\Gamma_g$ is the topic of a notable open problem called the monodromy conjecture.
This shows that the roots of $b(s)$ are a natural topic of study.

Estimation of the roots of the Bernstein-Sato polynomial has been done in terms of data from a resolution of singularities.
The resolution reduces the problem to the case where $f$ is a monomial in which case the Bernstein-Sato polynomial can be computed explicitly.
The main difficulty is to connect the easier problem to the original problem.
This relies on the sheaf-theoretic framework of $\D_X$-modules and their direct images.
Here $\D_X$ denotes the sheaf of differential operators on a space $X$.

The original estimate due to Kashiwara establishes that the roots of $b(s)$ are negative rational numbers.
A lower bound for the distance between the largest root and $0$ was established by Lichtin.
This refined estimate uses a similar methodology to Kashiwara but replaced $f^s$ with the distribution $f^s dx$.
\\

In this thesis a multivariate generalization of the problem is considered.
Let $f_1,\ldots, f_p \in \C[x_1,\ldots, x_n]$ be polynomials and define variables $s_1,\ldots, s_p$.
It is then still known that there exists a differential operator $P(x,\partial, s)$ such that
$$P(x,\partial,s) f_1^{s_1 + 1}\cdots f_p^{s_p + 1} = b(s)f_1^{s_1}\cdots f_p^{s_p}$$
for some polynomial $b(s) \in \C[s_1,\ldots, s_p]$.
The collection of all possible polynomial $b(s)$ form an ideal of $\C[s_1,\ldots,s_p]$ which is called the Bernstein-Sato ideal and is denoted $B_{F}$.
The roots of the Bernstein-Sato ideal are then generalised by the zero locust $Z(B_F)$.

Kashiwara's estimate for the roots of $b(s)$ has been generalised to a estimation of the Bernstein-Sato ideal by Budur et al.
This thesis also generalises the refined estimate due to Lichtin.
There are two main steps.
Firstly, one must check that the properties of the direct images of $\D_X$-modules generalise to $\D_X\langle s_1,\ldots, s_p \rangle$-modules.
Then, a inductive argument is used to reduce the number of $s_i$ to one.
Homological algebra is used to control error terms of the induction process.\\

The first chapter in this thesis gives a overview of various known results for $\D_X$-modules.
This builds up to the Riemann-Hilbert correspondence which provides a broad generalisation for the the equivalence between systems of differential equations and their solutions.
The second chapter generalises notions for $\D_X$-modules to $\D_X\langle s_1,\ldots, s_p \rangle$-modules and contains the proof for the improved estimate of $Z(B_F)$.
